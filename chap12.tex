%% Copyright (c) 2002, 2010 Sam Williams
%% Copyright (c) 2010 Richard M. Stallman
%% Permission is granted to copy, distribute and/or modify this
%% document under the terms of the GNU Free Documentation License,
%% Version 1.3 or any later version published by the Free Software
%% Foundation; with no Invariant Sections, no Front-Cover Texts, and
%% no Back-Cover Texts. A copy of the license is included in the
%% file called ``gfdl.tex''.


\chapter{A Brief Journey through Hacker Hell}
\chaptermark{A Brief Journey}

[RMS: In this chapter my only change is to add a few notes labeled like
this one.]


Richard Stallman stares, unblinking, through the windshield of a rental car, waiting for the light to change as we make our way through downtown Kihei.

The two of us are headed to the nearby town of Pa'ia, where we are scheduled to meet up with some software programmers and their wives for dinner in about an hour or so.

It's about two hours after Stallman's speech at the Maui High Performance Center, and Kihei, a town that seemed so inviting before the speech, now seems profoundly uncooperative. Like most beach cities, Kihei is a one-dimensional exercise in suburban sprawl. Driving down its main drag, with its endless succession of burger stands, realty agencies, and bikini shops, it's hard not to feel like a steel-coated morsel passing through the alimentary canal of a giant commercial tapeworm. The feeling is exacerbated by the lack of side roads. With nowhere to go but forward, traffic moves in spring-like lurches. 200 yards ahead, a light turns green. By the time we are moving, the light is yellow again.

For Stallman, a lifetime resident of the east coast, the prospect of spending the better part of a sunny Hawaiian afternoon trapped in slow traffic is enough to trigger an embolism. [RMS: Since I was driving, I was also losing time to answer my email, and that's a real pain since I can barely keep up anyway.] Even worse is the knowledge that, with just a few quick right turns a quarter mile back, this whole situation easily could have been avoided. Unfortunately, we are at the mercy of the driver ahead of us, a programmer from the lab who knows the way and who has decided to take us to Pa'ia via the scenic route instead of via the nearby Pilani Highway.

``This is terrible,'' says Stallman between frustrated sighs. ``Why didn't we take the other route?''

Again, the light a quarter mile ahead of us turns green. Again, we creep forward a few more car lengths. This process continues for another 10 minutes, until we finally reach a major crossroad promising access to the adjacent highway.

The driver ahead of us ignores it and continues through the intersection.

``Why isn't he turning?'' moans Stallman, throwing up his hands in frustration. ``Can you believe this?''

I decide not to answer either. I find the fact that I am sitting in a car with Stallman in the driver seat, in Maui no less, unbelievable enough. Until two hours ago, I didn't even know Stallman knew how to drive. Now, listening to Yo-Yo Ma's cello playing the mournful bass notes of ``Appalachian Journey'' on the car stereo and watching the sunset pass by on our left, I do my best to fade into the upholstery.

When the next opportunity to turn finally comes up, Stallman hits his right turn signal in an attempt to cue the driver ahead of us. No such luck. Once again, we creep slowly through the intersection, coming to a stop a good 200 yards before the next light. By now, Stallman is livid.

``It's like he's deliberately ignoring us,'' he says, gesturing and pantomiming like an air craft carrier landing-signals officer in a futile attempt to catch our guide's eye. The guide appears unfazed, and for the next five minutes all we see is a small portion of his head in the rearview mirror.

I look out Stallman's window. Nearby Kahoolawe and Lanai Islands provide an ideal frame for the setting sun. It's a breathtaking view, the kind that makes moments like this a bit more bearable if you're a Hawaiian native, I suppose. I try to direct Stallman's attention to it, but Stallman, by now obsessed by the inattentiveness of the driver ahead of us, blows me off.

When the driver passes through another green light, completely ignoring a ``Pilani Highway Next Right,'' I grit my teeth. I remember an early warning relayed to me by BSD programmer Keith Bostic. ``Stallman does not suffer fools gladly,'' Bostic warned me. ``If somebody says or does something stupid, he'll look them in the eye and say, `That's stupid.'\hspace{0.01in}''

Looking at the oblivious driver ahead of us, I realize that it's the stupidity, not the inconvenience, that's killing Stallman right now.

``It's as if he picked this route with absolutely no thought on how to get there efficiently,'' Stallman says.

The word ``efficiently'' hangs in the air like a bad odor. Few things irritate the hacker mind more than inefficiency. It was the inefficiency of checking the Xerox laser printer two or three times a day that triggered Stallman's initial inquiry into the printer source code. It was the inefficiency of rewriting software tools hijacked by commercial software vendors that led Stallman to battle Symbolics and to launch the GNU Project. If, as Jean Paul Sartre once opined, hell is other people, hacker hell is duplicating other people's stupid mistakes, and it's no exaggeration to say that Stallman's entire life has been an attempt to save mankind from these fiery depths.

This hell metaphor becomes all the more apparent as we take in the slowly passing scenery. With its multitude of shops, parking lots, and poorly timed street lights, Kihei seems less like a city and more like a poorly designed software program writ large. Instead of rerouting traffic and distributing vehicles through side streets and expressways, city planners have elected to run everything through a single main drag. From a hacker perspective, sitting in a car amidst all this mess is like listening to a CD rendition of nails on a chalkboard at full volume.

``Imperfect systems infuriate hackers,'' observes Steven Levy, another warning I should have listened to before climbing into the car with Stallman. ``This is one reason why hackers generally hate driving cars -- the system of randomly programmed red lights and oddly laid out one-way streets causes delays which are so goddamn \textit{unnecessary} [Levy's emphasis] that the impulse is to rearrange signs, open up traffic-light control boxes . . . redesign the entire system.''\endnote{See Steven Levy, \textit{Hackers} (Penguin USA [paperback], 1984): 40.}

More frustrating, however, is the duplicity of our trusted guide. Instead of searching out a clever shortcut -- as any true hacker would do on instinct -- the driver ahead of us has instead chosen to play along with the city planners' game. Like Virgil in Dante's \textit{Inferno}, our guide is determined to give us the full guided tour of this hacker hell whether we want it or not.

Before I can make this observation to Stallman, the driver finally hits his right turn signal. Stallman's hunched shoulders relax slightly, and for a moment the air of tension within the car dissipates. The tension comes back, however, as the driver in front of us slows down. ``Construction Ahead'' signs line both sides of the street, and even though the Pilani Highway lies less than a quarter mile off in the distance, the two-lane road between us and the highway is blocked by a dormant bulldozer and two large mounds of dirt.

It takes Stallman a few seconds to register what's going on as our guide begins executing a clumsy five-point U-turn in front of us. When he catches a glimpse of the bulldozer and the ``No Through Access'' signs just beyond, Stallman finally boils over.

``Why, why, why?'' he whines, throwing his head back. ``You should have known the road was blocked. You should have known this way wouldn't work. You did this deliberately.''  [RMS: I meant that he chose the slow road deliberately.  As explained below, I think these quotes are not exact.]

The driver finishes the turn and passes us on the way back toward the main drag. As he does so, he shakes his head and gives us an apologetic shrug. Coupled with a toothy grin, the driver's gesture reveals a touch of mainlander frustration but is tempered with a protective dose of islander fatalism. Coming through the sealed windows of our rental car, it spells out a succinct message: ``Hey, it's Maui; what are you gonna do?''

Stallman can take it no longer.

``Don't you fucking smile!'' he shouts, fogging up the glass as he does so. ``It's your fucking fault. This all could have been so much easier if we had just done it my way.'' [RMS: These quotes appear to be inaccurate, because I don't use ``fucking'' as an adverb.  This was not an interview, so Williams would not have had a tape recorder running.  I'm sure things happened overall as described, but these quotations probably reflect his understanding rather than my words.]

Stallman accents the words ``my way'' by gripping the steering wheel and pulling himself towards it twice. The image of Stallman's lurching frame is like that of a child throwing a temper tantrum in a car seat, an image further underlined by the tone of Stallman's voice. Halfway between anger and anguish, Stallman seems to be on the verge of tears.

Fortunately, the tears do not arrive. Like a summer cloudburst, the tantrum ends almost as soon as it begins. After a few whiny gasps, Stallman shifts the car into reverse and begins executing his own U-turn. By the time we are back on the main drag, his face is as impassive as it was when we left the hotel 30 minutes earlier.

It takes less than five minutes to reach the next cross-street. This one offers easy highway access, and within seconds, we are soon speeding off toward Pa'ia at a relaxing rate of speed. The sun that once loomed bright and yellow over Stallman's left shoulder is now burning a cool orange-red in our rearview mirror. It lends its color to the gauntlet wili wili trees flying past us on both sides of the highway.

For the next 20 minutes, the only sound in our vehicle, aside from the ambient hum of the car's engine and tires, is the sound of a cello and a violin trio playing the mournful strains of an Appalachian folk tune.

\theendnotes
\setcounter{endnote}{0}
