%% Copyright (c) 2002, 2010 Sam Williams
%% Copyright (c) 2010 Richard M. Stallman
%% Permission is granted to copy, distribute and/or modify this
%% document under the terms of the GNU Free Documentation License,
%% Version 1.3 or any later version published by the Free Software
%% Foundation; with no Invariant Sections, no Front-Cover Texts, and
%% no Back-Cover Texts. A copy of the license is included in the
%% file called ``gfdl.tex''.

\chapter{2001: A Hacker's Odyssey}

The New York University computer-science department sits inside Warren Weaver Hall, a fortress-like building located two blocks east of Washington Square Park. Industrial-strength air-conditioning vents create a surrounding moat of hot air, discouraging loiterers and solicitors alike. Visitors who breach the moat encounter another formidable barrier, a security check-in counter immediately inside the building's single entryway.

Beyond the security checkpoint, the atmosphere relaxes somewhat. Still, numerous signs scattered throughout the first floor preach the dangers of unsecured doors and propped-open fire exits. Taken as a whole, the signs offer a reminder: even in the relatively tranquil confines of pre-September 11, 2001, New York, one can never be too careful or too suspicious.

The signs offer an interesting thematic counterpoint to the growing number of visitors gathering in the hall's interior atrium. A few look like NYU students. Most look like shaggy-haired concert-goers milling outside a music hall in anticipation of the main act. For one brief morning, the masses have taken over Warren Weaver Hall, leaving the nearby security attendant with nothing better to do but watch Ricki Lake on TV and shrug her shoulders toward the nearby auditorium whenever visitors ask about ``the speech.''

Once inside the auditorium, a visitor finds the person who has forced this temporary shutdown of building security procedures. The person is Richard M. Stallman, founder of the GNU Project, original president of the Free Software Foundation, winner of the 1990 MacArthur Fellowship, winner of the Association of Computing Machinery's Grace Murray Hopper Award (also in 1990), corecipient of the Takeda Foundation's 2001 Takeda Award for Social/Economic Betterment, and former AI Lab hacker. As announced over a host of hacker-related web sites, including the GNU Project's own \url{http://www.gnu.org} site, Stallman is in Manhattan, his former hometown, to deliver a much anticipated speech in rebuttal to the Microsoft Corporation's recent campaign against the GNU General Public License.

The subject of Stallman's speech is the history and future of the free software movement. The location is significant. Less than a month before, Microsoft senior vice president Craig Mundie appeared at the nearby NYU Stern School of Business, delivering a speech blasting the GNU General Public License, or GNU GPL, a legal device originally conceived by Stallman 16 years before. Built to counteract the growing wave of software secrecy overtaking the computer industry -- a wave first noticed by Stallman during his 1980 troubles with the Xerox laser printer -- the GPL has evolved into a central tool of the free software community. In simplest terms, the GPL establishes a form of communal ownership -- what today's legal scholars now call the ``digital commons'' -- through the legal weight of copyright. The GPL makes this irrevocable; once an author gives code to the community in this way, that code can't be made proprietary by anyone else. Derivative versions must carry the same copyright license, if they use a substantial amount of the original source code. For this reason, critics of the GPL have taken to calling it a ``viral'' license, suggesting inaccurately that it spreads itself to every software program it touches.\endnote{Actually, the GPL's powers are not quite that potent: just putting your code in the same computer with a GPL-covered program does not put your code under the GPL.

``To compare something to a virus is very harsh,'' says Stallman. ``A spider plant is a more accurate comparison; it goes to another place if you actively take a cutting.''

For more information on the GNU General Public License, visit \url{http://www.gnu.org/copyleft/gpl.html}.}

In an information economy increasingly dependent on software and increasingly beholden to software standards, the GPL has become the proverbial ``big stick.'' Even companies that once derided it as ``software socialism'' have come around to recognize the benefits. Linux, the kernel developed by Finnish college student Linus Torvalds in 1991, is licensed under the GPL, as are most parts of the GNU system: GNU Emacs, the GNU Debugger, the GNU C Compiler, etc. Together, these tools form the components of the free software GNU/Linux operating system, developed, nurtured, and owned by the worldwide hacker community. Instead of viewing this community as a threat, high-tech companies like IBM, Hewlett Packard, and Sun Microsystems have come to rely upon it, selling software applications and services built to ride atop the ever-growing free software infrastructure.\endnote{Although these applications run on GNU/Linux, it does not follow that they are themselves free software.  On the contrary, most of them applications are proprietary software, and respect your freedom no more than Windows does.  They may contribute to the success of GNU/Linux, but they don't contribute to the goal of freedom for which it exists.}

They've also come to rely upon it as a strategic weapon in the hacker community's perennial war against Microsoft, the Redmond, Washington-based company that has dominated the PC-software marketplace since the late 1980s. As owner of the popular Windows operating system, Microsoft stands to lose the most in an industry-wide shift to the GPL license. Each program in the Windows colossus is covered by copyrights and contracts (End User License Agreements, or EULAs) asserting the proprietary status of the executable, as well as the underlying source code that users can't get anyway. Incorporating code protected by the ``viral'' GPL into one of these programs is forbidden; to comply with the GPL's requirements, Microsoft would be legally required to make that whole program free software. Rival companies could then copy, modify, and sell improved versions of it, taking away the basis of Microsoft's lock over the users.

Hence the company's growing concern over the GPL's rate of adoption. Hence the recent Mundie speech blasting the GPL and the ``open source'' approach to software development and sales. (Microsoft does not even acknowledge the term ``free software,'' preferring to use its attacks to direct attention towards the apolitical ``open source'' camp described in \autoref{chapter:open source}, and away from the free software movement.) And hence Stallman's decision to deliver a public rebuttal to that speech on the same campus here today.

20 years is a long time in the software industry. Consider this: in 1980, when Richard Stallman was cursing the AI Lab's Xerox laser printer, Microsoft, which dominates the worldwide software industry, was still a privately held startup. IBM, the company then regarded as the most powerful force in the computer hardware industry, had yet to introduce its first personal computer, thereby igniting the current low-cost PC market. Many of the technologies we now take for granted -- the World Wide Web, satellite television, 32-bit video-game consoles -- didn't even exist. The same goes for many of the companies that now fill the upper echelons of the corporate establishment, companies like AOL, Sun Microsystems, Amazon.com, Compaq, and Dell. The list goes on and on.

Among those who value progress above freedom,
the fact that the high-technology marketplace has come so far in such little time is cited both for and against the GNU GPL.  Some argue in favor of the GPL, pointing to the short lifespan of most computer hardware platforms. Facing the risk of buying an obsolete product, consumers tend to flock to companies with the best long-term survival. As a result, the software marketplace has become a winner-take-all arena.\endnote{See  Shubha Ghosh, ``Revealing the Microsoft Windows Source Code,'' \textit{Gigalaw.com} (January, 2000), \url{http://www.gigalaw.com/}.} The proprietary software environment, they say, leads to monopoly abuse and stagnation. Strong companies suck all the oxygen out of the marketplace for rival competitors and innovative startups.

Others argue just the opposite. Selling software is just as risky, if not more risky, than buying software, they say. Without the legal guarantees provided by proprietary software licenses, not to mention the economic prospects of a privately owned ``killer app'' (i.e., a breakthrough technology that launches an entirely new market),\endnote{Killer apps don't have to be proprietary. Still, I think the reader gets the point: the software marketplace is like the lottery. The bigger the potential payoff, the more people want to participate. For a good summary of the killer-app phenomenon, see Philip Ben-David, ``Whatever Happened to the `Killer App'?'', \textit{e-Commerce News} (December 7, 2000), \url{http://www.ecommercetimes.com/story/5893.html}.} companies lose the incentive to participate. Once again, the market stagnates and innovation declines. As Mundie himself noted in his May 3rd address on the same campus, the GPL's ``viral'' nature ``poses a threat'' to any company that relies on the uniqueness of its software as a competitive asset. Added Mundie:

\begin{quote}
It also fundamentally undermines the independent commercial software sector because it effectively makes it impossible to distribute software on a basis where recipients pay for the product rather than just the cost of distribution.\endnote{See Craig Mundie, ``The Commercial Software Model,'' senior vice president, Microsoft Corp., excerpted from an online transcript of Mundie's May 3, 2001, speech to the New York University Stern School of Business, \url{http://www.microsoft.com/presspass/exec/craig/05-03sharedsource.asp}.}
\end{quote}

The mutual success of GNU/Linux and Windows over the last 10 years suggests that both sides on this question are sometimes right.  However, free software activists such as Stallman think this is a side issue.   The real question, they say, isn't whether free or proprietary software will succeed more, it's which one is more ethical.

Nevertheless, the battle for momentum is an important one in the software industry. Even powerful vendors such as Microsoft rely on the support of third-party software developers whose tools, programs, and computer games make an underlying software platform such as Windows more attractive to the mainstream consumer. Citing the rapid evolution of the technology marketplace over the last 20 years, not to mention his own company's impressive track record during that period, Mundie advised listeners to not get too carried away by the free software movement's recent momentum:

\begin{quote}
Two decades of experience have shown that an economic model that protects intellectual property and a business model that recoups research and development costs can create impressive economic benefits and distribute them very broadly.\endnote{\textit{Ibid.}}
\end{quote}

Such admonitions serve as the backdrop for Stallman's speech today. Less than a month after their utterance, Stallman stands with his back to one of the chalk boards at the front of the room, edgy to begin.

If the last two decades have brought dramatic changes to the software marketplace, they have brought even more dramatic changes to Stallman himself. Gone is the skinny, clean-shaven hacker who once spent his entire days communing with his beloved PDP-10. In his place stands a heavy-set middle-aged man with long hair and rabbinical beard, a man who now spends the bulk of his time writing and answering email, haranguing fellow programmers, and giving speeches like the one today. Dressed in an aqua-colored T-shirt and brown polyester pants, Stallman looks like a desert hermit who just stepped out of a Salvation Army dressing room.

The crowd is filled with visitors who share Stallman's fashion and grooming tastes. Many come bearing laptop computers and cellular modems, all the better to record and transmit Stallman's words to a waiting Internet audience. The gender ratio is roughly 15 males to 1 female, and 1 of the 7 or 8 females in the room comes in bearing a stuffed penguin, the official Linux mascot, while another carries a stuffed teddy bear.

Agitated, Stallman leaves his post at the front of the room and takes a seat in a front-row chair, tapping commands into an already-opened laptop. For the next 10 minutes Stallman is oblivious to the growing number of students, professors, and fans circulating in front of him at the foot of the auditorium stage.

Before the speech can begin, the baroque rituals of academic formality must be observed. Stallman's appearance merits not one but two introductions. Mike Uretsky, codirector of the Stern School's Center for Advanced Technology, provides the first.

``The role of a university is to foster debate and to have interesting discussions,'' Uretsky says. ``This particular presentation, this seminar falls right into that mold. I find the discussion of open source particularly interesting.''

Before Uretsky can get another sentence out, Stallman is on his feet waving him down like a stranded motorist.

``I do free software,'' Stallman says to rising laughter. ``Open source is a different movement.''

The laughter gives way to applause. The room is stocked with Stallman partisans, people who know of his reputation for verbal exactitude, not to mention his much publicized 1998 falling out with the open source software proponents. Most have come to anticipate such outbursts the same way radio fans once waited for Jack Benny's trademark, ``Now cut that out!'' phrase during each radio program.

Uretsky hastily finishes his introduction and cedes the stage to Edmond Schonberg, a professor in the NYU computer-science department. As a computer programmer and GNU Project contributor, Schonberg knows which linguistic land mines to avoid. He deftly summarizes Stallman's career from the perspective of a modern-day programmer.

``Richard is the perfect example of somebody who, by acting locally, started thinking globally [about] problems concerning the unavailability of source code,'' says Schonberg. ``He has developed a coherent philosophy that has forced all of us to reexamine our ideas of how software is produced, of what intellectual property means, and of what the software community actually represents.''\endnote{If this were to be said today, Stallman would object to the term ``intellectual property'' as carrying bias and confusion. See \url{http://www.gnu.org/philosophy/not-ipr.html}.}

Schonberg welcomes Stallman to more applause. Stallman takes a moment to shut off his laptop, rises out of his chair, and takes the stage.

At first, Stallman's address seems more Catskills comedy routine than political speech. ``I'd like to thank Microsoft for providing me the opportunity to be on this platform,'' Stallman wisecracks. ``For the past few weeks, I have felt like an author whose book was fortuitously banned somewhere.''

For the uninitiated, Stallman dives into a quick free software warm-up analogy. He likens a software program to a cooking recipe. Both provide useful step-by-step instructions on how to complete a desired task and can be easily modified if a user has special desires or circumstances. ``You don't have to follow a recipe exactly,'' Stallman notes. ``You can leave out some ingredients. Add some mushrooms, 'cause you like mushrooms. Put in less salt because your doctor said you should cut down on salt -- whatever.''

Most importantly, Stallman says, software programs and recipes are both easy to share. In giving a recipe to a dinner guest, a cook loses little more than time and the cost of the paper the recipe was written on. Software programs require even less, usually a few mouse-clicks and a modicum of electricity. In both instances, however, the person giving the information gains two things: increased friendship and the ability to borrow interesting recipes in return.

``Imagine what it would be like if recipes were packaged inside black boxes,'' Stallman says, shifting gears. ``You couldn't see what ingredients they're using, let alone change them, and imagine if you made a copy for a friend. They would call you a pirate and try to put you in prison for years. That world would create tremendous outrage from all the people who are used to sharing recipes. But that is exactly what the world of proprietary software is like. A world in which common decency towards other people is prohibited or prevented.''

With this introductory analogy out of the way, Stallman launches into a retelling of the Xerox laser-printer episode. Like the recipe analogy, the laser-printer story is a useful rhetorical device. With its parable-like structure, it dramatizes just how quickly things can change in the software world. Drawing listeners back to an era before Amazon.com one-click shopping, Microsoft Windows, and Oracle databases, it asks the listener to examine the notion of software ownership free of its current corporate logos.

Stallman delivers the story with all the polish and practice of a local district attorney conducting a closing argument. When he gets to the part about the Carnegie Mellon professor refusing to lend him a copy of the printer source code, Stallman pauses.

``He had betrayed us,'' Stallman says. ``But he didn't just do it to us. Chances are he did it to you.''

On the word ``you,'' Stallman points his index finger accusingly at an unsuspecting member of the audience. The targeted audience member's eyebrows flinch slightly, but Stallman's own eyes have moved on. Slowly and deliberately, Stallman picks out a second listener to nervous titters from the crowd. ``And I think, mostly likely, he did it to you, too,'' he says, pointing at an audience member three rows behind the first.

By the time Stallman has a third audience member picked out, the titters have given away to general laughter. The gesture seems a bit staged, because it is. Still, when it comes time to wrap up the Xerox laser-printer story, Stallman does so with a showman's flourish. ``He probably did it to most of the people here in this room -- except a few, maybe, who weren't born yet in 1980,'' Stallman says, drawing more laughs. ``[That's] because he had promised to refuse to cooperate with just about the entire population of the planet Earth.''

Stallman lets the comment sink in for a half-beat. ``He had signed a nondisclosure agreement,'' Stallman adds.

Richard Matthew Stallman's rise from frustrated academic to political leader over the last 20 years speaks to many things. It speaks to Stallman's stubborn nature and prodigious will. It speaks to the clearly articulated vision and values of the free software movement Stallman helped build. It speaks to the high-quality software programs Stallman has built, programs that have cemented Stallman's reputation as a programming legend. It speaks to the growing momentum of the GPL, a legal innovation that many Stallman observers see as his most momentous accomplishment.

Most importantly, it speaks to the changing nature of political power in a world increasingly beholden to computer technology and the software programs that power that technology.

Maybe that's why, even at a time when most high-technology stars are on the wane, Stallman's star has grown. Since launching the GNU Project in 1984,\endnote{The acronym GNU stands for ``GNU's not Unix.'' In another portion of the May 29, 2001, NYU speech, Stallman summed up the acronym's origin:

\begin{quote}
We hackers always look for a funny or naughty name for a program, because naming a program is half the fun of writing the program. We also had a tradition of recursive acronyms, to say that the program that you're writing is similar to some existing program\ldots I looked for a recursive acronym for Something Is Not UNIX. And I tried all 26 letters and discovered that none of them was a word. I decided to make it a contraction. That way I could have a three-letter acronym, for Something's Not UNIX. And I tried letters, and I came across the word ``GNU.'' That was it.

Although a fan of puns, Stallman recommends that software users pronounce the ``g'' at the beginning of the acronym (i.e., ``gah-new''). Not only does this avoid confusion with the word ``gnu,'' the name of the African antelope, Connochaetes gnou, it also avoids confusion with the adjective ``new.'' ``We've been working on it for 17 years now, so it is not exactly new any more,'' Stallman says.
\end{quote}

Source: author notes and online transcript of ``Free Software: Freedom and Cooperation,'' Richard Stallman's May 29, 2001, speech at New York University, \url{http://www.gnu.org/events/rms-nyu-2001-transcript.txt}.} Stallman has been at turns ignored, satirized, vilified, and attacked--both from within and without the free software movement. Through it all, the GNU Project has managed to meet its milestones, albeit with a few notorious delays, and stay relevant in a software marketplace several orders of magnitude more complex than the one it entered 18 years ago. So too has the free software ideology, an ideology meticulously groomed by Stallman himself.

To understand the reasons behind this currency, it helps to examine Richard Stallman both in his own words and in the words of the people who have collaborated and battled with him along the way. The Richard Stallman character sketch is not a complicated one. If any person exemplifies the old adage ``what you see is what you get,'' it's Stallman.

``I think if you want to understand Richard Stallman the human being, you really need to see all of the parts as a consistent whole,'' advises Eben Moglen, legal counsel to the Free Software Foundation and professor of law at Columbia University Law School. ``All those personal eccentricities that lots of people see as obstacles to getting to know Stallman really `are' Stallman: Richard's strong sense of personal frustration, his enormous sense of principled ethical commitment, his inability to compromise, especially on issues he considers fundamental. These are all the very reasons Richard did what he did when he did.''

Explaining how a journey that started with a laser printer would eventually lead to a sparring match with the world's richest corporation is no easy task. It requires a thoughtful examination of the forces that have made software ownership so important in today's society. It also requires a thoughtful examination of a man who, like many political leaders before him, understands the malleability of human memory. It requires an ability to interpret the myths and politically laden code words that have built up around Stallman over time. Finally, it requires an understanding of Stallman's genius as a programmer and his failures and successes in translating that genius to other pursuits.

When it comes to offering his own summary of the journey, Stallman acknowledges the fusion of personality and principle observed by Moglen. ``Stubbornness is my strong suit,'' he says. ``Most people who attempt to do anything of any great difficulty eventually get discouraged and give up. I never gave up.''

He also credits blind chance. Had it not been for that run-in over the Xerox laser printer, had it not been for the personal and political conflicts that closed out his career as an MIT employee, had it not been for a half dozen other timely factors, Stallman finds it very easy to picture his life following a different career path. That being said, Stallman gives thanks to the forces and circumstances that put him in the position to make a difference.

``I had just the right skills,'' says Stallman, summing up his decision for launching the GNU Project to the audience. ``Nobody was there but me, so I felt like, `I'm elected. I have to work on this. If not me, who?'\hspace{0.01in}''

\theendnotes
\setcounter{endnote}{0}
