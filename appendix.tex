%% Copyright (c) 2002, 2010 Sam Williams
%% Copyright (c) 2010 Richard M. Stallman
%% Permission is granted to copy, distribute and/or modify this
%% document under the terms of the GNU Free Documentation License,
%% Version 1.3 or any later version published by the Free Software
%% Foundation; with no Invariant Sections, no Front-Cover Texts, and
%% no Back-Cover Texts. A copy of the license is included in the
%% file called ``gfdl.tex''.

\chapter{Appendix A -- Hack, Hackers, and Hacking} \label{Appendix A}

To understand the full meaning of the word ``hacker,'' it helps to examine the word's etymology over the years.

\textit{The New Hacker Dictionary}, an online compendium of software-programmer jargon, officially lists nine different connotations of the word ``hack'' and a similar number for ``hacker.'' Then again, the same publication also includes an accompanying essay that quotes Phil Agre, an MIT hacker who warns readers not to be fooled by the word's perceived flexibility. ``Hack has only one meaning,'' argues Agre. ``An extremely subtle and profound one which defies articulation.''  Richard Stallman tries to articulate it with the phrase, ``Playful cleverness.''

Regardless of the width or narrowness of the definition, most modern hackers trace the word back to MIT, where the term bubbled up as popular item of student jargon in the early 1950s. In 1990 the MIT Museum put together a journal documenting the hacking phenomenon. According to the journal, students who attended the institute during the fifties used the word ``hack'' the way a modern student might use the word ``goof.'' Hanging a jalopy out a dormitory window was a ``hack,'' but anything harsh or malicious -- e.g., egging a rival dorm's windows or defacing a campus statue -- fell outside the bounds. Implicit within the definition of ``hack'' was a spirit of harmless, creative fun.

This spirit would inspire the word's gerund form: ``hacking.'' A 1950s student who spent the better part of the afternoon talking on the phone or dismantling a radio might describe the activity as ``hacking.'' Again, a modern speaker would substitute the verb form of ``goof'' -- ``goofing'' or ``goofing off'' -- to describe the same activity.

As the 1950s progressed, the word ``hack'' acquired a sharper, more rebellious edge. The MIT of the 1950s was overly competitive, and hacking emerged as both a reaction to and extension of that competitive culture. Goofs and pranks suddenly became a way to blow off steam, thumb one's nose at campus administration, and indulge creative thinking and behavior stifled by the Institute's rigorous undergraduate curriculum. With its myriad hallways and underground steam tunnels, the Institute offered plenty of exploration opportunities for the student undaunted by locked doors and ``No Trespassing'' signs. Students began to refer to their off-limits explorations as ``tunnel hacking.'' Above ground, the campus phone system offered similar opportunities. Through casual experimentation and due diligence, students learned how to perform humorous tricks. Drawing inspiration from the more traditional pursuit of tunnel hacking, students quickly dubbed this new activity ``phone hacking.''

The combined emphasis on creative play and restriction-free exploration would serve as the basis for the future mutations of the hacking term. The first self-described computer hackers of the 1960s MIT campus originated from a late 1950s student group called the Tech Model Railroad Club. A tight clique within the club was the Signals and Power (S\&P) Committee -- the group behind the railroad club's electrical circuitry system. The system was a sophisticated assortment of relays and switches similar to the kind that controlled the local campus phone system. To control it, a member of the group simply dialed in commands via a connected phone and watched the trains do his bidding.

The nascent electrical engineers responsible for building and maintaining this system saw their activity as similar in spirit to phone hacking. Adopting the hacking term, they began refining it even further. From the S\&P hacker point of view, using one less relay to operate a particular stretch of track meant having one more relay for future play. Hacking subtly shifted from a synonym for idle play to a synonym for idle play that improved the overall performance or efficiency of the club's railroad system at the same time. Soon S\&P committee members proudly referred to the entire activity of improving and reshaping the track's underlying circuitry as ``hacking'' and to the people who did it as ``hackers.''

Given their affinity for sophisticated electronics -- not to mention the traditional MIT-student disregard for closed doors and ``No Trespassing'' signs -- it didn't take long before the hackers caught wind of a new machine on campus. Dubbed the TX-0, the machine was one of the first commercially marketed computers. By the end of the 1950s, the entire S\&P clique had migrated en masse over to the TX-0 control room, bringing the spirit of creative play with them. The wide-open realm of computer programming would encourage yet another mutation in etymology. ``To hack'' no longer meant soldering unusual looking circuits, but cobbling together software programs with little regard to ``official'' methods or software-writing procedures. It also meant improving the efficiency and speed of already-existing programs that tended to hog up machine resources. True to the word's roots, it also meant writing programs that served no other purpose than to amuse or entertain.

A classic example of this expanded hacking definition is the game Spacewar, the first computer-based video game. Developed by MIT hackers in the early 1960s, Spacewar had all the traditional hacking definitions: it was goofy and random, serving little useful purpose other than providing a nightly distraction for the dozen or so hackers who delighted in playing it. From a software perspective, however, it was a monumental testament to innovation of programming skill. It was also completely free. Because hackers had built it for fun, they saw no reason to guard their creation, sharing it extensively with other programmers. By the end of the 1960s, Spacewar had become a diversion for programmers around the world, if they had the (then rather rare) graphical displays.

This notion of collective innovation and communal software ownership distanced the act of computer hacking in the 1960s from the tunnel hacking and phone hacking of the 1950s. The latter pursuits tended to be solo or small-group activities. Tunnel and phone hackers relied heavily on campus lore, but the off-limits nature of their activity discouraged the open circulation of new discoveries. Computer hackers, on the other hand, did their work amid a scientific field biased toward collaboration and the rewarding of innovation. Hackers and ``official'' computer scientists weren't always the best of allies, but in the rapid evolution of the field, the two species of computer programmer evolved a cooperative -- some might say symbiotic -- relationship.

Hackers had little respect for bureaucrats' rules.  They regarded computer security systems that obstructed access to the machine as just another bug, to be worked around or fixed if possible.  Thus, breaking security (but not for malicious purposes) was a recognized aspect of hacking in 1970, useful for practical jokes (the victim might say, ``I think someone's hacking me'') as well as for gaining access to the computer.  But it was not central to the idea of hacking.  Where there was a security obstacle, hackers were proud to display their wits in surmounting it; however, given the choice, as at the MIT AI Lab, they chose to have no obstacle and do other kinds of hacking.  Where there is no security, nobody needs to break it.

It is a testament to the original computer hackers' prodigious skill that later programmers, including Richard M. Stallman, aspired to wear the same hacker mantle. By the mid to late 1970s, the term ``hacker'' had acquired elite connotations. In a general sense, a computer hacker was any person who wrote software code for the sake of writing software code. In the particular sense, however, it was a testament to programming skill. Like the term ``artist,'' the meaning carried tribal overtones. To describe a fellow programmer as a hacker was a sign of respect. To describe oneself as a hacker was a sign of immense personal confidence. Either way, the original looseness of the computer-hacker appellation diminished as computers became more common.

As the definition tightened, ``computer'' hacking acquired additional semantic overtones.  The hackers at the MIT AI Lab shared many other characteristics, including love of Chinese food, disgust for tobacco smoke, and avoidance of alcohol, tobacco and other addictive drugs.  These characteristics became part of some people's understanding of what it meant to be a hacker, and the community exerted an influence on newcomers even though it did not demand conformity.  However, these cultural associations disappeared with the AI Lab hacker community.  Today, most hackers resemble the surrounding society on these points.

As the hackers at elite institutions such as MIT, Stanford, and Carnegie Mellon conversed about hacks they admired, they also considered the ethics of their activity, and began to speak openly of a ``hacker ethic'': the yet-unwritten rules that governed a hacker's day-to-day behavior. In the 1984 book \textit{Hackers}, author Steven Levy, after much research and consultation, codified the hacker ethic as five core hacker tenets.

In the 1980s, computer use expanded greatly, and so did security breaking.  Mostly it was done by insiders with criminal intent, who were generally not hackers at all.  However, occasionally the police and administrators, who defined disobedience as evil, traced a computer ``intrusion'' back to a hacker whose idea of ethics was ``Don't hurt people.''   Journalists published articles in which ``hacking'' meant breaking security, and usually endorsed the administrators' view of the matter.  Although books like \textit{Hackers} did much to document the original spirit of exploration that gave rise to the hacking culture, for most newspaper reporters and readers the term ``computer hacker'' became a synonym for ``electronic burglar.''

By the late 1980s, many U.S. teenagers had access to computers.  Some were alienated from society; inspired by journalists' distorted picture of ``hacking,'' they expressed their resentment by breaking computer security much as other alienated teens might have done it by breaking windows. They began to call themselves ``hackers,'' but they never learned the MIT hackers' principle against malicious behavior. As younger programmers began employing their computer skills to harmful ends -- creating and disseminating computer viruses, breaking into computer systems for mischief, deliberately causing computers to crash -- the term ``hacker'' acquired a punk, nihilistic edge which attracted more people with similar attitudes.

Hackers have railed against this perceived misusage of their self-designator for nearly two decades.  Stallman, not one to take things lying down, coined the term ``cracking'' for ``security breaking'' so that people could more easily avoid calling it ``hacking.''  But the distinction between hacking and cracking is often misunderstood. These two descriptive terms are not meant to be exclusive.  It's not that ``Hacking is here, and cracking is there, and never the twain shall meet.''  Hacking and cracking are different attributes of activities, just as ``young'' and ``tall'' are different attributes of persons.

Most hacking does not involve security, so it is not cracking.  Most cracking is done for profit or malice and not in a playful spirit, so it is not hacking.  Once in a while a single act may qualify as cracking and as hacking, but that is not the usual case.  The hacker spirit includes irreverence for rules, but most hacks do not break rules.  Cracking is by definition disobedience, but it is not necessarily malicious or harmful.  The computer security field distinguishes between ``black hat'' and ``white hat'' crackers -- i.e., crackers who turn toward destructive, malicious ends versus those who probe security in order to fix it.

The hacker's central principle not to be malicious remains the primary cultural link between the notion of hacking in the early 21st century and hacking in the 1950s. It is important to note that, as the idea of computer hacking has evolved over the last four decades, the original notion of hacking -- i.e., performing pranks or exploring underground tunnels -- remains intact. In the fall of 2000, the MIT Museum paid tribute to the Institute's age-old hacking tradition with a dedicated exhibit, the Hall of Hacks. The exhibit includes a number of photographs dating back to the 1920s, including one involving a mock police cruiser. In 1993, students paid homage to the original MIT notion of hacking by placing the same police cruiser, lights flashing, atop the Institute's main dome. The cruiser's vanity license plate read IHTFP, a popular MIT acronym with many meanings. The most noteworthy version, itself dating back to the pressure-filled world of MIT student life in the 1950s, is ``I hate this fucking place.'' In 1990, however, the Museum used the acronym as a basis for a journal on the history of hacks. Titled \textit{The Journal of the Institute for Hacks, Tomfoolery, and Pranks}, it offers an adept summary of the hacking.

``In the culture of hacking, an elegant, simple creation is as highly valued as it is in pure science,'' writes \textit{Boston Globe} reporter Randolph Ryan in a 1993 article attached to the police car exhibit. ``A Hack differs from the ordinary college prank in that the event usually requires careful planning, engineering and finesse, and has an underlying wit and inventiveness,'' Ryan writes. ``The unwritten rule holds that a hack should be good-natured, non-destructive and safe. In fact, hackers sometimes assist in dismantling their own handiwork.''

The urge to confine the culture of computer hacking within the same ethical boundaries is well-meaning but impossible. Although most software hacks aspire to the same spirit of elegance and simplicity, the software medium offers less chance for reversibility. Dismantling a police cruiser is easy compared with dismantling an idea, especially an idea whose time has come.

Once a vague item of obscure student jargon, the word ``hacker'' has become a linguistic billiard ball, subject to political spin and ethical nuances. Perhaps this is why so many hackers and journalists enjoy using it.  We cannot predict how people will use the word in the future.  We can, however, decide how we will use it ourselves.  Using the term ``cracking'' rather than ``hacking,'' when you mean ``security breaking,'' shows respect for Stallman and all the hackers mentioned in this book, and helps preserve something which all computer users have benefited from: the hacker spirit.
