%% Copyright (c) 2002, 2010 Sam Williams
%% Copyright (c) 2010 Richard M. Stallman
%% Permission is granted to copy, distribute and/or modify this
%% document under the terms of the GNU Free Documentation License,
%% Version 1.3 or any later version published by the Free Software
%% Foundation; with no Invariant Sections, no Front-Cover Texts, and
%% no Back-Cover Texts. A copy of the license is included in the
%% file called ``gfdl.tex''.

\chapter{A Portrait of the Hacker as a Young Man}
\chaptermark{A Portrait of the Hacker}

Richard Stallman's mother, Alice Lippman, still remembers the moment she realized her son had a special gift.

``I think it was when he was eight,'' Lippman recalls.

The year was 1961, and Lippman, a recently divorced single mother, was whiling away a weekend afternoon within the family's tiny one-bedroom apartment on Manhattan's Upper West Side. Leafing through a copy of Scientific American, Lippman came upon her favorite section, the Martin Gardner-authored column titled ``Mathematical Games.'' A substitute art teacher at the time, Lippman enjoyed Gardner's column for the brain-teasers it provided. With her son already ensconced in a book on the nearby sofa, Lippman decided to take a crack at solving the week's feature puzzle.

``I wasn't the best person when it came to solving the puzzles,'' she admits. ``But as an artist, I found they really helped me work through conceptual barriers.''

Lippman says her attempt to solve the puzzle met an immediate brick wall. About to throw the magazine down in disgust, Lippman was surprised by a gentle tug on her shirt sleeve.

``It was Richard,'' she recalls, ``He wanted to know if I needed any help.''

Looking back and forth, between the puzzle and her son, Lippman says she initially regarded the offer with skepticism. ``I asked Richard if he'd read the magazine,'' she says. ``He told me that, yes, he had and what's more he'd already solved the puzzle. The next thing I know, he starts explaining to me how to solve it.''

Hearing the logic of her son's approach, Lippman's skepticism quickly gave way to incredulity. ``I mean, I always knew he was a bright boy,'' she says, ``but this was the first time I'd seen anything that suggested how advanced he really was.''

Thirty years after the fact, Lippman punctuates the memory with a laugh. ``To tell you the truth, I don't think I ever figured out how to solve that puzzle,'' she says. ``All I remember is being amazed he knew the answer.''

Seated at the dining-room table of her second Manhattan apartment -- the same spacious three-bedroom complex she and her son moved to following her 1967 marriage to Maurice Lippman, now deceased -- Alice Lippman exudes a Jewish mother's mixture of pride and bemusement when recalling her son's early years. The nearby dining-room credenza offers an eight-by-ten photo of Stallman glowering in full beard and doctoral robes. The image dwarfs accompanying photos of Lippman's nieces and nephews, but before a visitor can make too much of it, Lippman makes sure to balance its prominent placement with an ironic wisecrack.

``Richard insisted I have it after he received his honorary doctorate at the University of Glasgow,'' says Lippman. ``He said to me, `Guess what, mom? It's the first graduation I ever attended.'\hspace{0.01in}''\endnote{One of the major background sources for this chapter was the interview ``Richard Stallman: High School Misfit, Symbol of Free Software, MacArthur-Certified Genius'' by Michael Gross, author of the 1999 book \textit{Talking About My Generation}, a collection of interviews with notable personalities from the so-called ``Baby Boom'' generation. Although Stallman did not make it into the book, Gross published the interview as an online supplement to the book's web site. The URL for the interview has changed several times since I first came across it. According to various readers who have gone searching for it, you can now find the interview at \url{http://www.mgross.com/MoreThgsChng/interviews/stallman1.html}.}

Such comments reflect the sense of humor that comes with raising a child prodigy. Make no mistake, for every story Lippman hears and reads about her son's stubbornness and unusual behavior, she can deliver at least a dozen in return.

``He used to be so conservative,'' she says, throwing up her hands in mock exasperation. ``We used to have the worst arguments right here at this table. I was part of the first group of public city school teachers that struck to form a union, and Richard was very angry with me. He saw unions as corrupt. He was also very opposed to social security. He thought people could make much more money investing it on their own. Who knew that within 10 years he would become so idealistic? All I remember is his stepsister coming to me and saying, `What is he going to be when he grows up? A fascist?'\hspace{0.01in}''\endnote{RMS: I don't remember telling her this.  All I can say is I strongly disagree with those views now.  When I was in my teens, I lacked compassion for the difficulties most people encounter in life; my problems were different.  I did not appreciate how the wealthy will reduce most people to poverty unless we organize at all levels to stop them.  I did not understand how hard it is for most people to resist social pressure to do foolish things, such as spend all their money instead of saving, since I hardly even noticed the pressure myself. In addition, unions in the 60s, when they were very powerful, were sometimes arrogant or corrupt.  But they are much weaker today, and the result is that economic growth, when it occurs, benefits mainly the rich.}

As a single parent for nearly a decade -- she and Richard's father, Daniel Stallman, were married in 1948, divorced in 1958, and split custody of their son afterwards -- Lippman can attest to her son's aversion to authority. She can also attest to her son's lust for knowledge. It was during the times when the two forces intertwined, Lippman says, that she and her son experienced their biggest battles.

``It was like he never wanted to eat,'' says Lippman, recalling the behavior pattern that set in around age eight and didn't let up until her son's high-school graduation in 1970. ``I'd call him for dinner, and he'd never hear me. I'd have to call him 9 or 10 times just to get his attention. He was totally immersed.''

Stallman, for his part, remembers things in a similar fashion, albeit with a political twist.

``I enjoyed reading,'' he says. ``If I wanted to read, and my mother told me to go to the kitchen and eat or go to sleep, I wasn't going to listen. I saw no reason why I couldn't read. No reason why she should be able to tell me what to do, period. Essentially, what I had read about, ideas such as democracy and individual freedom, I applied to myself. I didn't see any reason to exclude children from these principles.''

The belief in individual freedom over arbitrary authority extended to school as well. Two years ahead of his classmates by age 11, Stallman endured all the usual frustrations of a gifted public-school student. It wasn't long after the puzzle incident that his mother attended the first in what would become a long string of parent-teacher conferences.

``He absolutely refused to write papers,'' says Lippman, recalling an early controversy. ``I think the last paper he wrote before his senior year in high school was an essay on the history of the number system in the west for a fourth-grade teacher.''  To be required to choose a specific topic when there was nothing he actually wanted to write about was almost impossible for Stallman, and painful enough to make him go to great lengths to avoid such situations.

Gifted in anything that required analytical thinking, Stallman gravitated toward math and science at the expense of his other studies. What some teachers saw as single-mindedness, however, Lippman saw as impatience. Math and science offered simply too much opportunity to learn, especially in comparison to subjects and pursuits for which her son seemed less naturally inclined. Around age 10 or 11, when the boys in Stallman's class began playing a regular game of touch football, she remembers her son coming home in a rage. ``He wanted to play so badly, but he just didn't have the coordination skills,'' Lippman recalls. ``It made him so angry.''

The anger eventually drove her son to focus on math and science all the more. Even in the realm of science, however, her son's impatience could be problematic. Poring through calculus textbooks by age seven, Stallman saw little need to dumb down his discourse for adults. Sometime, during his middle-school years, Lippman hired a student from nearby Columbia University to play big brother to her son. The student left the family's apartment after the first session and never came back. ``I think what Richard was talking about went over his head,'' Lippman speculates.

Another favorite maternal memory dates back to the early 1960s, shortly after the puzzle incident. Around age seven, two years after the divorce and relocation from Queens, Richard took up the hobby of launching model rockets in nearby Riverside Drive Park. What started as aimless fun soon took on an earnest edge as her son began recording the data from each launch. Like the interest in mathematical games, the pursuit drew little attention until one day, just before a major NASA launch, Lippman checked in on her son to see if he wanted to watch.

``He was fuming,'' Lippman says. ``All he could say to me was, `But I'm not published yet.' Apparently he had something that he really wanted to show NASA.''  Stallman doesn't remember the incident, but thinks it more likely that he was anguished because he didn't have anything to show.

Such anecdotes offer early evidence of the intensity that would become Stallman's chief trademark throughout life. When other kids came to the table, Stallman stayed in his room and read. When other kids played Johnny Unitas, Stallman played spaceman. ``I was weird,'' Stallman says, summing up his early years succinctly in a 1999 interview. ``After a certain age, the only friends I had were teachers.''\endnote{\textit{Ibid.}}  Stallman was not ashamed of his weird characteristics, distinguishing them from the social ineptness that he did regard as a failing.  However, both contributed together to his social exclusion.

Although it meant courting more run-ins at school, Lippman decided to indulge her son's passion. By age 12, Richard was attending science camps during the summer and private school during the school year. When a teacher recommended her son enroll in the Columbia Science Honors Program, a post-Sputnik program designed for gifted middle- and high-school students in New York City, Stallman added to his extracurriculars and was soon commuting uptown to the Columbia University campus on Saturdays.

Dan Chess, a fellow classmate in the Columbia Science Honors Program, recalls Richard Stallman seeming a bit weird even among the students who shared a similar lust for math and science. ``We were all geeks and nerds, but he was unusually poorly adjusted,'' recalls Chess, now a mathematics professor at Hunter College. ``He was also smart as shit. I've known a lot of smart people, but I think he was the smartest person I've ever known.''

Seth Breidbart, a fellow Columbia Science Honors Program alumnus, offers bolstering testimony. A computer programmer who has kept in touch with Stallman thanks to a shared passion for science fiction and science-fiction conventions, he recalls the 15-year-old, buzz-cut-wearing Stallman as ``scary,'' especially to a fellow 15-year-old.

``It's hard to describe,'' Breidbart says. ``It wasn't like he was unapproachable. He was just very intense. [He was] very knowledgeable but also very hardheaded in some ways.''

Such descriptions give rise to speculation: are judgment-laden adjectives like ``intense'' and ``hardheaded'' simply a way to describe traits that today might be categorized under juvenile behavioral disorder? A December, 2001, \textit{Wired} magazine article titled ``The Geek Syndrome'' paints the portrait of several scientifically gifted children diagnosed with high-functioning autism or Asperger Syndrome. In many ways, the parental recollections recorded in the \textit{Wired} article are eerily similar to the ones offered by Lippman. Stallman also speculates about this.  In the interview for a 2000 profile for the \textit{Toronto Star}, Stallman said he wondered if he were ``borderline autistic.''  The article inaccurately cited the speculation as a certainty.\endnote{See Judy Steed, \textit{Toronto Star}, \textit{BUSINESS}, (October 9, 2000): C03.

\begin{quote}
His vision of free software and social cooperation stands in stark contrast to the isolated nature of his private life. A Glenn Gould-like eccentric, the Canadian pianist was similarly brilliant, articulate, and lonely. Stallman considers himself afflicted, to some degree, by autism: a condition that, he says, makes it difficult for him to interact with people.
\end{quote}}

Such speculation benefits from the fast and loose nature of most so-called ``behavioral disorders'' nowadays, of course. As Steve Silberman, author of ``The Geek Syndrome,'' notes, American psychiatrists have only recently come to accept Asperger Syndrome as a valid umbrella term covering a wide set of behavioral traits. The traits range from poor motor skills and poor socialization to high intelligence and an almost obsessive affinity for numbers, computers, and ordered systems.\endnote{See Steve Silberman, ``The Geek Syndrome,'' \textit{Wired} (December, 2001), \url{http://www.wired.com/wired/archive/9.12/aspergers_pr.html}.}

``It's possible I could have had something like that,'' Stallman says. ``On the other hand, one of the aspects of that syndrome is difficulty following rhythms. I can dance. In fact, I love following the most complicated rhythms. It's not clear cut enough to know.''  Another possibility is that Stallman had a ``shadow syndrome'' which goes some way in the direction of Asperger's syndrome but without going beyond the limits of normality.\endnote{See John Ratey and Catherine Johnson, ``Shadow Syndromes.''}

Chess, for one, rejects such attempts at back-diagnosis. ``I never thought of him [as] having that sort of thing,'' he says. ``He was just very unsocialized, but then, we all were.''

Lippman, on the other hand, entertains the possibility. She recalls a few stories from her son's infancy, however, that provide fodder for speculation. A prominent symptom of autism is an oversensitivity to noises and colors, and Lippman recalls two anecdotes that stand out in this regard. ``When Richard was an infant, we'd take him to the beach,'' she says. ``He would start screaming two or three blocks before we reached the surf. It wasn't until the third time that we figured out what was going on: the sound of the surf was hurting his ears.'' She also recalls a similar screaming reaction in relation to color: ``My mother had bright red hair, and every time she'd stoop down to pick him up, he'd let out a wail.''

In recent years, Lippman says she has taken to reading books about autism and believes that such episodes were more than coincidental. ``I do feel that Richard had some of the qualities of an autistic child,'' she says. ``I regret that so little was known about autism back then.''

Over time, however, Lippman says her son learned to adjust. By age seven, she says, her son had become fond of standing at the front window of subway trains, mapping out and memorizing the labyrinthian system of railroad tracks underneath the city. It was a hobby that relied on an ability to accommodate the loud noises that accompanied each train ride. ``Only the initial noise seemed to bother him,'' says Lippman. ``It was as if he got shocked by the sound but his nerves learned how to make the adjustment.''

For the most part, Lippman recalls her son exhibiting the excitement, energy, and social skills of any normal boy. It wasn't until after a series of traumatic events battered the Stallman household, she says, that her son became introverted and emotionally distant.

The first traumatic event was the divorce of Alice and Daniel Stallman, Richard's father. Although Lippman says both she and her ex-husband tried to prepare their son for the blow, she says the blow was devastating nonetheless. ``He sort of didn't pay attention when we first told him what was happening,'' Lippman recalls. ``But the reality smacked him in the face when he and I moved into a new apartment. The first thing he said was, `Where's Dad's furniture?'\hspace{0.01in}''

For the next decade, Stallman would spend his weekdays at his mother's apartment in Manhattan and his weekends at his father's home in Queens. The shuttling back and forth gave him a chance to study a pair of contrasting parenting styles that, to this day, leaves Stallman firmly opposed to the idea of raising children himself. Speaking about his father, a World War II vet who died in early 2001, Stallman balances respect with anger. On one hand, there is the man whose moral commitment led him to learn French just so he could be more helpful to Allies when they'd finally fight the Nazis in France. On the other hand, there was the parent who always knew how to craft a put-down for cruel effect.\endnote{Regrettably, I did not get a chance to interview Daniel Stallman for this book. During the early research for this book, Stallman informed me that his father suffered from Alzheimer's. When I resumed research in late 2001, I learned, sadly, that Daniel Stallman had died earlier in the year.}

``My father had a horrible temper,'' Stallman says. ``He never screamed, but he always found a way to criticize you in a cold, designed-to-crush way.''

As for life in his mother's apartment, Stallman is less equivocal. ``That was war,'' he says. ``I used to say in my misery, `I want to go home,' meaning to the nonexistent place that I'll never have.''

For the first few years after the divorce, Stallman found the tranquility that eluded him in the home of his paternal grandparents. One died when he was 8, and the other when he was 10. For Stallman, the loss was devastating. ``I used to go and visit and feel I was in a loving, gentle environment,'' Stallman recalls. ``It was the only place I ever found one, until I went away to college.''

Lippman lists the death of Richard's paternal grandparents as the second traumatic event. ``It really upset him,'' she says. He was very close to both his grandparents. Before they died, he was very outgoing, almost a leader-of-the-pack type with the other kids. After they died, he became much more emotionally withdrawn.

From Stallman's perspective, the emotional withdrawal was merely an attempt to deal with the agony of adolescence. Labeling his teenage years a ``pure horror,'' Stallman says he often felt like a deaf person amid a crowd of chattering music listeners.

``I often had the feeling that I couldn't understand what other people were saying,'' says Stallman, recalling his sense of exclusion. ``I could understand the words, but something was going on underneath the conversations that I didn't understand. I couldn't understand why people were interested in the things other people said.''

For all the agony it produced, adolescence would have an encouraging effect on Stallman's sense of individuality. At a time when most of his classmates were growing their hair out, Stallman preferred to keep his short. At a time when the whole teenage world was listening to rock and roll, Stallman preferred classical music. A devoted fan of science fiction, \textit{Mad} magazine, and late-night TV, Stallman came to have a distinctly off-the-wall personality that met with the incomprehension of parents and peers alike.

``Oh, the puns,'' says Lippman, still exasperated by the memory of her son's teenage personality. ``There wasn't a thing you could say at the dinner table that he couldn't throw back at you as a pun.''

Outside the home, Stallman saved the jokes for the adults who tended to indulge his gifted nature. One of the first was a summer-camp counselor who lent Stallman a manual for the IBM 7094 computer during his 8th or 9th year. To a preteenager fascinated with numbers and science, the gift was a godsend.\endnote{Stallman, an atheist, would probably quibble with this description. Suffice it to say, it was something Stallman welcomed. See Gross (1999): ``As soon as I heard about computers, I wanted to see one and play with one.''} Soon, Stallman was writing out programs on paper in the instructions of the 7094.  There was no computer around to run them on, and he had no real applications to use one for, but he yearned to write a program -- any program whatsoever.  He asked the counselor for arbitrary suggestions of something to code.

With the first personal computer still a decade away, Stallman would be forced to wait a few years before getting access to his first computer. His chance finally came during his senior year of high school.  The IBM New York Scientific Center, a now-defunct research facility in downtown Manhattan, offered Stallman the chance to try to write his first real program.  His fancy was to write a pre-processor for the programming language PL/I, designed to add the tensor algebra summation convention as a feature to the language. ``I first wrote it in PL/I, then started over in assembler language when the compiled PL/I program was too big to fit in the computer,'' he recalls.

For the summer after high-school graduation, the New York Scientific Center hired him.  Tasked with writing a numerical analysis program in Fortran, he finished that in a few weeks, acquiring such a distaste for the Fortran language that he vowed never to write anything in it again.  Then he spent the rest of the summer writing a text-editor in APL.

Simultaneously, Stallman had held a laboratory-assistant position in the biology department at Rockefeller University. Although he was already moving toward a career in math or physics, Stallman's analytical mind impressed the lab director enough that a few years after Stallman departed for college, Lippman received an unexpected phone call. ``It was the professor at Rockefeller,'' Lippman says. ``He wanted to know how Richard was doing. He was surprised to learn that he was working in computers. He'd always thought Richard had a great future ahead of him as a biologist.''

Stallman's analytical skills impressed faculty members at Columbia as well, even when Stallman himself became a target of their ire. ``Typically once or twice an hour [Stallman] would catch some mistake in the lecture,'' says Breidbart. ``And he was not shy about letting the professors know it immediately. It got him a lot of respect but not much popularity.''

Hearing Breidbart's anecdote retold elicits a wry smile from Stallman. ``I may have been a bit of a jerk sometimes,'' he admits. ``But I found kindred spirits among teachers, because they, too, liked to learn. Kids, for the most part, didn't. At least not in the same way.''

Hanging out with the advanced kids on Saturday nevertheless encouraged Stallman to think more about the merits of increased socialization. With college fast approaching, Stallman, like many in his Columbia Science Honors Program, had narrowed his list of desired schools down to two choices: Harvard and MIT. Hearing of her son's desire to move on to the Ivy League, Lippman became concerned. As a 15-year-old high-school junior, Stallman was still having run-ins with teachers and administrators. Only the year before, he had pulled straight A's in American History, Chemistry, French, and Algebra, but a glaring F in English reflected the ongoing boycott of writing assignments. Such miscues might draw a knowing chuckle at MIT, but at Harvard, they were a red flag.

During her son's junior year, Lippman says she scheduled an appointment with a therapist. The therapist expressed instant concern over Stallman's unwillingness to write papers and his run-ins with teachers. Her son certainly had the intellectual wherewithal to succeed at Harvard, but did he have the patience to sit through college classes that required a term paper? The therapist suggested a trial run. If Stallman could make it through a full year in New York City public schools, including an English class that required term papers, he could probably make it at Harvard. Following the completion of his junior year, Stallman promptly enrolled in public summer school downtown and began making up the mandatory humanities classes he had shunned earlier in his high-school career.

By fall, Stallman was back within the mainstream population of New York City high-school students, at Louis D. Brandeis High School on on West 84th Street. It wasn't easy sitting through classes that seemed remedial in comparison with his Saturday studies at Columbia, but Lippman recalls proudly her son's ability to toe the line.

``He was forced to kowtow to a certain degree, but he did it,'' Lippman says. ``I only got called in once, which was a bit of a miracle. It was the calculus teacher complaining that Richard was interrupting his lesson. I asked how he was interrupting. He said Richard was always accusing the teacher of using a false proof. I said, `Well, is he right?' The teacher said, `Yeah, but I can't tell that to the class. They wouldn't understand.'\hspace{0.01in}''

By the end of his first semester at Brandeis High, things were falling into place. A 96 in English wiped away much of the stigma of the 60 earned 2 years before. For good measure, Stallman backed it up with top marks in American History, Advanced Placement Calculus, and Microbiology. The crowning touch was a perfect 100 in Physics. Though still a social outcast, Stallman finished his 10 months at Brandeis as the fourth-ranked student in a class of 789.

Outside the classroom, Stallman pursued his studies with even more diligence, rushing off to fulfill his laboratory-assistant duties at Rockefeller University during the week and dodging the Vietnam protesters on his way to Saturday school at Columbia. It was there, while the rest of the Science Honors Program students sat around discussing their college choices, that Stallman finally took a moment to participate in the preclass bull session.

Recalls Breidbart, ``Most of the students were going to Harvard and MIT, of course, but you had a few going to other Ivy League schools. As the conversation circled the room, it became apparent that Richard hadn't said anything yet. I don't know who it was, but somebody got up the courage to ask him what he planned to do.''

Thirty years later, Breidbart remembers the moment clearly. As soon as Stallman broke the news that he, too, would be attending Harvard University in the fall, an awkward silence filled the room. Almost as if on cue, the corners of Stallman's mouth slowly turned upward into a self-satisfied smile.

Says Breidbart, ``It was his silent way of saying, `That's right. You haven't got rid of me yet.'\hspace{0.01in}''

\theendnotes
\setcounter{endnote}{0}
