%% Copyright (c) 2002, 2010 Sam Williams
%% Copyright (c) 2010 Richard M. Stallman
%% Permission is granted to copy, distribute and/or modify this
%% document under the terms of the GNU Free Documentation License,
%% Version 1.3 or any later version published by the Free Software
%% Foundation; with no Invariant Sections, no Front-Cover Texts, and
%% no Back-Cover Texts. A copy of the license is included in the
%% file called ``gfdl.tex''.

\chapter{Портрет хакера в юности}
\chaptermark{Портрет хакера}

Элис Липпман, мать Ричарда Столлмана, до сих пор помнит момент, когда сын проявил свою одарённость.

\enquote{По-моему, это случилось, когда ему было 8 лет}, -- говорит она.

На дворе стоял 1961 год. Липпман недавно развелась и стала матерью-одиночкой. С сыном она перебралась в крошечную квартиру на одну спальню, расположенную в Верхнем Вест-Сайде Манхэттена. Здесь она и проводила тот выходной день. Листая номер Scientific American, Элис наткнулась на любимую колонку -- \enquote{Математические игры} Мартина Гарднера. В то время она работала учителем рисования на замене, и задачки Гарднера отлично годились на то, чтобы размять мозги. Расположившись на диване рядом с сыном, который увлечённо читал книгу, Элис взялась за головоломку недели.

\enquote{Меня нельзя было назвать специалистом по решению головоломок, -- признаётся Липпман, -- но для меня, художника, они были полезны тем, что тренировали интеллект и делали его гибче}.

Вот только сегодня все её попытки решить задачу разбивались вдребезги, как об стену. Элис уже готова была в сердцах зашвырнуть журнал куда подальше, как вдруг почувствовала, что её легонько дёргают за рукав. Это был Ричард. Он спрашивал, нужна ли помощь.

Элис посмотрела на сына, потом на головоломку, потом снова на сына, и выразила сомнение, что он сможет чем-то помочь. \enquote{Я спросила, читал ли он журнал. Он ответил: да, читал, и даже решил головоломку. И начинает объяснять мне, как она решается. Этот момент врезался мне в память на всю жизнь}.

Выслушав решение сына, Элис покачала головой -- её сомнение переросло в откровенное недоверие. \enquote{Ну то есть, он всегда был умным и способным мальчиком, -- говорит она, -- но тогда я впервые столкнулась с проявлением такого неожиданно развитого мышления}.

Сейчас, 30 лет спустя, Липпман вспоминает об этом со смехом. \enquote{Честно признаться, я даже толком не поняла его решения, ни тогда, ни позже, -- рассказывает Элис, -- я просто впечатлилась тем, что он знает ответ}.

Мы сидим за обеденным столом в просторной манхэттенской квартире с тремя спальнями -- сюда вместе с Ричардом Элис переехала в 1967 году, выйдя замуж за Мориса Липпмана. Вспоминая о ранних годах сына, Элис источает типичную для еврейской матери гордость вперемешку со смущением. Отсюда виден сервант, на котором стоит большая фотография Ричарда с окладистой бородой и в академических одеждах. Она величественно возвышается над маленькими фотокарточками племянниц и племянников Элис. Из этого можно было бы сделать далеко идущие выводы, но ироническое объяснение Липпман сглаживает впечатление: \enquote{Ричард настоял, чтобы я купила их после того, как он получил почётную докторскую степень от Университета Глазго. Он тогда сказал мне: \enquote{Знаешь что, мам? Это первый выпускной вечер, на котором я побывал}\hspace{0.01in}}\endnote{Одним из главных источников для этой главы послужило интервью \enquote{Richard Stallman: High School Misfit, Symbol of Free Software, MacArthur-Certified Genius}, Michael Gross}.

В подобных репликах отражается заряд юмора, который жизненно необходим для воспитания вундеркинда. Можете быть уверены: на каждую известную историю об упрямстве и эксцентричности Столлмана, его мать может рассказать ещё дюжину.

\enquote{Он был ярым консерватором, -- говорит она, всплескивая руками в картинном раздражении, -- мы уже даже привыкли выслушивать яростную реакционную риторику за обедом. Мы с другими учителями пыталась основать свой профсоюз, и Ричард очень сердился на меня. Он воспринимал профсоюзы как рассадники коррупции. Воевал он и против социального обеспечения. Он считал, что гораздо лучше будет, если люди сами себя станут обеспечивать через инвестирование. Кто знал, что через каких-то 10 лет он станет таким идеалистом? Я помню, как его сводная сестра однажды подошла ко мне и спросила: \enquote{Боже, кто из него вырастет? Фашист?}\hspace{0.01in}}\endnote{РМС: Я не помню, чтобы вёл такие речи. Я могу сказать на этот счёт только то, что сейчас категорически не согласен с такими взглядами. Будучи подростком, я не испытывал сострадания к жизненным трудностям людей, потому что у меня были совсем другие проблемы. Я недооценивал то, как быстро богатства уходят от основной массы населения к узкой прослойке людей, если не препятствовать этому процессу на всех уровнях. Я не понимал, как трудно большинству людей сопротивляться давлению социума, из-за которого они тратят деньги на всякие глупости, вместо того, чтобы разумно ими распоряжаться. Сам-то я почти не ощущал этого давления. Да, профсоюзы 60-х годов, находясь на пике своего могущества, были зачастую надменными и коррумпированными. Но сейчас они намного слабее, и в результате львиная доля благ экономического роста достаётся богатым}.

Элис вышла замуж за отца Ричарда, Даниэля Столлмана, в 1948 году, развелась с ним через 10 лет, и с тех пор растила сына почти в одиночку, хотя отец оставался его опекуном. Поэтому Элис может с полным правом заявить, что хорошо знает характер сына, в частности -- его явное отвращение к власти. Также она подтверждает его фанатичную тягу к знаниям. От этих качеств ей пришлось несладко. Дом превратился в поле битвы.

\enquote{Проблемы были даже с питанием, ему как будто вообще никогда не хотелось есть, -- вспоминает Липпман о том, что происходило с Ричардом примерно с 8 лет и до окончания школы, -- я зову его ужинать, а он игнорирует меня, как будто не слышит. Только после девятого-десятого раза он, наконец, отвлекался и обращал на меня внимание. Он с головой погружался в свои занятия, и вытащить его оттуда было трудно}.

В свою очередь, Ричард описывает те события похожим образом, но придаёт им политический оттенок.

\enquote{Я обожал читать, -- говорит он, -- если я погружался в чтение, а мама говорила мне идти есть или спать, я просто не слушал её. Я просто не понимал, почему мне не дают читать. Не видел ни малейшей причины, почему я должен делать то, что мне велят. По сути, я примерял на себя и отношения в семье всё то, что я читал о демократии и личной свободе. Я отказывался понимать, почему эти принципы не распространяют на детей}.

Ричард и в школе предпочитал следовать соображениям личной свободы вместо требований откуда-то свыше. К 11 годам он на две ступени опередил своих сверстников, и получил массу разочарований, типичных для одарённого ребёнка в условиях средней школы. Вскоре после памятного эпизода с решением головоломки, для матери Ричарда началась эпоха регулярных споров и объяснений с учителями.

\enquote{Он совершенно игнорировал письменные работы, -- вспоминает Элис первые конфликты, -- по-моему, последней его работой в младшей школе было эссе по истории использования систем счисления на Западе в 4 классе}. Он отказывался писать на темы, которые его не интересовали. Столлман, обладая феноменальным аналитическим мышлением, углубился в математику и точные науки в ущерб остальным дисциплинам. Некоторые учителя считали это целеустремлённостью, но Липпман видела в этом нетерпение и несдержанность. Точные науки и без того были представлены в программе намного шире, чем те, которые Ричард не любил. Когда Столлману было 10 или 11 лет, его одноклассники затеяли игру в одну из разновидностей американского футбола, после которой Ричард пришёл домой в ярости. \enquote{Он очень хотел поиграть, но оказалось, что его координация и прочие физические навыки оставляют желать лучшего, -- рассказывает Липпман, -- это его сильно разозлило}.

Разозлившись, Столлман ещё сильнее сконцентрировался на математике и точных науках. Однако даже в этих родных для Ричарда областях его нетерпение иногда создавало проблемы. Уже к семи годам погружаясь в учебники алгебры, он не считал нужным быть проще в общении со взрослыми. Однажды, когда Столлман учился в средней ступени, Элис наняла для него репетитора в лице студента Колумбийского Университета. Первого же занятия хватило, чтобы студент больше не появлялся на пороге их квартиры. \enquote{Видимо, то, что говорил ему Ричард, просто не укладывалось в его бедной голове}, -- предполагает Липпман.

Другое любимое воспоминание матери относится к началу 60-х годов, когда Столлману было около семи лет. С момента развода родителей прошло 2 года, Элис с сыном переехали из Квинса в Верхний Вест-Сайд, где Ричард полюбил ходить в парк на Риверсайд-Драйв, чтобы запускать там игрушечные модели ракет. Скоро развлечение переросло в серьёзное, основательное занятие -- он даже стал вести подробные записи о каждом запуске. Как и на его интерес к математическим задачам, на это увлечение не обращали особого внимания, пока однажды перед масштабным запуском НАСА мать в шутку не поинтересовалась у сына, не хочет ли он посмотреть, правильно ли космическое агентство следует его записям.

\enquote{Он вскипел, -- рассказывает Липпман, -- и смог ответить только: \enquote{Я ещё не показывал им свои записи!}. Наверное, он действительно собирался что-то показать НАСА}. Сам Столлман не помнит этого случая, но говорит, что в такой ситуации ему было бы стыдно из-за того, что показывать НАСА на самом деле нечего.

Эти семейные анекдоты были первыми проявлениями характерной одержимости Столлмана, которая не покидает его до сих пор. Когда дети бежали к столу, Ричард продолжал читать в своей комнате. Когда дети играли в футбол, подражая легендарному Джонни Юнайтасу, Ричард изображал космонавта. \enquote{Я был странным, -- подытоживает Столлман свои детские годы в интервью 1999 года, -- к определённому возрасту у меня в друзьях остались только учителя}.\endnote{\textit{Ibid.}} Ричард не стыдился своих странных черт и наклонностей, в отличие от своего неумения ладить с людьми, которое он считал настоящей бедой. Тем не менее, и то, и другое в равной степени привело его к отчуждению от всех.

Элис решила дать полный зелёный свет увлечениям сына, хоть это и грозило новыми сложностями в школе. В 12 лет Ричард всё лето посещал научные лагеря, а с началом учебного года стал дополнительно ходить в частную школу. Один из учителей посоветовал Липпман записать сына в Колумбийскую программу естественнонаучных достижений, которую разработали в Нью-Йорке для одарённых школьников средних и старших классов. Столлман без возражений добавил занятия по этой программе к своим внеклассным урокам, и вскоре каждую субботу стал наведываться в кампус Колумбийского Университета, расположенный среди жилых массивов.

По воспоминаниям Дэна Чесса, одного из сокурсников Столлмана по Колумбийской программе, Ричард выделялся даже на фоне этого сборища таких же одержимых математикой и точными науками. \enquote{Конечно, мы там все были ботанами и гиками, -- рассказывает Чесс, теперь уже профессор математики в Хантерском колледже, -- но Столлман был совсем уж явно не от мира сего. Он был просто охренеть каким умником. Я знаю немало умных людей, но, думаю, Столлман умнейший человек из всех, кого я вообще встречал когда-либо}.

Программист Сет Брайдбарт, также выпускник этой программы, полностью согласен с этим. Он хорошо общался с Ричардом, потому что так же как и он увлекался научной фантастикой, и посещал конвенты. Сет помнит Столлмана как 15-летнего парня в удручающих шмотках, производящего на людей \enquote{жуткое впечатление}, особенно на таких же пятнадцатилеток.

\enquote{Это трудно объяснить, -- говорит Брайдбарт, -- он не то чтобы полностью замыкался в себе, он был просто чересчур одержимым. Ричард впечатлял своими глубокими познаниями, но явная отрешённость отнюдь не добавляла ему привлекательности}.

Такие описания наталкивают на размышления: есть ли основания предполагать, что под эпитетами вроде \enquote{одержимости} и \enquote{отрешённости} скрывалось то, что сегодня считается подростковыми расстройствами поведения? В декабре 2001 года в журнале \textit{Wired} вышла статья под заголовком \enquote{Синдром гика}, в ней описаны научно одарённые дети с высокофункциональным аутизмом и синдромом Аспергера. Воспоминания их родителей, изложенные в статье, во многом похожи на рассказы Элис Липпман. Столлман и сам задумывается над этим. В интервью 2000 года для \textit{Toronto Star} он высказал предположение, что может иметь \enquote{пограничное аутистическое расстройство}. Правда, в статье его предположение неосторожно выставили как уверенность\endnote{Источник: Judy Steed, \textit{Toronto Star}, \textit{BUSINESS}, (October 9, 2000): C03.

\begin{quote}
Его взгляды касательно свободного ПО и сотрудничества между людьми сильно разнятся с его личной социальной изоляцией. Подобно эксцентричному канадскому пианисту Гленну Гульду, Столлман блестящ, красноречив и одинок. Он считает, что в некоторой степени подвержен аутизму: это состояние затрудняет взаимодействия с людьми.
\end{quote}}

В свете того, что определения многих так называемых \enquote{расстройств поведения} до сих пор весьма расплывчаты, это предположение выглядит особенно реалистичным. Как заметил Стив Сильберман, автор статьи \enquote{Синдром гика}, американские психиатры не так давно признали, что под синдромом Аспергера скрывается очень широкий набор поведенческих черт, начиная плохими моторными и социальными навыками, и заканчивая одержимостью числами, компьютерами и упорядоченными структурами. \endnote{Источник: Steve Silberman, \enquote{The Geek Syndrome,} \textit{Wired} (December, 2001)}.

\enquote{Может быть, у меня в самом деле что-то подобное, -- говорит Столлман, -- с другой стороны, один из симптомов синдрома Аспергера это трудности с чувством ритма. А я могу танцевать. Больше того, мне нравится следовать самым сложным ритмам. В общем, нельзя сказать наверняка}. Речь может идти о некой градации синдрома Аспергера, которая большей частью вписывается в рамки нормальности. \endnote{На эту тему можно почитать: John Ratey, Catherine Johnson, \enquote{Shadow Syndromes.}}

Дэн Чесс, впрочем, не разделяет этого стремления поставить сейчас диагноз Ричарду. \enquote{У меня ни разу не возникало мысли, что он реально какой-то ненормальный, в медицинском смысле, -- говорит он, -- он просто был очень отрешённым от окружающих людей и их проблем, был довольно необщительным, но если уж на то пошло -- тогда мы все были такими, в той или иной мере}.

Элис Липпман вообще веселят все эти споры вокруг психических расстройств Ричарда, хотя она помнит парочку историй, которые можно добавить к аргументам \enquote{за}. Характерным симптомом аутистических расстройств считается нетерпимость к шуму и ярким цветам, и когда Ричарда младенцем брали с собой на пляж, он за два-три квартала до океана начинал плакать. Только потом догадались, что шум прибоя доводил его до боли в ушах и голове. Другой пример: у бабушки Ричарда были яркие огненно-красные волосы, и каждый раз, как она наклонялась над колыбелькой, он вопил, как будто от боли.

В последние годы Липпман стала много читать об аутизме, и всё чаще ловит себя на мысли, что особенности её сына -- не случайные причуды. \enquote{Я правда начинаю думать, что Ричард мог быть аутистичным ребёнком, -- говорит она, -- очень жаль, что в то время об этом так мало знали и говорили}.

Впрочем, по её словам, со временем Ричард стал приспосабливаться. В семилетнем возрасте ему полюбилось вставать у переднего окна в поездах метро, чтобы изучать лабиринты тоннелей под городом. Это хобби явно противоречило его нетерпимости к шуму, которого в метро было предостаточно. \enquote{Но шум шокировал его лишь поначалу, -- рассказывает Липпман, -- потом нервная система Ричарда приучилась адаптироваться под влиянием его горячего желания изучать метрополитен}.

Ранний Ричард запомнился матери вполне нормальным ребёнком -- его мысли, поступки, модели общения были как у обычного маленького мальчика. Лишь после череды драматичных событий в семье он стал замкнутым и отрешённым.

Первым таким событием стал развод родителей. Хотя Элис с мужем старались подготовить сына к этому и смягчить удар, у них ничего не вышло. \enquote{Он как будто пропустил мимо ушей все наши с ним разговоры, -- вспоминает Липпман, -- а потом реальность просто ударила его под дых при переезде на другую квартиру. Первое, что Ричард тогда спросил, было: \enquote{А где папины вещи?}\hspace{0.01in}}

С этого момента начался десятилетний период жизни на две семьи, когда Столлман на выходные перемещался от матери на Манхэттене к отцу в Квинс. Характеры родителей отличались разительно, и так же сильно различались их подходы к воспитанию, не согласуясь друг с другом. Семейная жизнь была настолько безрадостной, что Ричард до сих пор не желает и думать о том, чтобы завести собственных детей. Вспоминая отца, умершего в 2001 году, он испытывает смешанные чувства -- это был довольно крутой в обращении, суровый мужчина, ветеран Второй Мировой. Столлман уважает его за высочайшую ответственность и чувство долга -- например, отец хорошо освоил французский язык только потому, что того требовали боевые задачи против нацистов во Франции. С другой стороны, Ричарду было за что злиться на отца, ибо тот не скупился на жёсткие методы воспитания. \endnote{Очень жаль, что я не смог побеседовать с Даниэлем Столлманом. Когда я только начал собирать материал, Ричард сообщил мне, что его отец страдает от болезни Альцгеймера. В конце 2001 года я снова занялся книгой после перерыва и узнал, что Дэниель Столлман уже умер}.

\enquote{У отца был тяжёлый характер, -- рассказывает Ричард, -- он никогда не кричал, но всегда находил повод холодной и обстоятельной критикой разнести всё, что ты говоришь или делаешь}.

Взаимоотношения с матерью Столлман описывает однозначно: \enquote{Это была война. Дошло до того, что говоря себе \enquote{хочу домой}, я представлял себе какое-то нереальное место, сказочную гавань спокойствия, которую видел только в мечтах}.

Первые несколько лет после развода родителей Ричард спасался у бабушки с дедушкой по отцовской линии. \enquote{Когда я был у них, я ощущал любовь и нежность, и полностью успокаивался, -- вспоминает он, -- это было единственное моё любимое место до того, как я пошёл в колледж}. Когда ему было 8 лет, ушла из жизни бабушка, а всего через 2 года за нею последовал и дедушка, и это был второй тяжелейший удар, от которого Ричард долго не мог оправиться.

\enquote{Это по-настоящему травмировало его}, -- говорит Липпман. К бабушке с дедушкой Столлман был очень привязан. Именно после их смерти из общительного заводилы он превратился в отрешённого молчуна, всегда стоящего где-то в стороне.

Сам Ричард считает тогдашний свой уход в себя чисто возрастным явлением, когда кончается детство и многое переосмысливается и переоценивается. Он называет подростковые годы \enquote{полным кошмаром} и говорит, что ощущал себя глухонемым в толпе непрестанно болтающих любителей музыки.

\enquote{Я постоянно ловил себя на мысли, что не понимаю, о чём все вокруг толкуют, -- описывает он свою отчуждённость, -- я настолько отстал от жизни, что воспринимал лишь отдельные слова в их потоке сленга. Но вникать в их разговоры мне не хотелось, я даже не мог понять, как их могут интересовать все эти музыкальные исполнители, что были тогда на слуху}.

Но было в этой отчуждённости и кое-что полезное и даже приятное -- она воспитывала в Ричарде индивидуальность. Когда одноклассники стремились отрастить длинные лохмы на голове, он продолжал носить короткую аккуратную причёску. Когда подростки вокруг сходили с ума по рок-н-роллу, Столлман слушал классику. Преданный фанат научной фантастики, журнала \textit{Mad} и ночных телепередач, Ричард даже не и не думал идти в ногу со всеми, и это множило непонимание между ним и окружающими, не исключая и его собственных родителей.

\enquote{А ещё эти каламбуры! -- восклицает Элис, взвинченная воспоминаниями о подростковом периоде сына, -- за обедом нельзя было и фразы сказать, чтобы он не вернул тебе её, обыграв и вывернув чёрти во что}.

Вне семьи Столлман придерживал шутки для тех взрослых, что симпатизировали его одарённости. Одним из первых таких людей в его жизни стал воспитатель в летнем лагере, который дал ему почитать руководство к компьютеру IBM 7094. Ричарду было тогда 8 или 9 лет. Для ребёнка, страстно увлечённого математикой и информатикой, это было настоящим божьим даром. \endnote{Столлман, будучи атеистом, наверняка сказал это не в прямом смысле, а как обозначение подарка судьбы, о котором и помыслить было нельзя. Сам он говорил: \enquote{Узнав о компьютерах, я сгорал от желания увидеть их и поиграться с ними}}. Прошло совсем немного времени, и Ричард уже писал программы для IBM 7094, правда, только на бумаге, даже не надеясь когда-либо запустить их на реальном компьютере. Его просто увлекало составление череды инструкций для выполнения какой-нибудь задачи. Когда иссякли собственные идеи для программ, Ричард стал обращаться за ними к воспитателю.

Первые персональные ЭВМ появились только через 10 лет, так что возможности поработать на компьютере Столлману пришлось бы ждать долгие годы. Однако судьба и тут подкинула шанс: уже в последний год старшей школы Нью-Йоркский научный центр IBM предложил Ричарду составить программу -- препроцессор для PL/1, который добавлял бы в язык возможность работы с тензорной алгеброй. \enquote{Сначала я написал этот препроцессор на языке PL/1, а потом переписал его на языке ассемблера, потому что скомпилированная программа на PL/1 получилась слишком большой и не влезала в память компьютера}, -- вспоминает Столлман.

Летом, когда Ричард окончил школу, научный центр IBM пригласил его на работу. Первой задачей, которую ему поручили, стала программа численного анализа на Фортране. Столлман написал её за несколько недель, и заодно так возненавидел Фортран, что поклялся себе никогда больше не притрагиваться к этому языку. Оставшуюся часть лета он писал текстовый редактор на APL.

Одновременно Столлман работал лаборантом на биологическом факультете Университета Рокфеллера. Аналитический ум Ричарда очень впечатлил начальника лаборатории, и он ждал от Столлмана блестящей работы в биологии. Через пару лет, когда Ричард уже учился в колледже, в квартире Элис Липпман раздался звонок. \enquote{Это был тот самый профессор из Рокфеллера, начальник лаборатории, -- рассказывает Липпман, -- он хотел узнать, как поживает мой сын. Я сказала, что Ричард работает с компьютерами, и профессор страшно удивился. Он-то думал, что Ричард вовсю строит карьеру биолога}.

Мощь интеллекта Столлмана впечатляла и преподавателей Колумбийской программы, даже когда он многих стал раздражать. \enquote{Обычно раз или два за лекцию они ошибались, и Столлман всегда поправлял их, -- вспоминает Брайдбарт, -- так росли уважение к его уму и неприязнь к самому Ричарду}.

Столлман сдержанно улыбается при упоминании этих слов Брайдбарта. \enquote{Иногда я, конечно, вёл себя как придурок, -- признаётся он, -- но в конечном счёте это помогло мне найти родственные души среди преподавателей, которым тоже нравилось узнавать новое и уточнять свои знания. Ученики, как правило, не позволяли себе поправлять преподавателя. По крайней мере, настолько открыто}.

Общение с продвинутыми ребятами по субботам заставили Столлмана задуматься о плюсах социальных отношений. Стремительно приближался колледж, нужно было выбирать где учиться, и Столлман, подобно многим участникам Колумбийской программы естественнонаучных достижений, сузил набор желанных вузов до двух -- Гарварда и МТИ. Услышав, что сын всерьёз раздумывает поступить в вуз Лиги Плюща, Липпман забеспокоилась. В свои 15 лет Столлман продолжал воевать с учителями и должностными лицами. Годом раньше он получил высшие оценки по американской истории, химии, математике и французскому языку, но вот за английский красовался \enquote{неуд} -- Ричард продолжал игнорировать письменные работы. На всё это могли посмотреть сквозь пальцы в МТИ и многих других вузах, но только не в Гарварде. Столлман прекрасно подходил этому вузу по интеллекту, и совершенно не соответствовал требованиям дисциплины.

Психотерапевт, который в младшей школе обратил внимание на Ричарда из-за его выходок, предложил ему пройти пробную версию обучения в вузе, а именно -- полный год в любой школе Нью-Йорка без плохих оценок и споров с учителями. Так что до осени Столлман ходил на летние уроки по гуманитарным предметам, а потом вернулся в старший класс школы на Западной 84 улице. Ему пришлось очень нелегко, но Липпман с гордостью рассказывает, что сыну удалось справиться с собой.

\enquote{Он прогнулся в некоторой степени, -- говорит она, -- меня только раз вызывали из-за Ричарда -- он постоянно указывал учителю математики на неточности в доказательствах. Я спросила: \enquote{Ну, он хотя бы прав?}. Учитель ответил: \enquote{Да, но иначе многие не поймут доказательства}.\hspace{0.01in}}

В конце первого семестра Столлман набрал 96 баллов по английскому языку, высшие отметки по американской истории, микробиологии и углублённому курсу математики. По физике он вовсе набрал 100 баллов из ста. Он был в лидерах класса по успеваемости, и всё таким же аутсайдером в личной жизни.

На внешкольные занятия Ричард продолжал ходить с большой охотой, работа в биологической лаборатории тоже приносила ему удовольствие, и он мало обращал внимания на то, что происходило вокруг. На пути в Колумбийский Университет он одинаково быстро и невозмутимо протискивался и сквозь толпы прохожих, и через демонстрации против войны во Вьетнаме. Однажды он пошёл на неформальную тусовку сокурсников по Колумбийской программе. Все обсуждали, куда лучше поступить.

Как вспоминает Брайдбард: \enquote{Конечно, большинство учеников собирались в Гарвард и МТИ, но некоторые выбрали другие вузы Лиги Плюща. И тут кто-то спросил у Столлмана, куда он будет поступать. Когда Ричард ответил, что в Гарвард -- все как-то поутихли и стали переглядываться. Ричард же еле заметно улыбнулся, как бы говоря: \enquote{Да-да, мы с вами ещё не расстаёмся!}\hspace{0.01in}}.

\theendnotes
\setcounter{endnote}{0}
