% Copyright (c) 2010 Free Software Foundation
%% Permission is granted to copy, distribute and/or modify this
%% document under the terms of the GNU Free Documentation License,
%% Version 1.3 or any later version published by the Free Software
%% Foundation; with no Invariant Sections, no Front-Cover Texts, and
%% no Back-Cover Texts. A copy of the license is included in the
%% file called ``gfdl.tex''.


\chapter{Эпилог от Сэма Вильямса: Сокрушительное одиночество}
\chaptermark{Эпилог: Сокрушительное одиночество}

[РМС: Поскольку глава написана лично Сэмом Вильямсом, я внёс в неё ряд правок, помеченных буквами РМС или просто квадратными скобками. Они проясняют юридические и технические моменты. Также я удалил несколько откровенно враждебных и неинформативных фрагментов. Ещё я добавил несколько примечаний, чтобы ответить на кое-какие вопросы. Вильямс тоже добавлял некоторые правки, но они никак не отмечены в тексте.]

Писать биографию живого человека -- всё равно что ставить пьесу. Действие, что разворачивается на сцене, не идёт ни в какое сравнение с драмой, что закипает за кулисами.

В книге \enquote{Автобиография Малкольма Икса} Алекс Хейли даёт читателю редкую возможность погрузиться в эту закулисную драму. Выйдя из роли призрака-повествователя, Хейли зачитывает эпилог книги собственным голосом. Эпилог этот объясняет, как внештатный репортёр, которого \enquote{Нация ислама} поначалу называла \enquote{инструментом} и \enquote{шпионом}, сумел пробиться сквозь личные и политические барьеры, чтобы изложить на бумаге историю Малкольма Икса.

Я не решаюсь сравнивать свою книгу с \enquote{Автобиографией Малкольма Икса}, но я очень благодарен Хейли за его пример откровенного эпилога. Целый год он служил своего рода инструкцией к тому, как справляться с героем биографии, который целую карьеру построил на несоглашательстве. [РМС: Я построил свою карьеру на отрицании вещей, которые у других людей даже не вызывали вопросов, но если я иногда выгляжу или на самом деле являюсь нонконформистом, то это не намеренная самоцель, а лишь следствие.] С самого начала я собирался завершить свою книгу таким же образом, чтобы выразить уважение Хейли и дать читателям понять, как появилась эта книга.

Итак, история эта начинается в оклендских апартаментах и проходит через Кремниевую долину, Мауи, Бостон и Кембридж. В конечном счёте это история связывает два города: книжную столицу мира Нью-Йорк, и книжную столицу округа Сонома -- калифорнийский Севастополь.

Апрель 2000 года, я пишу статьи для злополучного сайта BeOpen.com. В числе моих первых заданий -- взять по телефону интервью у Ричарда Мэттью Столлмана. Интервью проходит хорошо, настолько хорошо, что Slashdot (\url{http://www.slashdot.org}), популярнейший \enquote{новостной ресурс для гиков}, размещает у себя ссылку на него. Через пару часов серверы BeOpen взвыли от нагрузки -- на сайт повалила публика.

Кажется, что на этом история и должна закончиться. Однако 3 месяца спустя, в городе Монтерей, куда я приехал на O'Reilly Open Source Conference, я получаю электронное письмо от Трейси Паттисон, менеджера по внешним делам крупного нью-йоркского издательства:

\begin{quote}
Кому: \url{sam@BeOpen.com}\\Тема: Интервью с РМС\\Когда: понедельник, 10 июля 2000 года 15:56:37 -0400

Уважаемый господин Вильямс!

Я с огромным интересом читаю Ваше интервью, взятое у Ричарда Столлмана для BeOpen. Я вообще интересуюсь РМС и его работой, так что была рада найти Ваш текст. Думаю, у Вас отлично получилось ухватить и передать тот дух, которым пронизано всё, что Столлман пытается делать для GNU-Linux и фонда свободного ПО.

Однако мне хотелось бы почитать про это куда больше, и думаю, я в этом желании не одинока. Как Вы считаете, можно ли найти больше информации и/или её источников, чтобы обновить и расширить Ваше интервью, рассказать больше о личности Столлмана? Может быть, рассказать побольше о его жизни, чтобы заинтересовать и просветить людей вне узкого круга хакеров и программистов?
\end{quote}

В конце письма -- просьба позвонить для дальнейшего обсуждения идеи. Я, конечно, звоню. Трейси рассказывает мне, что её компания запускает новую линейку электронных книг, и ей нужны интересные истории, которые понравятся тестовой аудитории. Формат электронной книги включает в себя 30 тысяч слов на 100 листах, и Трейси подумала, что этого достаточно для жизнеописания какой-нибудь крупной фигуры хакерского сообщества. Её начальству понравилась эта идея, так что она стала подыскивать подходящие кандидатуры, и наткнулась на моё интервью. После чего тут же написала мне.

Трейси спрашивает меня, хочу ли я заняться написанием полноценной биографии.

В тот же миг я отвечаю утвердительно. Трейси предлагает мне составить синопсис, который можно будет показать её начальству. Через пару дней я высылаю ей текст. Ещё спустя неделю Трейси пишет, что начальство даёт зелёный свет.

Должен признать, идея привлечь Столлмана к работе над книгой пришла ко мне далеко не сразу. Мне довелось освещать зарождение движения открытого кода, и тогда я познакомился со столлмановской одержимостью, получив с десяток электронных писем, упрекающих меня за использование названия \enquote{Linux} вместо \enquote{GNU/Linux}.

С другой стороны, я понимал, как Столлман жаждет донести свои идеи до широкой публики. Наверное, предложи я ему участвовать с этой целью, он был бы куда покладистее. Если же нет, то я всегда мог положиться на прорву документов, интервью, записей онлайн-дискуссий, которые Столлман щедро оставляет в интернете. Собрать из этого материала неофициальную биографию нетрудно.

Начав собирать материал, я натолкнулся на эссе \enquote{Свобода или авторское право?}, которую Столлман опубликовал в журнале МТИ \textit{Technology Review} за июнь 2000 года. Эссе называет электронные книги одним из рассадников \enquote{грехов} программного обеспечения. В нём Столлман сетует, что для чтения таких книг людям приходится использовать собственнические программы, а защита от несанкционированного копирования реализуется очень жёсткими методами. Читатели не загружают обычный файл HTML или PDF, вместо этого они получают зашифрованный файл, для которого нужно купить уникальный ключ расшифровки. Любая попытка открыть зашифрованный файл без ключа трактуется Законом об авторском праве в цифровую эпоху как уголовное преступление. Таким же преступлением считается преобразование файла в открытый формат, даже если читатель делает это только для себя. Читатель не имеет права одолжить, скопировать или подарить электронную книгу, как это люди делают с обычными бумажными книжками. Они имеют право читать её только на авторизованном устройстве, говорит Столлман:

\begin{quote}
В использовании бумажных книг мы всё ещё свободны. Но толку от этого будет мало, если электронные книги их вытеснят. Благодаря \enquote{электронным чернилам} можно будет загружать всё новый и новый текст на один и тот же лист бумаги, так что даже газеты могут уйти в прошлое. Только представьте: больше нет магазинов подержанных книг; вы больше не можете одолжить у друга книгу почитать; больше нельзя взять книгу в общественной библиотеке -- никаких больше \enquote{утечек}, дающих возможность читать бесплатно. (И, судя по Microsoft Reader, больше никаких анонимных покупок книг). Именно такой мир издатели готовят для нас. \endnote{\enquote{Freedom -- Or Copyright?} (May, 2000), \url{http://www.technologyreview.com/articles/stallman0500.asp}.}
\end{quote}

Не стоит и говорить о том, как эссе меня обеспокоило. Я ведь никогда не обсуждал с Трейси, какое ПО и какую лицензию её компания будет использовать для будущей электронной книги. Упомянув статью в \textit{Technology Review}, я прошу её рассказать мне о политике компании в отношении электронных книг. Трейси обещает скоро ответить мне.

Не желая сидеть без дела, я звоню Столлману и рассказываю ему о будущей книге. Ричард сильно заинтересован, но и обеспокоен тоже. \enquote{Вы читали моё эссе об электронных книгах?} -- спрашивает он.

Я отвечаю: да, прочитал, и теперь жду ответа от издателя. Ричард выдвигает 2 условия: во-первых, он не хочет поддерживать механизм лицензирования электронных книг, против которого выступает в принципе, во-вторых -- не хочет торговать лицом. \enquote{Я не хочу участвовать в чём-то, что выставило бы меня лицемером}, -- говорит он.

Для Столлмана проблемы с софтом не так важны, как проблемы с авторским правом. Он говорит, что готов смотреть сквозь пальцы на любое несвободное ПО, которое использует издатель или его партнёры, если только издатель разрешит свободно копировать будущую книгу. В качестве возможного примера Ричард приводит \enquote{Плющ} Стивена Кинга. В июне 2000 года Кинг на своём официальном сайте объявил, что начнёт самостоятельно публиковать книгу по частям, и будет продолжать писать её, пока как минимум 75\% читателей будут платить взносы по \$1 каждый. План сработал: к августу Кинг опубликовал две части и принялся за третью.

\enquote{Я готов принять какую-нибудь такую схему, -- говорит Столлман, -- если это не помешает свободному копированию}. [РМС: Насколько я помню, я также поднял вопрос о шифровании, на это указывает и дальнейший текст. Я бы не согласился опубликовать книгу в таком виде, чтобы для её чтения была \textit{необходима} несвободная программа.]

Конечно, я передаю все его слова Трейси. Чувствуя, что мы без труда придём к справедливому соглашению, я снова звоню Столлману и назначаю первое интервью для книги. Ричард соглашается, не поднимая снова тему о правовых вопросах. Вскоре после первого интервью я спешу назначить и второе, стараясь втиснуть его перед отъездом Столлмана в двухнедельный отпуск на Таити. [РМС: Это был не совсем отпуск, я там тоже выступил с речью.]

И уже когда Ричард отдыхает на Таити, от Трейси приходят плохие новости: юридический отдел её компании отказывается менять лицензионное соглашение для будущей электронной книги. Если читатели захотят сделать книгу переносимой, им придётся [сначала взломать шифрование, чтобы перевести книгу в свободный формат вроде HTML. Конечно, это незаконно и может повлечь уголовную ответственность.]

У меня на руках только два свежих интервью, и для книги мне нужно куда больше нового материала. Так что я тут же вылетаю в Нью-Йорк, чтобы встретиться со своим агентом и с Трейси -- попытаться найти какой-нибудь компромисс.

Сначала я встречаюсь со своим агентом, Хеннингом Гуттманом -- впервые в жизни. Он преисполнен пессимизма в отношении возможности компромисса, по крайней мере, со стороны издателя. К этому времени крупные издательства уже смотрят с подозрением на электронный формат, и нисколько не желают экспериментировать с лицензионными соглашениями, чтобы не дать потребителям шансов читать бесплатно. Хеннинг специализируется на технической литературе, и его интригует характер моего затруднительного положения. Я рассказываю ему о двух уже взятых интервью и о том, как пообещал не публиковать книгу на таких условиях, чтобы Столлман стал \enquote{выглядеть лицемером}. Хеннинг соглашается, что я теперь связан этическими обязательствами, и предлагает выставить это нашей позицией в переговорах.

Если это не сработает, говорит Хеннинг, мы всегда можем прибегнуть к методу кнута и пряника. Пряником будет реклама, которую получит издательство при публикации электронной книги, уважающей этические принципы хакерского сообщества. Кнутом будут риски, сопровождающие публикацию книги, которая не уважает эти принципы. Дмитрий Скляров станет интернет-знаменитостью только через 9 месяцев, но мы уже знаем, что взлом защищённой электронной книги каким-нибудь предприимчивым программистом это лишь вопрос времени. Мы также понимаем, что опубликовать зашифрованную электронную книгу о Ричарде Столлмане это всё равно что написать на её обложке \enquote{Взломай меня}.

После встречи с Хеннингом я звоню Столлману. Надеясь сделать пряник как можно слаще, я обсуждаю с ним несколько возможных компромиссов. Что если издатель выпустит книгу под [двойной] лицензией, как сделала компания Sun Microsystems со своим пакетом OpenOffice? Ведь издатель мог бы опубликовать защищённую DRM \endnote{РМС: Вильямс тут пишет \enquote{коммерческую}, что не совсем верно, потому что это слово означает \enquote{связанную с бизнесом}, но ведь любая книга, выпущенная компанией -- коммерческая.} версию электронной книги со всеми [присущими ей] преимуществами, и вместе с нею -- свободно копируемую HTML-версию в менее эстетичном виде.

Ричард отвечает, что не возражает против [двойного лицензирования], но ему не нравится идея об ухудшении или ограничении свободно копируемой версии. Кроме того, по его словам, [это совсем другой случай, хотя бы потому, что тут он может] контролировать результат. Например, отказавшись сотрудничать.

[РМС: Дело в том, что для меня неправильно было бы согласиться на ограниченную версию. Я могу одобрить свободную версию OpenOffice, потому что свободная версия наряду с несвободной это лучше чем ничего, при этом несвободную версию я отвергаю. Здесь нет никакого противоречия с моими принципами, потому что Sun не спрашивала у меня одобрения для несвободной версии, и я не несу ответственности за её существование. Но если бы я согласился с ограниченной несвободной версией книги, ответственность за это легла бы уже на меня.]

Я предлагаю ещё несколько вариантов, но толку от этого немного. Я добился от Ричарда только одной уступки [РМС: То есть, одного компромисса]: лицензия электронной книги ограничит все формы раздачи одним только \enquote{распространением с некоммерческими целями}.

В конце разговора Столлман предлагает мне сообщить издателю о моём обещании, что книга будет [раздаваться свободно]. Я отвечаю, что не могу на это пойти [РМС: хотя, вообще-то, в самом начале он согласился на мои условия], но говорю, что без сотрудничества с ним книги не будет. Ричард, видимо, удовлетворён этим -- он говорит на прощание своё фирменное: \enquote{Удачного хакерства} и кладёт трубку.

На следующий день я вместе с Хеннингом иду на встречу с Трейси. Она сообщает \enquote{радостную} весть: издатель разрешает копировать фрагменты текста, но длиной не более 500 слов. Хеннинг отвечает, что этого явно недостаточно для того, чтобы выполнить мои этические обязательства перед Столлманом. На это Трейси говорит, что её компания сама связана определёнными обязательствами перед онлайн-продавцами вроде Amazon. Если издатель разрешит полностью копировать книгу, он наверняка получит массу исков от своих партнёров. Тут уж остаётся надеяться на душевный порыв целой индустрии. Или на то, что Столлман вдруг поступится своими принципами. Или же мне самому придётся как-то выпутываться из положения. Например, нарушить свои договорённости со Столлманом и выжать максимум из уже полученных интервью, либо вообще забыть об этой книге, нарушив устную договорённость с Трейси -- в этом случае, впрочем, совесть можно сохранить чистой, сославшись на журналистскую этику.

Встреча закончена, мы с агентом едем в бар на Третьей авеню. Я звоню Столлману с мобильника Хеннинга, но никто не отвечает, так что я оставляю ему сообщение. Потом Хеннинг куда-то уходит, давая мне время собраться с мыслями, но скоро возвращается и протягивает мне телефон.

\enquote{Это Столлман}.

Разговор не задаётся с самого начала. Я вкратце пересказываю ему наши переговоры с Трейси.

\enquote{Ну, -- без обиняков отвечает Ричард, -- и почему мне должно быть не плевать на их деловые обязательства?}

Я сбивчиво отвечаю что-то вроде: наверное, потому, что очень непросто вынудить крупное издательство начать судебную войну со своими партнёрами ради электронной книжки объёмом в 30 тысяч слов. [РМС: Скрытый смысл его слов гласил: ну не можешь же ты отказаться от сделки из-за одного только принципа.]

\enquote{А ты не понимаешь? -- говорит Столлман. -- Именно это я и хочу сделать. Хочу добиться знаковой победы. Чтобы они сделали выбор между бизнесом и свободой}.

Слова \enquote{знаковая победа} отдаются эхом в моей голове. Я смотрю на поток пешеходов снаружи. Когда мы входили в бар, я с удовольствием заметил, что он находится менее чем в полуквартале от угла улиц, который Ramones увековечили в своей песне \enquote{53rd and 3rd} 1976 года -- знаковой для меня песне, потому что я в своё время обожал играть её. Подобно вечно разочарованному уличному бродяге из этой песни, я давно уже чувствую, что всё быстротечно, всё распадается так же быстро, как и образуется. Такая вот ирония. Я несколько недель провёл в снисходительных записях жалоб различных людей, и теперь вдруг пытаюсь совершить редчайший подвиг: добиться компромисса от Ричарда Столлмана. Я продолжаю хмыкать и хныкать, отстаивая позицию издателя и демонстрируя симпатию к ней, и Столлман набрасывается на меня, как почуявший кровь дикий зверь.

\enquote{И что, это всё? Собираешься просто кинуть меня? Собираешься прогнуться под них?}

[РМС: Приведённые Вильямсом цитаты показывают, что он совершенно неверно понял наш разговор. Он тут сравнивает меня с хищником, но я всего лишь отказывался от обидной для меня сделки. В конце концов, я уже согласился на некоторые компромиссы, и тут он просит меня полностью предать свои принципы. Конечно, я отказался от этого, я часто это делаю, и люди из-за этого говорят, что я \enquote{вообще отказался идти на компромисс}, что неверно -- прочтите, например, статью: \url{http://www.gnu.org/philosophy/compromise.html}. Я в тот момент боялся, что он собирается нарушить нашу договорённость и опубликовать книгу с DRM, несмотря на мой отказ. Так что я почувствовал не \enquote{кровь}, а возможное предательство.]

Я снова поднимаю вопрос о двойном авторском праве.

\enquote{Ты имеешь в виду лицензию}, -- поправляет меня Ричард.

\enquote{Да, лицензию. Авторское право. Да что угодно}, -- отвечаю я, внезапно ощущая себя раненым тунцом, что пробивается сквозь густую плазму.

\enquote{А-а, да почему ты просто не сделал, нахрен, то, что я сказал сделать!} -- кричит он. [РМС: Думаю, это совершенно искажённая цитата, потому что я не использую слово \enquote{нахрен}, и потому что выражения явно не соответствуют обстоятельствам. Такими выражениями начальники распекают подчинённых, а Вильямс не был моим подчинённым. К тому же, я понимал, что он связан обязательствами. Так что я не мог разговаривать с ним в таких выражениях. Он восстанавливал диалог по своим заметкам, а не по диктофону, и потому исказил формулировки.]

Я продолжаю отстаивать позицию издателя, и Столлман -- я в точности записал его последнюю реплику -- весьма резко заканчивает разговор: \enquote{Мне плевать. То, что они делают -- это зло, а я зло не поддерживаю. До свидания}. [РМС: Похоже, в тот момент я подумал, что он никогда не примет моё \enquote{нет}, и единственное, что мне остаётся, это повесить трубку.]

Как только я кладу трубку, Хеннинг подаёт мне кружку Гиннесса. \enquote{По-моему, тебе это позарез нужно, -- смеётся он, -- прям видно, как тебя начало трясти к концу}.

Меня действительно трясёт. Трясёт, пока я не опустошаю кружку до половины. Так странно было слышать, что я, оказывается, эмиссар \enquote{зла}. [РМС: Слова, приведённые в завершающей реплике, относились к издателю, а не лично к Вильямсу. Если он воспринял это на свой счёт, это может говорить о том, что он взял на себя этическую ответственность за сделку, которую заставил меня принять вначале.] Ещё страннее всё это выглядит, если вспомнить, что каких-то 3 месяца назад я сидел в своей оклендской квартире в раздумьях о том, что писать дальше. А теперь я после встречи с представителями издательств сижу в незнакомом мне уголке мира, о котором я слышал только в песнях, и пью пиво с агентом, которого раньше в глаза не видел и знал только по деловой переписке. Какой-то сюрреализм, похожий на сыро смонтированный фильм.

Тут включается мой внутренний абсурдометр, и я захожусь в судорогах смеха. Хеннинг, наверное, думает, что видит перед собой очередного душевно неуравновешенного писателя, который переживает свой внеочередной эмоциональный срыв. А я только-только начинаю оценивать циничную прелесть своего положения. Состоится сделка или нет -- у меня уже есть о чём рассказать. Остаётся только подыскать место для рассказа. Отсмеявшись, я поднимаю кружку.

\enquote{Добро пожаловать на передовую, друг мой, -- говорю я, чокаясь с агентом, -- теперь расслабляемся и получаем удовольствие}.

Будь эта история в самом деле пьесой, здесь тоже было бы самое место романтическому моменту. Напряжённость нашей встречи обескуражила Трейси, и потому она приглашает нас с Хеннингом пойти выпить с нею и её коллегами. Мы выходим из бара на Третьей авеню и направляемся в Ист-Виллидж. Там я общаюсь с Трейси, избегая любых деловых разговоров. Очень спокойное общение, одно удовольствие. Прежде чем расстаться, мы уговариваемся встретиться завтра вечером. И в следующую встречу мы беззаботно болтаем, так что электронная книга о Столлмане становится каким-то далёким воспоминанием.

Наконец, я возвращаюсь в Окленд и обзваниваю друзей и знакомых журналистов, чтобы рассказать о своих злоключениях. Многие журят меня за то, что я слишком много наобещал Столлману на первых переговорах. [РМС: Кто прочитал всю книгу -- знает, что я никогда не нарушаю обещания.] Бывший преподаватель журналистской школы предлагает мне махнуть рукой на слова Столлмана о \enquote{лицемерии} и всё же написать книгу. Знакомые с медийной смекалкой Ричарда журналисты выражают сочувствие, но сходятся в одном: решать мне.

Я решаю закинуть книгу в долгий ящик. Даже с этими интервью я не особенно богат на материал. К тому же, это даёт мне возможность видеться с Трейси без необходимости подключать Хеннинга. К Рождеству мы обмениваемся визитами: сначала она летит ко мне на Западное побережье, потом я снова лечу в Нью-Йорк. За день до Нового года я делаю предложение. В феврале я собираю вещи, включая ноутбук и записи о Столлмане, и мы выезжаем в аэропорт им. Кеннеди. 11 мая мы женимся. Вот тебе и неудавшаяся книга!

Летом я всё-таки начинаю подумывать о том, чтобы развернуть свои записи в полноценную статью для журнала. С точки зрения этики, тут всё чисто, потому что в нашей изначальной договорённости не было ни слова о традиционных печатных изданиях. Честно сказать, уверенности мне придали и 8 месяцев молчания между мной и Столлманом. С момента нашего последнего телефонного разговора я получил от него лишь пару электронных писем, в которых он отчитал меня за использование \enquote{Linux} вместо \enquote{GNU/Linux} в парочке статей для веб-журнала \textit{Upside Today}. Если не обращать на это внимания, я всё это время наслаждался тишиной и спокойствием. Так что в июне я решаю написать статью о Столлмане размером в 5 тысяч слов. И слова просто хлынули из меня. Полагаю, что расстояние восстановило должную эмоциональную перспективу моего взгляда на Ричарда.

В июле, спустя год после того момента, как я получил первое письмо от Трейси, мне звонит Хеннинг. Он сообщает, что O'Reilly \& Associates, издательство из калифорнийского Севастополя, заинтересовано в написании биографии Столлмана. [РМС: Я смутно припоминаю, что это я посоветовал связаться с O'Reilly, но после стольких лет уже не уверен.] Это радостная новость. О'Рейли -- тот самый издатель, что опубликовал эссе Эрика Реймонда \enquote{Собор и Базар}, и я думаю, что из всех издателей мира он лучше всего понимает проблемы, которые убили мои ранние попытки написать книгу. Тем более, я уже использовал книгу O'Reilly \enquote{Открытый код} как источник информации, и знал, что лицензия некоторых её глав, включая написанную Столлманом, разрешает свободное копирование. Всё это очень обнадёживает меня.

Но, конечно же, возникают проблемы. Через Хеннинга я узнаю, что О'Рейли собирается не только опубликовать книгу, но и распространять её через свой сервис Safari Tech Books Online. А его лицензионное соглашение предусматривает специальные ограничения. \endnote{\enquote{Safari Tech Books Online; Subscriber Agreement: Terms of Service} \url{http://my.safaribooksonline.com/termsofservice}. По состоянию на декабрь 2009 года, сервис требует собственнической программы для чтения электронных книг, так что людям следовало бы отказаться от его использования.} Однако Хеннинг говорит, что О'Рейли собирается разрешить читателям копировать и раздавать текст книги. Так что у меня, как у автора, возникает выбор между двумя лицензиями: OPL и GNU FDL.

Я принялся изучать каждую из них. Лицензия открытых публикаций (Open Publication License, OPL) \endnote{\enquote{The Open Publication License: Draft v1.0} (June 8, 1999), \url{http://opencontent.org/openpub/}.} даёт читателям право копировать и раздавать текст, полностью или частично, на любых носителях, будь они \enquote{физическими или электронными}, если только копии сохранят лицензию OPL. Также она позволяет редактировать текст при соблюдении определённых условий. Наконец, Лицензия открытых публикаций предусматривает возможности ограничить редактирование текста без предварительного согласия автора.

Лицензия свободной документации GNU (GNU Free Documentation License, GFDL) также разрешает копировать и раздавать документы на любых носителях, если лицензия сохраняется. \endnote{\enquote{The GNU Free Documentation License: Version 1.3} (November, 2008), \url{http://www.gnu.org/copyleft/fdl.html}.} Разрешается и редактировать текст на определённых условиях. Но есть и отличия от OPL. Например, она не предусматривает возможности ограничивать редактирование. В частности, автор не может запрещать редактирование, которое способно породить конкурирующий продукт. Но если такой продукт захотят издать тиражом более 100 копий, потребуется снабдить переднюю и заднюю обложки специальными надписями.

Я не забыл зайти и на страницу проекта GNU, посвящённую различным лицензиям.\endnote{See \url{http://www.gnu.org/philosophy/license-list.html}.} Там Столлман, в частности, критикует OPL. Он недоволен возможностями, которые даёт эта лицензия для ограничения редактирования текста. Авторам, которые хотят избежать всяких ограничений для производных работ, лучше выбирать GNU FDL. Потому что, как говорит Столлман, даже отказ автора выбирать одну из ограничительных возможностей OPL не гарантирует, что эти возможности не выберут авторы производных работ.

Внимание, которое обе лицензии уделяют редактированию, отражает их первоначальную цель -- дать пользователям возможность улучшить руководства к программам и передать их сообществу. Моя книга как будто не собирается быть руководством к программе, так что я не беспокоюсь о правовых условиях редактирования. Единственное, что меня заботит -- право читателей копировать текст и свободно раздавать экземпляры будущей электронной книги, как если бы это была реальная, бумажная книжка. Обе лицензии прекрасно подходят для этого, так что я без колебаний подписываю контракт, который О'Рейли высылает мне.

Впрочем, мысль о неограниченном редактировании интригует меня. В первых моих переговорах с Трейси я рассказывал ей о преимуществах GPL-подобных лицензий для электронных книг. Минимальный эффект, которого можно добиться такой лицензией -- хорошая реклама. А в лучшем случае можно привлечь читателей к написанию книги. Мне, как автору, очень нравится мысль о читателях, которые делают мою работу, при условии, что я получаю максимальное вознаграждение. Да и потом, наблюдать за развитием книги со стороны очень интересно. Воображение рисует мне поздние версии книги, обрамлённые массой комментариев читателей, подобно моим постам на онлайн-ресурсах.

Фантазия эта берёт за основу Xanadu (\url{http://www.xanadu.com}) -- концептуальный программный проект, основанный Тедом Нельсоном в 1960 году. На O'Reilly Open Source Conference в 1999 году меня поразила первая демонстрация [свободного] ответвления этого проекта -- Udanax. На ней производный документ показали рядом с исходным, и одно нажатие клавиши отобразило исходные строки с их ответвлениями в производном документе. Электронную биографию Столлмана не обязательно реализовывать поверх Udanax, но с другой стороны, почему бы не дать читателям возможность поиграться с нею?

Лори Петрицки, мой редактор из O'Reilly, велит мне выбрать между OPL и GFDL, и я снова включаю фантазию. На дворе сентябрь 2001 года, и электронные книги мертвы. Многие издательства, включая то, в котором работает Трейси, свернули свои проекты с электронными книгами. Я задаюсь вопросом: если бы эти компании смотрели на электронные книги не как на бизнес, а как на способ создания сообществ -- что это изменило бы?

Я даю знать Столлману, что работа над книгой возобновилась, и что O'Reilly требует от меня выбрать лицензию для неё. Я добавляю, что склоняюсь к OPL, потому что не хочу давать конкурентам О'Рейли возможности выпустить такую же книгу под собственной обложкой. Ричард на это возражает, что О'Рейли уже использовал раньше GFDL несколько раз. Я машу рукой на прошлогодние события и предлагаю Столлману новую сделку: я выбираю GFDL, а он даёт мне новые интервью и помогает О'Рейли с публикацией книги. Ричард соглашается давать интервью, но говорит, что его участие в рекламных мероприятиях будет зависеть от содержания книги. Я нахожу эти условия совершенно справедливыми, и назначаю встречу в Кембридже на 17 декабря 2001 года.

Дату встречи я выбрал так, чтобы совместить её с поездкой Трейси в Бостон по делам. За пару дней до отъезда она предлагает мне пригласить Столлмана на ужин.

\enquote{Если уж на то пошло, -- говорит она, -- именно благодаря ему мы вместе}.

Я пишу Ричарду электронное письмо с приглашением, и ответное согласие не заставляет себя ждать. Когда я приезжаю в Бостон, я забираю Трейси из отеля, и мы спускаемся в подземку, чтобы добраться до Массачусетского технологического института. В Техносквере отыскиваем нужный кабинет и стучимся в дверь. Слышно, как Столлман увлечённо разговаривает с кем-то.

\enquote{Надеюсь, вы не возражаете}, -- говорит он, открывая дверь. Оказалось, что его собеседник -- молодая женщина лет двадцати пяти по имени Сара.

\enquote{Я позволил себе пригласить ещё кое-кого на ужин}, -- объясняет он, улыбаясь своей кошачьей улыбкой, которую я потом снова увижу в ресторане Пало-Альто.

Если честно, то я не слишком удивлён этим. Несколько недель тому назад мать Столлмана любезно поделилась со мной новостью о подруге сына. \enquote{Я знаю, что они уже вдвоём ездили в Японию, чтобы Ричард получил там премию Такеда}, -- добавила тогда Липпман. \endnote{Увы, я проморгал новость о присуждении Столлману, Линусу Торвальдсу и Кену Сакамуре премии фонда Такеда, и узнал об этом только когда Ричард уже отправился в Японию. Чтобы узнать побольше о премии Такеда размером в миллион долларов, посетите её официальный сайт \url{http://www.takeda-foundation.jp}.}

По дороге в ресторан мы разговариваем о том, как Ричард и Сара познакомились. Их рассказ кажется нам удивительно знакомым. Сара работала над своей фантастической книгой, услышала о Столлмане, и решила вписать его в книгу как прототип одного из героев. Чтобы поближе узнать Ричарда, она договорилась об интервью, и после этого всё завертелось. В начале 2001 года они начали встречаться.

\enquote{Меня восхищает, как Ричард создал целое политическое движение, чтобы решить свою личную проблему}, -- говорит Сара, объясняя свою симпатию к Столлману.

Моя жена тут же спрашивает: \enquote{Что за проблема такая?}

\enquote{Сокрушительное одиночество}, -- следует ответ.

Во время обеда я позволяю женщинам болтать между собой, а сам тем временем пытаюсь понять, насколько мягче стал Столлман за последний год, если вообще стал. Не вижу ни одной причины считать, что это произошло. Хотя Ричард куда более игрив и мил по сравнению с тогдашними нашими разговорами, его колючки тоже никуда не делись. В какой-то момент моя жена восклицает \enquote{не дай бог!}, и Ричард тут же отпускает свою характерную колкость.

\enquote{Не хочу разбивать тебе сердце, но бога нет}, -- говорит он. [РМС: Наверное, это вышло чересчур всерьёз. Он мог бы справедливо упрекнуть меня в занудности, но не в колкости.]

Когда ужин закончился и Сара ушла, Столлман как будто ослабил свои защитные барьеры. Мы идём в книжный магазин неподалёку, и он признаётся, что последний год здорово поменял его взгляды на жизнь. \enquote{Я думал, что всегда буду один, -- говорит Ричард, -- и очень рад, что ошибался}.

Перед расставанием Столлман вручает мне \enquote{удовольственную карточку} -- визитку с его адресом, телефонным номером и любимыми занятиями (\enquote{обмен хорошими книгами, вкусной едой, экзотической музыкой и танцами}), чтобы я мог подготовиться к последнему интервью.

На следующей день за едой Ричард выглядит ещё более влюблённым, чем вчера. Вспоминая дебаты в общежитии Карриер-Хаус о плюсах и минусах сыворотки бессмертия, он выражает надежду, что не за горами открытие ключа к вечной жизни. \enquote{Сейчас я счастлив, и потому хочу пожить подольше}, -- признаётся он.

Я повторяю реплику Сары о \enquote{сокрушительном одиночестве}. Столлман на это отвечает, что нет связи между физическим или духовным одиночеством, и одиночеством в хакерском понимании. \enquote{Делиться кодом тебя побуждает дружба, но дружба эта довольно ограниченна}, -- говорит он. Позже Ричард признаётся, что физическое одиночество и страх перед абсолютным одиночеством [РМС: то есть, даже в хакерском понимании] сыграли важную роль в его решимости основать проект GNU.

\enquote{Моё увлечение компьютерами ничем больше и не объясняется, -- рассказывает он, -- будь я популярен и не обделён женским вниманием, я бы не уделял всё своё время компьютерам. На меня глубоко повлияло и другое чувство -- вечная беспризорность, когда я терял свой дом раз за разом. Общежития университета я лишился. Лабораторию ИИ просто разрушили. Ощущение надёжной и спокойной гавани больше не возвращалось ко мне. И я стал бороться за его возвращение}.

После такого интервью я уже не могу не ощущать некоторого эмоционального сходства между нами. После реплики Сары, после слов Ричарда я задумываюсь о собственных мотивах написания этой книги. С июля 2000 года я научился ценить как привлекательные, так и отталкивающие черты личности Ричарда Столлмана. Как и Эбен Моглен, я чувствую всю глубину неправильности описания этой личности, как побочной или вредной по отношению к движению за свободное ПО. Во многих отношениях эти противоположные черты настолько сильно определяют друг друга, что почти неразличимы.

[РМС: Вильямс приписывает мне своё восприятие моих черт, как позитивное, так и негативное, но это восприятие -- производное от его собственных представлений о внешности и успешности.]

Я уверен, что далеко не все читатели почувствуют симпатию к герою моей книги\ldots но также я уверен, что очень немногие люди столь же интересны, как личность Ричарда Мэттью Столлмана. Я искренне надеюсь, что GFDL побудит людей добавить своё видение этой личности к нарисованному мной портрету.

\theendnotes
\setcounter{endnote}{0}
