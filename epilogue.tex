% Copyright (c) 2010 Free Software Foundation
%% Permission is granted to copy, distribute and/or modify this
%% document under the terms of the GNU Free Documentation License,
%% Version 1.3 or any later version published by the Free Software
%% Foundation; with no Invariant Sections, no Front-Cover Texts, and
%% no Back-Cover Texts. A copy of the license is included in the
%% file called ``gfdl.tex''.


\chapter{Epilogue from Sam Williams: Crushing Loneliness}
\chaptermark{Epilogue: Crushing Loneliness}

[RMS: Because this chapter is so personally from Sam Williams, I have indicated all changes to the text with square brackets or ellipses, and I have made such changes only to clear up technical or legal points, and to remove passages that I found to be hostile and devoid of information.  I have also added notes labeled `RMS:' to respond to certain points.  Williams has also changed the text of this chapter; changes made by Williams are not explicitly indicated.]

Writing the biography of a living person is a bit like producing a play. The drama in front of the curtain often pales in comparison to the drama backstage.

In \textit{The Autobiography of Malcolm X}, Alex Haley gives readers a rare glimpse of that backstage drama. Stepping out of the ghostwriter role, Haley delivers the book's epilogue in his own voice. The epilogue explains how a freelance reporter originally dismissed as a ``tool'' and ``spy'' by the Nation of Islam spokesperson managed to work through personal and political barriers to get Malcolm X's life story on paper.

While I hesitate to compare this book with \textit{The Autobiography of Malcolm X}, I do owe a debt of gratitude to Haley for his candid epilogue. Over the last 12 months, it has served as a sort of instruction manual on how to deal with a biographical subject who has built an entire career on being disagreeable.  [RMS: I have built my career on saying no to things others accept without much question, but if I sometimes seem or am disagreeable, it is not through specific intention.]  From the outset, I envisioned closing this biography with a similar epilogue, both as an homage to Haley and as a way to let readers know how this book came to be.

The story behind this story starts in an Oakland apartment, winding its way through the various locales mentioned in the book -- Silicon Valley, Maui, Boston, and Cambridge. Ultimately, however, it is a tale of two cities: New York, New York, the book-publishing capital of the world, and Sebastopol, California, the book-publishing capital of Sonoma County.

The story starts in April, 2000. At the time, I was writing stories for the ill-fated web site BeOpen.com. One of my first assignments was a phone interview with Richard M. Stallman. The interview went well, so well that Slashdot (\url{http://www.slashdot.org}), the popular ``news for nerds'' site owned by VA Software, Inc. (formerly VA Linux Systems and before that, VA Research), gave it a link in its daily list of feature stories. Within hours, the web servers at BeOpen were heating up as readers clicked over to the site.

For all intents and purposes, the story should have ended there. Three months after the interview, while attending the O'Reilly Open Source Conference in Monterey, California, I received the following email message from Tracy Pattison, foreign-rights manager at a large New York publishing house:

\begin{quote}
To: \url{sam@BeOpen.com}\\Subject: RMS Interview\\Date: Mon, 10 Jul 2000 15:56:37 -0400

Dear Mr. Williams,

I read your interview with Richard Stallman on BeOpen with great interest. I've been intrigued by RMS and his work for some time now and was delighted to find your piece which I really think you did a great job of capturing some of the spirit of what Stallman is trying to do with GNU-Linux and the Free Software Foundation.

What I'd love to do, however, is read more - and I don't think I'm alone. Do you think there is more information and/or sources out there to expand and update your interview and adapt it into more of a profile of Stallman? Perhaps including some more anecdotal information about his personality and background that might really interest and enlighten readers outside the more hardcore programming scene?
\end{quote}

Tracy ended the email with a request that I give her a call to discuss the idea further. I did just that. Tracy told me her company was launching a new electronic book line, and it wanted stories that appealed to an early-adopter audience. The e-book format was 30,000 words, about 100 pages, and she had pitched her bosses on the idea of profiling a major figure in the hacker community. Her bosses liked the idea, and in the process of searching for interesting people to profile, she had come across my BeOpen interview with Stallman. Hence her email to me.

That's when Tracy asked me: would I be willing to expand the interview into a full-length feature profile?

My answer was instant: yes. Before accepting it, Tracy suggested I put together a story proposal she could show her superiors. Two days later, I sent her a polished proposal. A week later, Tracy sent me a follow up email. Her bosses had given it the green light.

I have to admit, getting Stallman to participate in an e-book project was an afterthought on my part. As a reporter who covered the open source beat, I knew Stallman was a stickler. I'd already received a half dozen emails at that point upbraiding me for the use of ``Linux'' instead of ``GNU/Linux.''

Then again, I also knew Stallman was looking for ways to get his message out to the general public. Perhaps if I presented the project to him that way, he would be more receptive. If not, I could always rely upon the copious amounts of documents, interviews, and recorded online conversations Stallman had left lying around the Internet and do an unauthorized biography.

During my research, I came across an essay titled ``Freedom -- Or Copyright?'' Written by Stallman and published in the June, 2000, edition of the MIT \textit{Technology Review}, the essay blasted e-books for an assortment of software sins. Not only did readers have to use proprietary software programs to read them, Stallman lamented, but the methods used to prevent unauthorized copying were overly harsh. Instead of downloading a transferable HTML or PDF file, readers downloaded an encrypted file. In essence, purchasing an e-book meant purchasing a nontransferable key to unscramble the encrypted content. Any attempt to open a book's content without an authorized key constituted a criminal violation of the Digital Millennium Copyright Act, the 1998 law designed to bolster copyright enforcement on the Internet. Similar penalties held for readers who converted a book's content into an open file format, even if their only intention was to read the book on a different computer in their home. Unlike a normal book, the reader no longer held the right to lend, copy, or resell an e-book. They only had the right to read it on an authorized machine, warned Stallman:

\begin{quote}
We still have the same old freedoms in using paper books. But if e-books replace printed books, that exception will do little good. With ``electronic ink,'' which makes it possible to download new text onto an apparently printed piece of paper, even newspapers could become ephemeral. Imagine: no more used book stores; no more lending a book to your friend; no more borrowing one from the public library -- no more ``leaks'' that might give someone a chance to read without paying. (And judging from the ads for Microsoft Reader, no more anonymous purchasing of books either.) This is the world publishers have in mind for us.\endnote{See ``Freedom -- Or Copyright?'' (May, 2000), \url{http://www.technologyreview.com/articles/stallman0500.asp}.}
\end{quote}

Needless to say, the essay caused some concern. Neither Tracy nor I had discussed the software her company would use nor had we discussed the type of copyright [license] that would govern the e-book's usage. I mentioned the \textit{Technology Review} article and asked if she could give me information on her company's e-book policies. Tracy promised to get back to me.

Eager to get started, I decided to call Stallman anyway and mention the book idea to him. When I did, he expressed immediate interest and immediate concern. ``Did you read my essay on e-books?'' he asked.

When I told him, yes, I had read the essay and was waiting to hear back from the publisher, Stallman laid out two conditions: he didn't want to lend support to an e-book licensing mechanism he fundamentally opposed, and he didn't want to come off as lending support. ``I don't want to participate in anything that makes me look like a hypocrite,'' he said.

For Stallman, the software issue was secondary to the copyright issue. He said he was willing to ignore whatever software the publisher or its third-party vendors employed just so long as the company specified within the copyright that readers were free to make and distribute verbatim copies of the e-book's content. Stallman pointed to Stephen King's \textit{The Plant} as a possible model. In June, 2000, King announced on his official web site that he was self-publishing \textit{The Plant} in serial form. According to the announcement, the book's total cost would be \$13, spread out over a series of \$1 installments. As long as at least 75\% of the readers paid for each chapter, King promised to continue releasing new installments. By August, the plan seemed to be working, as King had published the first two chapters with a third on the way.

``I'd be willing to accept something like that,'' Stallman said. ``As long as it also permitted verbatim copying.'' [RMS: As I recall, I also raised the issue of encryption; the text two paragraphs further down confirms this.  I would not have agreed to publish the book in a way that \textit{required} a nonfree program to read it.]

I forwarded the information to Tracy. Feeling confident that she and I might be able to work out an equitable arrangement, I called up Stallman and set up the first interview for the book. Stallman agreed to the interview without making a second inquiry into the status issue. Shortly after the first interview, I raced to set up a second interview (this one in Kihei), squeezing it in before Stallman headed off on a 14-day vacation to Tahiti. [RMS: That was not a pure vacation; I gave a speech there too.]

It was during Stallman's vacation that the bad news came from Tracy. Her company's legal department didn't want to adjust its [license] notice on the e-books. Readers who wanted to make their books transferable would [first have to crack the encryption code, to be able to convert the book to a free, public format such as HTML. This would be illegal and they might face criminal penalties.]

With two fresh interviews under my belt, I didn't see any way to write the book without resorting to the new material. I quickly set up a trip to New York to meet with my agent and with Tracy to see if there was a compromise solution.

When I flew to New York, I met my agent, Henning Guttman. It was our first face-to-face meeting, and Henning seemed pessimistic about our chances of forcing a compromise, at least on the publisher's end. The large, established publishing houses already viewed the e-book format with enough suspicion and weren't in the mood to experiment with copyright language that made it easier for readers to avoid payment. As an agent who specialized in technology books, however, Henning was intrigued by the novel nature of my predicament. I told him about the two interviews I'd already gathered and the promise not to publish the book in a way that made Stallman ``look like a hypocrite.'' Agreeing that I was in an ethical bind, Henning suggested we make that our negotiating point.

Barring that, Henning said, we could always take the carrot-and-stick approach. The carrot would be the publicity that came with publishing an e-book that honored the hacker community's internal ethics. The stick would be the risks associated with publishing an e-book that didn't. Nine months before Dmitry Sklyarov became an Internet \textit{cause célèbre}, we knew it was only a matter of time before an enterprising programmer revealed how to hack e-books. We also knew that a major publishing house releasing an [encrypted] e-book on Richard M. Stallman was the software equivalent of putting ``Steal This E-Book'' on the cover.

After my meeting with Henning, I called Stallman. Hoping to make the carrot more enticing, I discussed a number of potential compromises. What if the publisher released the book's content under a [dual] license, something similar to what Sun Microsystems had done with OpenOffice, the free software desktop applications suite? The publisher could then release DRM-restricted\endnote{RMS: Williams wrote ``commercial'' here, but that is a misnomer, since it means ``connected with business.''  All these versions would be commercial if a company published them.} versions of the e-book under [its usual] format, taking advantage of all the bells and whistles that went with the e-book software, while releasing the copyable version under a less aesthetically pleasing HTML format.

Stallman told me he didn't mind the [dual-license] idea, but he did dislike the idea of making the freely copyable version inferior to the restricted version. Besides, he said [on second thought, this case was different precisely because he had] a way to control the outcome. He could refuse to cooperate.

[RMS: The question was whether it would be wrong for me to agree to the restricted version.  I can endorse the free version of Sun's OpenOffice, because it is free software and much better than nothing, while at the same time I reject the nonfree version.  There is no self-contradiction here, because Sun didn't need or ask my approval for the non-free version; I was not responsible for its existence. In this case, if I had said yes to the non-freely-copyable version, the onus would fall on me.]

I made a few more suggestions with little effect. About the only thing I could get out of Stallman was a concession [RMS: i.e., a further compromise] that the e-book's  [license] restrict all forms of file sharing to ``noncommercial redistribution.''

Before I signed off, Stallman suggested I tell the publisher that I'd promised Stallman that the work would be [freely sharable]. I told Stallman I couldn't agree to that statement [RMS: though it was true, since he had accepted my conditions at the outset] but that I did view the book as unfinishable without his cooperation. Seemingly satisfied, Stallman hung up with his usual sign-off line: ``Happy hacking.''

Henning and I met with Tracy the next day. Tracy said her company was willing to publish copyable excerpts in a unencrypted format but would limit the excerpts to 500 words. Henning informed her that this wouldn't be enough for me to get around my ethical obligation to Stallman. Tracy mentioned her own company's contractual obligation to online vendors such as Amazon.com. Even if the company decided to open up its e-book content this one time, it faced the risk of its partners calling it a breach of contract. Barring a change of heart in the executive suite or on the part of Stallman, the decision was up to me. I could use the interviews and go against my earlier agreement with Stallman, or I could plead journalistic ethics and back out of the verbal agreement to do the book.

Following the meeting, my agent and I relocated to a pub on Third Ave. I used his cell phone to call Stallman, leaving a message when nobody answered. Henning left for a moment, giving me time to collect my thoughts. When he returned, he was holding up the cell phone.

``It's Stallman,'' Henning said.

The conversation got off badly from the start. I relayed Tracy's comment about the publisher's contractual obligations.

``So,'' Stallman said bluntly. ``Why should I give a damn about their contractual obligations?''

Because asking a major publishing house to risk a legal battle with its vendors over a 30,000-word e-book is a tall order, I suggested. [RMS: His unstated premise was that I couldn't possibly refuse this deal for mere principle.]

``Don't you see?'' Stallman said. ``That's exactly why I'm doing this. I want a signal victory. I want them to make a choice between freedom and business as usual.''

As the words ``signal victory'' echoed in my head, I felt my attention wander momentarily to the passing foot traffic on the sidewalk. Coming into the bar, I had been pleased to notice that the location was less than half a block away from the street corner memorialized in the 1976 Ramones song, ``53rd and 3rd,'' a song I always enjoyed playing in my days as a musician. Like the perpetually frustrated street hustler depicted in that song, I could feel things falling apart as quickly as they had come together. The irony was palpable. After weeks of gleefully recording other people's laments, I found myself in the position of trying to pull off the rarest of feats: a Richard Stallman compromise. When I continued hemming and hawing, pleading the publisher's position and revealing my growing sympathy for it, Stallman, like an animal smelling blood, attacked. 

``So that's it? You're just going to screw me? You're just going to bend to their will?''

[RMS: The quotations show that Williams' interpretation of this conversation was totally wrong.  He compares me to a predator, but I was only saying no to the deal he was badgering me to accept.  I had already made several compromises, some described above; I just refused to compromise my principles entirely away.  I often do this; people who aren't satisfied say I ``refused to compromise at all,'' but that is an exaggeration; see \url{http://www.gnu.org/philosophy/compromise.html}. Then I feared he was going to disregard the conditions he had previously agreed to, and publish the book with DRM despite my refusal.  What I smelled was not his ``blood'' but possible betrayal.]

I brought up the issue of a dual-copyright again.

``You mean license,'' Stallman said curtly.

``Yeah, license. Copyright. Whatever,'' I said, feeling suddenly like a wounded tuna trailing a rich plume of plasma in the water.

``Aw, why didn't you just fucking do what I told you to do!'' he shouted.  [RMS: I think this quotation was garbled, both because using ``fucking'' as an adverb was never part of my speech pattern, and because the words do not fit the circumstances.  The words he quotes are a rebuke to a disobedient subordinate.  I felt he had an ethical obligation, but he was not my subordinate, and I would not have spoken to him as one.  Using notes rather than a recorder, he could not reliably retain the exact words.]

I must have been arguing on behalf of the publisher to the very end, because in my notes I managed to save a final Stallman chestnut: ``I don't care. What they're doing is evil. I can't support evil. Good-bye.''  [RMS: It sounds like I had concluded that he would never take no for an answer, and the only way to end the conversation without accepting his proposition was to hang up on him.]

As soon as I put the phone down, my agent slid a freshly poured Guinness to me. ``I figured you might need this,'' he said with a laugh. ``I could see you shaking there towards the end.''

I was indeed shaking. The shaking wouldn't stop until the Guinness was more than halfway gone. It felt weird, hearing myself characterized as an emissary of ``evil.'' [RMS: My words as quoted criticize the publisher, not Williams personally.  If he took it personally, perhaps that indicates he was starting to take ethical responsibility for the deal he had pressed me to accept.]  It felt weirder still, knowing that three months before, I was sitting in an Oakland apartment trying to come up with my next story idea. Now, I was sitting in a part of the world I'd only known through rock songs, taking meetings with publishing executives and drinking beer with an agent I'd never even laid eyes on until the day before. It was all too surreal, like watching my life reflected back as a movie montage.

About that time, my internal absurdity meter kicked in. The initial shaking gave way to convulsions of laughter. To my agent, I must have looked like a another fragile author undergoing an untimely emotional breakdown. To me, I was just starting to appreciate the cynical beauty of my situation. Deal or no deal, I already had the makings of a pretty good story. It was only a matter of finding a place to tell it. When my laughing convulsions finally subsided, I held up my drink in a toast.

``Welcome to the front lines, my friend,'' I said, clinking pints with my agent. ``Might as well enjoy it.''

If this story really were a play, here's where it would take a momentary, romantic interlude. Disheartened by the tense nature of our meeting, Tracy invited Henning and me to go out for drinks with her and some of her coworkers. We left the bar on Third Ave., headed down to the East Village, and caught up with Tracy and her friends.

Once there, I spoke with Tracy, careful to avoid shop talk. Our conversation was pleasant, relaxed. Before parting, we agreed to meet the next night. Once again, the conversation was pleasant, so pleasant that the Stallman e-book became almost a distant memory.

When I got back to Oakland, I called around to various journalist friends and acquaintances. I recounted my predicament. Most upbraided me for giving up too much ground to Stallman in the preinterview negotiation. [RMS: Those who have read the whole book know that I would never have dropped the conditions.] A former j-school professor suggested I ignore Stallman's ``hypocrite'' comment and just write the story. Reporters who knew of Stallman's media-savviness expressed sympathy but uniformly offered the same response: it's your call.

I decided to put the book on the back burner. Even with the interviews, I wasn't making much progress. Besides, it gave me a chance to speak with Tracy without running things past Henning first. By Christmas we had traded visits: she flying out to the west coast once, me flying out to New York a second time. The day before New Year's Eve, I proposed. Deciding which coast to live on, I picked New York. By February, I packed up my laptop computer and all my research notes related to the Stallman biography, and we winged our way to JFK Airport. Tracy and I were married on May 11. So much for failed book deals.

During the summer, I began to contemplate turning my interview notes into a magazine article. Ethically, I felt in the clear doing so, since the original interview terms said nothing about traditional print media. To be honest, I also felt a bit more comfortable writing about Stallman after eight months of radio silence. Since our telephone conversation in September, I'd only received two emails from Stallman. Both chastised me for using ``Linux'' instead of ``GNU/Linux'' in a pair of articles for the web magazine \textit{Upside Today}. Aside from that, I had enjoyed the silence. In June, about a week after the New York University speech, I took a crack at writing a 5,000-word magazine-length story about Stallman. This time, the words flowed. The distance had helped restore my lost sense of emotional perspective, I suppose.

In July, a full year after the original email from Tracy, I got a call from Henning. He told me that O'Reilly \& Associates, a publishing house out of Sebastopol, California, was interested in the running the Stallman story as a biography. [RMS: I have a vague memory that I suggested contacting O'Reilly, but I can't be sure after all these years.] The news pleased me. Of all the publishing houses in the world, O'Reilly, the same company that had published Eric Raymond's \textit{The Cathedral and the Bazaar}, seemed the most sensitive to the issues that had killed the earlier e-book. As a reporter, I had relied heavily on the O'Reilly book \textit{Open Sources} as a historical reference. I also knew that various chapters of the book, including a chapter written by Stallman, had been published with [license] notices that permitted redistribution. Such knowledge would come in handy if the issue of electronic publication ever came up again.

Sure enough, the issue did come up. I learned through Henning that O'Reilly intended to publish the biography both as a book and as part of its new Safari Tech Books Online subscription service. The Safari user license would involve special restrictions,\endnote{See ``Safari Tech Books Online; Subscriber Agreement: Terms of Service'' \url{http://my.safaribooksonline.com/termsofservice}.  As of December, 2009, these e-books require nonfree reader software, so people should refuse to use them.} Henning warned, but O'Reilly was willing to allow for a copyright that permitted users to copy and share the book's text regardless of medium. Basically, as author, I had the choice between two licenses: the Open Publication License or the GNU Free Documentation License.

I checked out the contents and background of each license. The Open Publication License (OPL)\endnote{See ``The Open Publication License: Draft v1.0'' (June 8, 1999), \url{http://opencontent.org/openpub/}.} gives readers the right to reproduce and distribute a work, in whole or in part, in any medium ``physical or electronic,'' provided the copied work retains the Open Publication License. It also permits modification of a work, provided certain conditions are met. Finally, the Open Publication License includes a number of options, which, if selected by the author, can limit the creation of ``substantively modified'' versions or book-form derivatives without prior author approval.

The GNU Free Documentation License (GFDL), meanwhile, permits the copying and distribution of a document in any medium, provided the resulting work carries the same license.\endnote{See ``The GNU Free Documentation License: Version 1.3'' (November, 2008), \url{http://www.gnu.org/copyleft/fdl.html}.} It also permits the modification of a document provided certain conditions. Unlike the OPL, however, it does not give authors the option to restrict certain modifications. It also does not give authors the right to reject modifications that might result in a competitive book product. It does require certain forms of front- and back-cover information if a party other than the copyright holder wishes to publish more than 100 copies of a protected work, however.

In the course of researching the licenses, I also made sure to visit the GNU Project web page titled ``Various Licenses and Comments About Them.''\endnote{See \url{http://www.gnu.org/philosophy/license-list.html}.} On that page, I found a Stallman critique of the Open Publication License. Stallman's critique related to the creation of modified works and the ability of an author to select either one of the OPL's options to restrict modification. If an author didn't want to select either option, it was better to use the GFDL instead, Stallman noted, since it minimized the risk of the nonselected options popping up in modified versions of a document.

The importance of modification in both licenses was a reflection of their original purpose -- namely, to give software-manual owners a chance to improve their manuals and publicize those improvements to the rest of the community. Since my book wasn't a manual, I had little concern about the modification clause in either license. My only concern was giving users the freedom to exchange copies of the book or make copies of the content, the same freedom they would have enjoyed if they purchased a hardcover book. Deeming either license suitable for this purpose, I signed the O'Reilly contract when it came to me.

Still, the notion of unrestricted modification intrigued me. In my early negotiations with Tracy, I had pitched the merits of a GPL-style license for the e-book's content. At worst, I said, the license would guarantee a lot of positive publicity for the e-book. At best, it would encourage readers to participate in the book-writing process. As an author, I was willing to let other people amend my work just so long as my name always got top billing. Besides, it might even be interesting to watch the book evolve. I pictured later editions looking much like online versions of the \textit{Talmud}, my original text in a central column surrounded by illuminating, third-party commentary in the margins.

My idea drew inspiration from Project Xanadu (\url{http://www.xanadu.com}), the legendary software concept originally conceived by Ted Nelson in 1960. During the O'Reilly Open Source Conference in 1999, I had seen the first demonstration of the project's [free] offshoot Udanax and had been wowed by the result. In one demonstration sequence, Udanax displayed a parent document and a derivative work in a similar two-column, plain-text format. With a click of the button, the program introduced lines linking each sentence in the parent to its conceptual offshoot in the derivative. An e-book biography of Richard M. Stallman didn't have to be Udanax-enabled, but given such technological possibilities, why not give users a chance to play around?\endnote{Anybody willing to ``port'' this book over to Udanax, the free software version of Xanadu, will receive enthusiastic support from me. To find out more about this intriguing technology, visit \url{http://www.udanax.com}.}

When Laurie Petrycki, my editor at O'Reilly, gave me a choice between the OPL or the GFDL, I indulged the fantasy once again. By September of 2001, the month I signed the contract, e-books had become almost a dead topic. Many publishing houses, Tracy's included, were shutting down their e-book imprints for lack of interest. I had to wonder. If these companies had treated e-books not as a form of publication but as a form of community building, would those imprints have survived?

After I signed the contract, I notified Stallman that the book project was back on. I mentioned the choice O'Reilly was giving me between the Open Publication License and the GNU Free Documentation License. I told him I was leaning toward the OPL, if only for the fact I saw no reason to give O'Reilly's competitors a chance to print the same book under a different cover. Stallman wrote back, arguing in favor of the GFDL, noting that O'Reilly had already used it several times in the past. Despite the events of the past year, I suggested a deal. I would choose the GFDL if it gave me the possibility to do more interviews and if Stallman agreed to help O'Reilly publicize the book. Stallman agreed to participate in more interviews but said that his participation in publicity-related events would depend on the content of the book. Viewing this as only fair, I set up an interview for December 17, 2001 in Cambridge.

I set up the interview to coincide with a business trip my wife Tracy was taking to Boston. Two days before leaving, Tracy suggested I invite Stallman out to dinner.

``After all,'' she said, ``he is the one who brought us together.''

I sent an email to Stallman, who promptly sent a return email accepting the offer. When I drove up to Boston the next day, I met Tracy at her hotel and hopped the T to head over to MIT. When we got to Tech Square, I found Stallman in the middle of a conversation just as we knocked on the door.

``I hope you don't mind,'' he said, pulling the door open far enough so that Tracy and I could just barely hear Stallman's conversational counterpart. It was a youngish woman, mid-20s I'd say, named Sarah.

``I took the liberty of inviting somebody else to have dinner with us,'' Stallman said, matter-of-factly, giving me the same cat-like smile he gave me back in that Palo Alto restaurant.

To be honest, I wasn't too surprised. The news that Stallman had a new female friend had reached me a few weeks before, courtesy of Stallman's mother. ``In fact, they both went to Japan last month when Richard went over to accept the Takeda Award,'' Lippman told me at the time.\endnote{Alas, I didn't find out about the Takeda Foundation's decision to award Stallman, along with Linus Torvalds and Ken Sakamura, with its first-ever award for ``Techno-Entrepreneurial Achievement for Social/Economic Well-Being'' until after Stallman had made the trip to Japan to accept the award. For more information about the award and its accompanying \$1 million prize, visit the Takeda site, \url{http://www.takeda-foundation.jp}.}

On the way over to the restaurant, I learned the circumstances of Sarah and Richard's first meeting. Interestingly, the circumstances were very familiar. Working on her own fictional book, Sarah said she heard about Stallman and what an interesting character he was. She promptly decided to create a character in her book on Stallman and, in the interests of researching the character, set up an interview with Stallman. Things quickly went from there. The two had been dating since the beginning of 2001, she said.

``I really admired the way Richard built up an entire political movement to address an issue of profound personal concern,'' Sarah said, explaining her attraction to Stallman.

My wife immediately threw back the question: ``What was the issue?''

``Crushing loneliness.''

During dinner, I let the women do the talking and spent most of the time trying to detect clues as to whether the last 12 months had softened Stallman in any significant way. I didn't see anything to suggest they had. Although more flirtatious than I remembered, Stallman retained the same general level of prickliness. At one point, my wife uttered an emphatic ``God forbid'' only to receive a typical Stallman rebuke.

``I hate to break it to you, but there is no God,'' Stallman said. [RMS: I must have been too deadpan. He could justly accuse me of being a wise guy, but not of rebuking.]

Afterwards, when the dinner was complete and Sarah had departed, Stallman seemed to let his guard down a little. As we walked to a nearby bookstore, he admitted that the last 12 months had dramatically changed his outlook on life. ``I thought I was going to be alone forever,'' he said. ``I'm glad I was wrong.''

Before parting, Stallman handed me his ``pleasure card,'' a business card listing Stallman's address, phone number, and favorite pastimes (``sharing good books, good food and exotic music and dance'') so that I might set up a final interview.

The next day, over another meal of dim sum, Stallman seemed even more lovestruck than the night before. Recalling his debates with Currier House dorm maters over the benefits and drawbacks of an immortality serum, Stallman expressed hope that scientists might some day come up with the key to immortality. ``Now that I'm finally starting to have happiness in my life, I want to have [a longer life],'' he said.

When I mentioned Sarah's ``crushing loneliness'' comment, Stallman failed to see a connection between loneliness on a physical or spiritual level and loneliness on a hacker level. ``The impulse to share code is about friendship but friendship at a much lower level,'' he said. Later, however, when the subject came up again, Stallman did admit that loneliness, or the fear of perpetual loneliness [RMS: at the hacker-to-hacker, community level, that is], had played a major role in fueling his determination during the earliest days of the GNU Project.

``My fascination with computers was not a consequence of anything else,'' he said. ``I wouldn't have been less fascinated with computers if I had been popular and all the women flocked to me. However, it's certainly true the experience of feeling I didn't have a home, finding one and losing it, finding another and having it destroyed, affected me deeply. The one I lost was the dorm. The one that was destroyed was the AI Lab. The precariousness of not having any kind of home or community was very powerful. It made me want to fight to get it back.''

After the interview, I couldn't help but feel a certain sense of emotional symmetry. Hearing Sarah describe what attracted her to Stallman and hearing Stallman himself describe the emotions that prompted him to take up the free software cause, I was reminded of my own reasons for writing this book. Since July, 2000, I have learned to appreciate both the seductive and the repellent sides of the Richard Stallman persona. Like Eben Moglen before me, I feel that dismissing that persona as epiphenomenal or distracting in relation to the overall free software movement would be a grievous mistake. In many ways the two are so mutually defining as to be indistinguishable.

[RMS: Williams objectifies his reactions, both positive and negative, as parts of me, but they are functions also of his own attitudes about appearance, conformity, and business success.]

While I'm sure not every reader feels the same level of affinity for Stallman\ldots I'm sure most will agree [that] few individuals offer as singular a human portrait as Richard M. Stallman. It is my sincere hope that, with this initial portrait complete and with the help of the GFDL, others will feel a similar urge to add their own perspective to that portrait.

\theendnotes
\setcounter{endnote}{0}
