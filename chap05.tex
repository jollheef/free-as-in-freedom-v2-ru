%% Copyright (c) 2002, 2010 Sam Williams
%% Copyright (c) 2010 Richard M. Stallman
%% Permission is granted to copy, distribute and/or modify this
%% document under the terms of the GNU Free Documentation License,
%% Version 1.3 or any later version published by the Free Software
%% Foundation; with no Invariant Sections, no Front-Cover Texts, and
%% no Back-Cover Texts. A copy of the license is included in the
%% file called ``gfdl.tex''.


\chapter{Ручеёк свободы}

[РМС: В этой главе я исправил немало высказываний о моих мыслях и чувствах, и сгладил необоснованную враждебность в описании некоторых событий. Высказывания Вильямса приведены в оригинальном виде, если не указано обратного.]

Спросите любого, кто провёл в обществе Ричарда Столлмана больше минуты, и все они скажут вам примерно одно: забудьте о его длинных волосах, забудьте о его чудачествах, первое что вы заметите -- взгляд. Только раз посмотрите в его зелёные глаза, и поймёте, что видите перед собой настоящего адепта.

Назвать Столлмана одержимым -- преуменьшение. Он не смотрит на вас, он смотрит сквозь вас. Когда вы отводите взгляд из чувства такта, глаза Столлмана начинают жечь вашу голову подобно двум лазерным лучам.

Наверное, поэтому большинство авторов описывают Столлмана в религиозной стилистике. В статье на \textit{Salon.com} за 1998 год под заголовком \enquote{Святитель свободного ПО} Эндрю Леонард называет зелёные глаза Столлмана \enquote{излучающими силу ветхозаветного пророка}\endnote{Andrew Leonard, \enquote{The Saint of Free Software,} \textit{Salon.com} (August 1998)}. Статья 1999 года в журнале \textit{Wired} утверждает, что борода Столлмана делает его \enquote{похожим на Распутина}.\endnote{See Leander Kahney, \enquote{Linux's Forgotten Man,} \textit{Wired News} (March 5, 1999)}. А в Столлмановском досье \textit{London Guardian} его улыбка названа \enquote{улыбкой апостола после встречи с Иисусом}\endnote{\enquote{Programmer on moral high ground; Free software is a moral issue for Richard Stallman believes in freedom and free software,} \textit{London Guardian} (November 6, 1999), \url{http://www.guardian.co.uk/uk/1999/nov/06/andrewbrown}.

Это далеко не все сравнения в религиозном ключе. На сегодняшний день самое радикальное сравнение подобного рода сделал Линус Торвальдс в своей автобиографии \enquote{Просто удовольствия ради: Рассказ нечаянного революционера} (\textit{Just For Fun: The Story of an Accidental Revolutionary} (HarperCollins Publishers, Inc., 2001): 58). Он сказал коротко и ясно: \enquote{Ричард Столлман это бог свободного ПО}.

Ларри Лессиг в своей книге \enquote{Будущее идей} (\textit{The Future of Ideas} (Random House, 2001): 270) проводит параллели между Столлманом и Моисеем:

\begin{quote}

\ldots Как и в случае с Моисеем, был другой лидер, Линус Торвальдс, который довёл до конца переход в земли обетованные, разработав последнюю деталь в конструкторе операционной системы. Как и Моисей, Столлман познал преданность и предательство среди своих последователей. Он не идёт на компромиссы и не прощает, чем вдохновляет многих. Он создал и возглавляет очень важную часть современной культуры. Я безмерно уважаю твёрдость, упорство и принципиальность этого человека, хотя также не могу не уважать и тех, кому хватает смелости критиковать идеи Столлмана и этим навлекать на себя его гнев.

\end{quote}

В последнем интервью я спросил Столлмана, что он думает об этой религиозной окраске его личности и дел. \enquote{Некоторые сравнивают меня с ветхозаветным пророком, потому что пророки эти объявляли плохими и неправильными какие-то обычаи, законы, общественные нормы. Они были бескомпромиссны в моральных вопросах. От них невозможно было откупиться, и за это их презирали}.}

Такие аналогии впечатляют, но не соответствуют действительности. Они рисуют какое-то недосягаемое, сверхъестественное существо, тогда как реальный Столлман уязвим, как и все люди. Понаблюдайте за его глазами некоторое время, и вы поймёте: Ричард не гипнотизировал вас и не сверлил взглядом, он пытался наладить зрительный контакт. Так проявляется синдром Аспергера, тень которого лежит на психике Столлмана. Ричарду трудно взаимодействовать с людьми, он не ощущает контакта, в общении ему приходится опираться на теоретические умозаключения, а не на чувства. Ещё один признак -- периодические погружения в себя. Глаза Столлмана даже при ярком свете могут остановиться и поблекнуть, как у раненого животного, что вот-вот испустит дух.

Впервые я столкнулся с этим странным взглядом Столлмана в марте 1999 года, на \enquote{LinuxWorld Conference and Expo } в Сан-Хосе. Это была конференция для людей и компаний, связанных со свободным программным обеспечением, своеобразный \enquote{вечер признания}. Был вечер таковым и для Столлмана -- он решил принять самое активное участие, донести до журналистов и широкой публики историю проекта GNU и его идеологию.

Тогда я впервые получил руководство по обращению со Столлманом, причём невольно. Это случилось на пресс-конференции, посвящённой выходу GNOME 1.0, свободной графической среды рабочего стола. Сам того не подозревая, я нажал на горячие клавиши взвинчивания Столлмана, просто спросив: \enquote{Как вы думаете, зрелость GNOME повлияет на коммерческий успех операционной системы Linux?}

\enquote{Прошу вас прекратить называть операционную систему просто Линуксом, -- ответил Столлман, немедленно уперев свой взгляд в меня, -- ядро Linux это только малая часть операционной системы. Многие утилиты и приложения, из которых состоит операционная система, которую вы называете просто Линуксом, были разработаны не Торвальдсом, а добровольцами проекта GNU. Они потратили своё личное время, чтобы у людей была свободная операционная система. Невежливо и невежественно пренебрегать вкладом этих людей. Поэтому я прошу: когда вы говорите об операционной системе, называйте её GNU/Linux, пожалуйста}.

Записав эту тираду в свой репортёрский блокнот, я поднял глаза и обнаружил, что Столлман сверлит меня немигающим взглядом среди звенящей тишины. Неуверенно раздался вопрос другого журналиста -- в этом вопросе, конечно, прозвучало уже \enquote{GNU/Linux}, а не просто \enquote{Linux}. Отвечать стал Мигель де Иказа, лидер проекта GNOME, и только в середине его ответа Столлман, наконец, отвёл взгляд, и по спине у меня пробежала дрожь облегчения. Когда Столлман отчитывает кого-то ещё за ошибку в названии системы, радуешься, что он смотрит не на тебя.

Тирады Столлмана дают результат: многие журналисты перестают называть операционную систему просто Линуксом. Для Столлмана отчитывать людей за опускание GNU в названии системы  -- не более, чем практичный способ напомнить о ценности проекта GNU. В итоге Wired.com в своей статье сравнивает Ричарда с ленинским большевиком-революционером, которого потом стёрли из истории вместе с его делами. Так же и компьютерная индустрия, особенно в лице некоторых компаний, старается преуменьшить значение GNU и его философии. Потом последовали другие статьи, и хотя немногие журналисты на письме называют систему GNU/Linux, большинство из них всё-таки отдают должное Столлману за создание свободного программного обеспечения.

После этого я не встречал Столлмана почти 17 месяцев. За это время он ещё раз побывал в Кремниевой долине на августовском шоу LinuxWorld 1999 года, и без всяких официальных выступлений украсил мероприятие своим присутствием. Принимая от лица Фонда свободного софта премию имени Линуса Торвальдса за работу на благо общества, Столлман остроумно заметил: \enquote{Награждать Фонд свободного ПО премией имени Линуса Торвальдса -- это как награждать Повстанческий альянс премией имени Хана Соло}.

Но в этот раз слова Ричарда не наделали шума в СМИ. В середине недели компания Red Hat, крупный производитель программного обеспечения, связанного с GNU/Linux, становилась публичной через размещение акций на бирже. Эта новость подтверждала то, о чём раньше лишь подозревали: \enquote{Linux} становился модным словечком на Уолл-стрит, какими до этого были \enquote{электронная коммерция} и \enquote{дотком}. Фондовый рынок близился к максимуму, и потому все политические темы вокруг свободного софта и открытого кода отошли на второй план.

Может быть, поэтому на третьем LinuxWorld в 2000 году Столлмана уже не было. И вскоре после этого я во второй раз встретился с Ричардом и его коронным пронзительным взглядом. Я услышал, что он собирается в Кремниевую долину, и пригласил его на интервью в Пало-Альто. Выбор места придавал интервью нотку иронии -- за исключением Редмонда, немногие города США могут более красноречиво, чем Пало-Альто, подтвердить экономическую ценность собственнического ПО. Любопытно было посмотреть, как Столлман с его непримиримой войной против эгоизма и алчности будет держать себя в городе, где жалкий гараж стоит не меньше 500 тысяч долларов.

Следуя указаниям Столлмана, я добираюсь до штаб-квартиры Art.net, некоммерческого \enquote{виртуального объединения художников}. Эта штаб-квартира -- чуть подлатанная хибара за живой изгородью на северном краю города. Вот так неожиданно картина \enquote{Столлман в сердце Кремниевой долины} теряет весь сюрреализм.

Столлмана я нахожу в тёмной комнате, он сидит за ноутбуком и настукивает по клавишам. Как только я вхожу, он встречает меня своими 200-ваттными зелёными лазерами, но при этом вполне умиротворённо приветствует меня, и я приветствую его в ответ. Ричард опускает глаза обратно на экран ноутбука.

\enquote{Закончил сейчас писать статью о хакерском духе, -- говорит он, стуча клавишами, и предлагает, -- взгляни}.

Я смотрю. В комнате сумрачно, экран залит чёрным цветом, и на этом фоне выстроились ряды светло-зелёных букв. Очень непривычно для глаз, пришлось ждать, пока они смогут читать текст. Статья рассказывает о том, как Столлман пошёл однажды в корейский ресторан, и официант положил перед ним целых шесть палочек вместо двух. Ричард стал изучать других посетителей и понял, что все они пользуются только двумя палочками, не обращая внимания на остальные. Но склад ума Столлмана не позволил ему просто взять и начать есть, он увидел в этом задачу: найти интересную комбинацию использования всех доступных палочек с заказанными блюдами. Это суть хакерского мышления, и как всякий программный хак, решение с палочками может оказаться эффективным и в то же время диким.

Читая статью, я чувствую, что Столлман смотрит на меня. Поворачиваю голову и вижу -- на его лице творческой гордостью светится почти детская улыбка. На мои похвалы Ричард едва бровью повёл.

\enquote{Ещё минуту, и буду готов}, -- говорит он.

И снова набирает что-то на ноутбуке. Это серый, угловатый ноутбук, точно не из разряда тех новеньких прилизанных ноутов, что были у каждого программиста на последнем LinuxWorld. Поверх клавиш лежит подключаемая клавиатура меньших размеров, более удобная для усталых рук Ричарда. Ещё в середине 90-х годов боли в руках стали такими сильными, что Столлману пришлось нанять машинистку. Сейчас он просто использует специальные клавиатуры, на которых очень легко нажимать кнопки.

У Столлмана есть способность полностью уходить в себя во время работы. Наблюдая за тем, как его глаза неотрывно смотрят в экран, а пальцы бегают по клавишам, можно поймать себя на ощущении, что это увлечённо общаются близкие друзья.

Ричард заканчивает работу несколькими громкими щелчками клавиш и начинает разбирать ноутбук для переноски.

\enquote{Перекусим?} -- спрашивает он.

Мы идём к машине. Столлман прихрамывает, оберегая больную лодыжку, и жалуется, что 3 года назад потянул сухожилие, причём так неудачно, что пришлось даже бросить свои любимые народные танцы. \enquote{Я люблю танцевать, и теперь мне тяжело обходиться без этого. Для меня это настоящая трагедия}, -- вздыхает Ричард.

Тело Столлмана молча подтверждает его слова. Без физической активности оно раздалось, оплыло лишним весом на щеках и животе, потеряв ту стройность, что была ещё год назад. Этот груз тяжёл для Ричарда, он даже изгибает спину при ходьбе, как беременная женщина.

Наше \enquote{путешествие} к машине замедляется не только из-за этого -- Столлман ещё и нюхает каждый цветок, который встретит. Заметив особенно красивый цветок, он наклоняется и глубоко вдыхает, потом отступает назад с довольным выдохом.

\enquote{М-м-м, ринофитофилия}, -- произносит он, держась за спину. \endnote{Сначала я думал, что Столлман озвучил название цветка на латыни, и только через несколько месяцев узнал, что \textit{ринофитофилией} Ричард в шутку назвал своё наслаждение от сования носа в цветы, проведя параллель с сексуальными отклонениями. Ещё один забавный случай с цветами можно найти здесь: \url{http://www.stallman.org/articles/texas.html}}.

До ресторана мы едем не больше 3 минут. По совету Тима Нея, бывшего исполнительного директора Фонда свободного ПО, я оставляю выбор ресторана за Ричардом. Если во многих аспектах жизни Столлман почти аскет, то в еде -- настоящий гедонист. Ему нравится ездить по всему миру и продвигать свободное ПО ещё и потому, что так можно перепробовать уйму вкусных блюд со всего мира. \enquote{Прилетите в почти любой крупнейший город мира, и Ричард, скорее всего, будет знать, куда стоит пойти поесть, -- говорил мне Ней, -- будет знать что у них в меню, и наверняка закажет еды на весь стол}.

Сейчас Ричард выбрал кантонский ресторан димсам в 2 кварталах от Университетской авеню, главной улицы Пало-Альто. На этот вариант Столлмана сподвигла недавняя поездка в Китай и Гонконг, а также нелюбовь к острой хунаньской и сычуаньской кухне. \enquote{Я не любитель острых приправ}, -- признаётся Столлман.

Мы приезжаем в 12 часу дня, и нас просят подождать снаружи 20 минут. Зная, как хакеры не любят терять время, я готовлюсь к тому, что Столлман рассердится или расстроится, но он совершенно спокоен.

\enquote{Плохо, что нас только двое, -- говорит он, -- есть в большой компании всегда веселее}.

Пока мы ждём, Ричард пробует сделать несколько танцевальных па. Его движения осторожные, но точные и уверенные. Потом мы обсуждаем новости. Столлман сожалеет, что не поехал на последний LinuxWorld, потому что неплохо было бы поучаствовать в конференции, на которой объявили о создании фонда проекта GNOME. Поддержка этого фонда компаниями Sun Microsystems и IBM доказывает правоту Столлмана, когда он говорил, что свободное ПО и свободный рынок не исключают друг друга. Впрочем, он находит в этой новости и поводы для недовольства.

\enquote{Компании только и говорили, что о Линуксе, без всякого упоминания проекта GNU}, -- сетует Ричард.

А вот своими недавними поездками по развивающимся странам Столлман полностью доволен. Достаточно посмотреть на маршрут этих передвижений, чтобы понять, насколько популярным стал свободный софт в 2000 году. Из последних 115 дней Ричард лишь 12 дней пробыл в США, остальное время он колесил по Индии, Китаю и Бразилии. Эти разъезды дали отличную возможность оценить, как философия свободного ПО ложится на языки и культуры разных стран.

\enquote{В Индии многих людей заинтересовывает свободный софт, они сразу видят в нём способ построить собственную информационную инфраструктуру без огромных капиталовложений, -- рассказывает Столлман. -- В Китае заинтересовать людей труднее. Трудно было сравнивать свободное ПО со свободой слова в стране, где нет свободы слова. Но те, кто там интересуются свободным софтом -- интересуются очень сильно}.

Разговор переходит на компанию Napster, производителя ПО из Сан-Матео, которая в последнее время стала настоящей медийной звездой. Она продаёт спорный программный продукт, который образует децентрализованную одноранговую сеть любителей музыки. Любой пользователь этой сети может посмотреть MP3-файлы любого другого пользователя и скопировать их себе на компьютер без всякой платы, что ужасно огорчает и злит звукозаписывающие компании.

Программа Napster закрыта и несвободна, но использует философскую концепцию Столлмана, которая гласит: когда продукт переходит в цифровую форму и его копирование больше не требует материала или труда, то больше невозможно сдерживать естественную человеческую потребность делиться с ближним. И компания Napster использовала эту потребность, чтобы завоевать популярность среди пользователей, после чего монетизировала сопутствующие сервисы своей сети.

Неожиданный успех Napster озадачил и испугал традиционные звукозаписывающие компании, и им действительно было чего бояться. За несколько дней до нашей со Столлманом встречи в Пало-Альто судья окружного суда Мэрилин Патель удовлетворила ходатайство Американской звукозаписывающей ассоциации (RIAA), запретив Napster. Спустя некоторое время этот запрет был приостановлен Девятым окружным апелляционным судом США, но в начале 2001 года этот же суд признал компанию Napster нарушителем авторских прав. \endnote{Источник: Cecily Barnes and Scott Ard, \enquote{Court Grants Stay of Napster Injunction,} \textit{News.com} (July 28, 2000)}. Это решение представитель RIAA Хиллари Розен объявила \enquote{чистой победой творческого сообщества и легального онлайн-рынка}.\endnote{\enquote{A Clear Victory for Recording Industry in Napster Case,} RIAA (February 12, 2001)}.

У хакеров вроде Столлмана бизнес-модель Napster вызывает множество вопросов и претензий. Компания пылко продвигает старый хакерский принцип общественного владения информацией и безграничного распространения файлов, но делает это с использованием несвободной программы, что вызывает смешанные чувства. Столлман, будучи известным блюстителем точности и продуманности высказываний, очень сдержанно отзывается о Napster. Например, он говорит о парочке интересных социальных явлений, связанных с этой программой.

\enquote{До появления Napster я считал, что людям хватает частного, индивидуального обмена развлекательными продуктами, -- говорит Столлман, -- но огромная популярность Napster показывает, что очень важна возможность делиться такими продуктами не только с отдельными близкими, но и с очень широким кругом неопределённых лиц. И эту возможность нельзя отнимать у людей}.

Ричард ещё говорит, когда двери ресторана распахиваются, и хозяин приглашает нас войти. Через минуту мы сидим в углу зала рядом с зеркалом во всю стену.

Столлман быстро просматривает меню и делает выбор ещё до того, как нам приносят воду. \enquote{Обжаренный во фритюре рулет из креветок, покрытый соевым творогом, -- зачитывает он, -- оболочка из соевого творога -- это может быть интересным. Думаю, нам стоит это попробовать}.

Так начинается обсуждение китайской кухни и недавней поездки Ричарда в Китай. \enquote{Китайские блюда очень изысканные, -- рассказывает Столлман, и в его голосе впервые за всё утро звучит воодушевление, -- там используется очень много всего, чего не найдёшь в США -- всякие местечковые грибы и овощи. Я даже начал записывать, что китайцы едят, до того это интересно}.

Потом мы переходим к корейской кухне. В июне 2000 года Столлман наведался в Южную Корею. Его прибытие наделало шуму в местных СМИ, потому что в то же время там проходил визит Билла Гейтса, основателя и главы Microsoft. Ричард говорит, что местная еда по принесённому удовольствию стоит на втором месте в той поездке. Первое место осталось, конечно, за огромной фотографией его самого в главной сеульской газете. Фотографию эту поместили на первую полосу перед фотографией Билла Гейтса.

\enquote{Я пообедал чашкой холодной лапши наэн-мён, -- вспоминает Ричард, -- очень интересные ощущения. В большинстве мест не найти лапши такого сорта, как там, и такой изысканности -- тоже}.

Слово \enquote{изысканный} из уст Столлмана это очень высокая похвала. Я знаю это, потому что через несколько минут после восторгов Ричарда в адрес наэн-мён я чувствую, как его лазерные лучи опаливают мне верхушку правого плеча.

\enquote{За тобой сидит очень изысканная женщина}, -- говорит он.

Я оглядываюсь мельком и вижу спину молодой девушки лет двадцати, одетой в белое с пайетками платье. Она и её спутник оплачивают заказ. Я не глядя могу точно сказать, когда они ушли -- глаза Ричарда в этот момент затуманиваются.

\enquote{Как жаль! -- вздыхает он, -- они ушли. Думаю, я вряд ли увижу её снова}.

Ричард ненадолго задумывается. Это очень хороший повод затронуть тему отношений Столлмана с противоположным полом. Сведения о них весьма противоречивы. Некоторые хакеры рассказывают, что Ричард приветствует женщин поцелуем руки. \endnote{Mae Ling Mak, \enquote{A Mae Ling Story} (December 17, 1998)}. Пока что Мэй -- единственная женщина, которая может это подтвердить, хотя я слышал об этой манере Столлмана и от других женщин. Мэй даже танцевала с ним на LinuxWorld в 1999 году, преодолев первоначальную неприязнь. 26 мая 2000 года в \textit{Salon.com} вышла статья, которая рисует Столлмана кем-то вроде ловеласа от мира хакеров. Пытаясь вывести связь между свободным софтом и свободной любовью, журналист Аннали Ньюиц излагает слова Ричарда: \enquote{Я верю в любовь, но не в моногамию}.\endnote{Annalee Newitz, \enquote{If Code is Free Why Not Me?}, \textit{Salon.com} (May 26, 2000)}

Столлман даже отвлекается от чтения меню, когда я поднимаю тему женщин. \enquote{Знаешь, многие мужчины, похоже, очень нуждаются в сексе, и в то же время относятся к женщинам с презрением, -- замечает он, -- даже к тем женщинам, с которыми встречаются или живут. Я никогда этого не понимал}.

Я вспоминаю, как в книге \enquote{Открытые исходники} 1999 года (\textit{Open Sources}) Столлман рассказывает, как хотел назвать ядро GNU именем подруги. Её звали Аликс, что прекрасно подходило под Unix-стиль именования операционных систем, с его обычаем выбирать слова с буквой \enquote{икс} на конце. Аликс была системным администратором и в шутку говорила друзьям: \enquote{Кто-то должен назвать ядро в честь меня}. Поэтому Столлман дал ядру название Alix, чтобы приятно удивить её, но главный разработчик ядра изменил название на Hurd. Впрочем, название Alix осталось у одной из подсистем ядра, и когда кто-то увидел это в дереве исходников и показал Аликс, она была растрогана. В дальнейшем мало-помалу все упоминания Alix потерялись при редактировании кода.\endnote{Richard Stallman, \enquote{The GNU Operating System and the Free Software Movement,} \textit{Open Sources} (O'Reilly \& Associates, Inc., 1999): 65.

[РМС: Вильямс преподносит эту историю так, будто я -- большой романтик, и будто я пытался произвести впечатление на случайную женщину. Ни один хакер из МТИ не поверил бы в эти домыслы, потому что все мы довольно рано поняли, что ни одна женщина нами не заинтересуется, не говоря уж о том, чтобы полюбить, учитывая нашу одержимость программированием. Эта история произошла только потому, что у меня в то время была девушка. Если я и был тогда романтиком, то не безнадёжным и не опытным, скорее -- временно успешным.

На основании такой наивной интерпретации истории Вильямс сравнил меня с Дон-Кихотом.

Чтобы закрыть вопрос, я неточно процитирую первое издание: \enquote{На самом деле, я не пытался быть романтичным. Скорее, я хотел подразнить её. Ну то есть, это было немного романтично, но больше провокационно, понимаете? Получился классный прикол}.]}

Ричард улыбается. Я поднимаю руку и целую её, имитируя его предполагаемую манеру приветствия женщин. \enquote{Да, я так делаю, -- говорит Столлман, -- я обнаружил, что этот способ выразить симпатию нравится многим женщинам. Для меня это неплохая возможность выразить тёплые чувства и получить благодарность в ответ}.

Любовь нитью проходит через всю жизнь Ричарда, и он до боли откровенен, когда отвечает на такие вопросы. \enquote{В моей жизни любовь не выходила за границы фантазии}, -- говорит он. Разговор становится неловким. Несколько раз ответив односложными репликами, Столлман снова погружается в изучение меню, отсекая таким образом дальнейшие расспросы.

\enquote{Не хочешь попробовать шумаи?} -- спрашивает он.

Когда блюда в меню заканчиваются, мы уже разговариваем о том и о сём понемногу. Обсуждаем привязанность хакеров к китайской еде, их еженедельные набеги на Чайнатаун Бостона в бытность Столлмана программистом Лаборатории ИИ, обсуждаем логику китайского языка и его письменности. На каждую тему, что я предлагаю, Ричард откликается, как хорошо информированный человек.

\enquote{Я слышал, как люди разговаривают на шанхайском диалекте, когда был в Китае, -- рассказывает он, -- это интересно было слышать. Звучит не так, как мандаринский. Я попросил их сказать несколько пар слов на шанхайском и мандаринском, чтобы сравнить произношение. Оказалось, что разница -- в тональностях. Это интересная штука, потому что есть теория, что тональности это рудименты дополнительных слогов, которые потерялись со временем. Если это правда, то получается, что китайские диалекты разошлись ещё до потери этих дополнительных слогов}.

Принесли первое блюдо -- тарелку с зажаристым пирогом из турнепса. Мы нарезаем его на большие прямоугольные пирожные, которые пахнут отварной капустой, но вкусом напоминают картофельные оладьи на жиру.

Мне захотелось обсудить тему подросткового отчуждения Столлмана, спросить о возможном влиянии того периода на его нынешнюю довольно непопулярную непримиримость в некоторых вопросах, вроде названия операционной системы \enquote{GNU/Linux}.

\enquote{Думаю, годы, которые я когда-то провёл на положении изгоя, помогают мне не прогибаться под всякие популярные взгляды и мнения, -- отвечает Столлман, -- я никогда не понимал, как люди могут идти на поводу моды, господствующих мнений, взглядов окружающих. Не понимал, как это всё давит на людей. Думаю, причина в том, что я был безнадёжно отвергнут обществом, и потому у меня не было нужды как-то подстраиваться под окружающих. Я от этого априори ничего бы не выиграл, так что и пытаться не стоило}.

Ричард приводит в пример свои музыкальные вкусы, которые шли вразрез с любой модой. Подростком он слушал классическую музыку, тогда как его сверстники предпочитали эйсид-рок и популярных афроамериканских исполнителей. Он вспоминает об одном забавном случае тех лет. После того, как Битлз появились в 1964 году на шоу Эда Салливана, одноклассники Ричарда бросились скупать последние альбомы и синглы Битлов. Именно тогда Столлман объявил Великолепной четвёрке бойкот.

\enquote{Мне нравилась кое-какая популярная музыка до Битлов, -- говорит он, -- но вот Битлз мне совсем не нравились. Особенно мне не нравился ажиотаж вокруг них. Не хватало только организовать собрание, где каждый старался бы доказать, что он обожает Битлов сильнее, чем другие}.

Бойкот, впрочем, не получился, и Столлман начал раздумывать о других способах высмеять стадное мышление своих сверстников. Какое-то время он даже хотел собрать собственную рок-группу, которая пародировала бы Ливерпульскую четвёрку.

\enquote{Хотел назвать её так: Tokyo Rose and the Japanese Beetles}.

Помня о его любви к народной музыке разных стран мира, я спрашиваю, нравится ли ему Боб Дилан и другие фолк-музыканты ранних 60-х. Столлман кивает и говорит: \enquote{Peter, Paul and Mary -- да. Их музыка похожа на классный филк}.

Я не понимаю, что такое филк, и Ричард объясняет: так называется фольклорная музыка от научно-фантастического фандома. Классические примеры: \enquote{На вершине спагетти} -- переделанная песня \enquote{На вершине Старого Смоки}, и \enquote{Йода} -- переделка  \enquote{Лолы} в стилистике \enquote{Звёздных войн}.

Ричард спрашивает, хочу ли я услышать филк. Я, конечно, соглашаюсь, и он тут же начинает петь неожиданно чистым голосом на мотив \enquote{Blowin' in the Wind}:

\begin{verse}
Сколько дров наколол бы сурок,\\
Если б он умел колоть дрова?\\
Сколько полюсов запер бы поляк,\\
Если б он запирал полюса?\\
Сколько коленей вырастил бы негр,\\
Если б у него росли эти колени?\\
Ответ, дорогуша --\\
Палкой себе в уши\\
Возьми да засунь себе в уши.\ldots
\end{verse}

Песня замолкает, Столлман снова улыбается своей детской улыбкой. Я оглядываюсь. Выходцы из Азии целыми семьями уплетают воскресный обед, и никто не обращает внимания на бородатого вокалиста. Я расслабляюсь и улыбаюсь тоже. \endnote{Филк-творчество Столлмана можно найти здесь:  \url{http://www.stallman.org/doggerel.html}. Чтобы услышать, как он поёт \enquote{Песню о свободном софте}, зайдите сюда: \url{http://www.gnu.org/music/free-software-song.html}.}

\enquote{Возьмёшь последний кукурузный шарик?} -- спрашивает Ричард, сверкая глазами. Прежде, чем я что-то говорю или делаю, Столлман хватает его двумя палочками, гордо поднимает вверх и торжественно объявляет: \enquote{Он предназначался именно мне!}

Трапеза заканчивается, разговор снова становится похож на нормальное интервью. Столлман откидывается на спинку стула, держа в руках чашку чая. Мы возвращаемся к обсуждению Napster и его связи с движением свободного ПО. Я спрашиваю, можем ли мы расширить принципы свободного софта на другие области, вроде распространения музыки?

\enquote{Было бы неправильно механически переносить решение из одной области в другую, -- отвечает Ричард, -- правильно было бы проанализировать каждый тип деятельности в отдельности, и из этого уже делать выводы}.

Как объясняет Столлман, он делит все объекты авторского права на 3 категории. Первая категория состоит из \enquote{функциональных} продуктов -- компьютерных программ, словарей, учебников. Вторая категория включает в себя продукты, которые можно назвать \enquote{свидетельскими} -- например, научные работы и исторические документы. Существование этих продуктов лишается смысла, если другие могут их редактировать как захочется. Сюда же относятся и результаты личного самовыражения: дневники, личные истории, автобиографии. Редактировать такие документы -- значит искажать точку зрения автора, что Столлман считает неправильным. Наконец, третья категория -- это произведения искусства и развлекательная продукция.

Разумно было бы, считает Столлман, предоставить людям право модифицировать продукты первой категории без всяких ограничений. Но что делать с продуктами второй и третьей категорий -- должны решать только их создатели. При этом свобода людей неограниченно копировать и бесплатно раздавать распространяется на продукты всех трёх категорий. Когда пользователи интернета могут свободно тиражировать статьи, изображения, песни, книги -- это очень хорошо и полезно для человечества. \enquote{Очевидно, что спонтанная раздача информационных продуктов в частном порядке должна быть разрешена, если мы не хотим получить полицейское государство, ибо только оно и может остановить такую раздачу, -- говорит Столлман, -- ставить государство в лице карательных органов между друзьями и родственниками -- совершенно аморально. Napster также натолкнул меня на мысль, что нужно разрешить и некоммерческую раздачу продуктов неопределённому кругу лиц. Ведь это интересно огромному числу людей, они находят это полезным и приятным}.

Я спрашиваю, пересмотрят ли суды, по его мнению, свою правовую практику, но Ричард останавливает меня, не дав договорить.

\enquote{Вопрос некорректен, -- говорит он, -- ты примеряешь этику на правовую сферу, но законы никогда в полной мере не соответствуют морали. Суды всегда будут придерживаться жёсткой линии, потому что отстаивают экономические интересы, в нашем случае -- интересы издателей}.

Эти слова выражают суть политической философии Столлмана: люди не обязаны играть по правилам, которые задают юристы под нажимом бизнеса. Свобода это этическое понятие, а не юридическое. \enquote{Я смотрю дальше всяких законов, нынешних и будущих, -- говорит Столлман, -- и даже не пытаюсь думать о каких-то законопроектах. Я обращаю внимание только на то, что закон делает в действительности. И если закон запрещает тебе делиться информационными продуктами с друзьями, то он ничем не лучше законов Джима Кроу. Уважения к такому закону нет}.

Упоминание Джима Кроу переводит разговор в другое русло. Политические фигуры 50-60-х годов -- влияли они на Ричарда, вдохновляли его на что-то, или нет? Те же движения за гражданские права, они широко обращались к вечным ценностям: свободе, справедливости, честности.

Столлман слушает меня вполуха, увлечённо ковыряясь в свалявшейся пряди волос. Когда я сравниваю Ричарда с Мартином Лютером Кингом, он засовывает колтун себе в рот.

\enquote{Я не в его лиге, но играю в ту же игру}, -- отвечает он, разжёвывая волосяной узел.

А что насчёт Малькольма Икс? Как и этот лидер \enquote{Нации ислама}, Столлман прославился разжиганием споров, отталкиванием потенциальных союзников, превозношением самодостаточности над растворением в обществе.

Ричарду это сравнение не нравится. \enquote{Моё послание всё-таки ближе к посланию Кинга. Это довольно общее послание. Оно категорично осуждает некоторые социальные практики, угнетающие других людей. И оно обращено не к узкой группе людей, а ко всем. Я предлагаю каждому стать свободным}.

На Столлмана обрушивается много критики за его нежелание создавать политические союзы, некоторые считают это проявлением психологических и личных особенностей. Его неприязнь к термину \enquote{открытый код} и отказ сотрудничать с этим движением имеет понятные причины -- всё-таки Столлман вложил уйму времени и сил в свободное ПО, и теперь это движение составляет его политический капитал. Но такие едкие комментарии, как сравнение Торвальдса с Ханом Соло на LinuxWorld, придают его репутации заметный оттенок замшелого упрямца, который не хочет идти в ногу с политическими и экономическими трендами.

\enquote{Я восхищаюсь Ричардом и глубоко уважаю его дело, -- сказал Роберт Янг, президент компании Red Hat, -- но трудно смириться с тем, что Ричард зачастую с друзьями обращается хуже, чем с врагами}.

[РМС: Людей вроде Янга и компании вроде Red Hat можно назвать друзьями лишь отчасти. Да, они широко участвуют в разработке свободного ПО, в том числе и программ GNU. Но также они делают много других вещей, которые идут вразрез с целями философии свободного софта. Например, поставляют несвободное ПО в составе своих систем GNU/Linux. Хватает недружественных не только действий, но и слов: употребление названия Linux вместо GNU/Linux, продвижение термина \enquote{открытый код} вместо \enquote{свободное ПО}. Так что я могу работать с Янгом и Red Hat по некоторым направлениям, но этого недостаточно, чтобы считать их союзниками].

Столлман не связывает движение за свободное ПО с другими политическими движениями, но это не значит, что ему неинтересны цели этих движений. Если вы наведаетесь в его кабинет в МТИ, вы увидите целый информационный центр, который получает статьи и сводки от мировых новостных ресурсов левой направленности. Зайдите на его личный сайт \url{stallman.org}, и увидите там его нападки на законы DMCA, на борьбу с наркотиками, на ВТО. Ричард объясняет это так: \enquote{Мы должны быть очень осторожны в выборе союзников, потому что эти союзники могут не понравиться каким-то участникам нашего движения. Поэтому мы априори избегаем любых политических партий. Не хочется, чтобы внутри движения за свободное ПО были расколы}.

Тогда я спрашиваю: \enquote{Почему вы сфокусировались именно на свободном ПО, а не на каких-то других идеях, которые вас волнуют?}

\enquote{Я не хочу преувеличивать важность этого маленького ручейка свободы, -- отвечает он, -- потому что, конечно же, есть куда более обширные, важные и актуальные проблемы лучшего мироустройства. Нельзя сказать, что свободный софт может сравниться с ними по важности. Дело в том, что проблема свободы программ свалилась именно на меня, и именно я знаю, как эту проблему можно решить. Что же касается таких глобальных проблем, как жестокость полиции, борьба с наркотиками, расизм, нарушения прав человека, религиозное мракобесие -- я не представляю, что с этим всем можно поделать, и как за это взяться}.

Вот ещё одно доказательство зависимости его политической активности от личной уверенности в защищаемой области. Чтобы создать, доработать и отточить философию свободного ПО, Столлману потребовались многие годы, мощное техническое образование и огромный опыт программиста. Он уверен, что не сможет предложить такой же качественной проработки идей в других важных областях.

\enquote{Хотелось бы добиться серьёзных улучшений в этих глобальных вопросах, и я бы безмерно гордился, если бы смог это сделать. Но эти вопросы настолько объёмны и сложны, что над ними не очень-то успешно работают миллионы людей по всему миру, которые в этих областях куда опытнее и образованнее меня. Но это и к лучшему, ведь пока все эти люди боролись с глобальными проблемами, я увидел ещё одну проблему, которую никто не замечал. И начал с ней бороться. Может быть, это не такая уж и большая проблема, но всё-таки проблема, и ею никто не занимался}.

Наконец, разжевав и распутав колтун, Столлман предлагает мне оплатить чек. Прежде, чем официант уносит деньги, Ричард бросает в кучку купюр ещё одну. Её даже брать в руки не нужно, чтобы понять -- она вышла совсем не из ворот Монетного двора США, настолько она очевидно фальшивая. Вместо Вашингтона или Линкольна на лицевой стороне нарисован мультяшный поросёнок. Вместо слов \enquote{United States of America} написано \enquote{Untied Status of Avarice} (\enquote{Распущенное Состояние Алчности}). Достоинство купюры -- 0 долларов. \endnote{РМС: Вильямс несправедливо назвал эту банкноту фальшивой. Это законное платёжное средство стоимостью 0 долларов. Любое правительственное учреждение США примет его и выдаст вам золота на 0 долларов}.

Столлман тянет официанта за рукав и говорит: \enquote{Я добавил ещё ноль баксов в оплату счёта}. И по-детски улыбается. Официант непонимающе улыбается в ответ и убегает.

\enquote{Думаю, теперь мы свободны}, -- говорит Столлман.

\theendnotes
\setcounter{endnote}{0}
