%% Copyright (c) 2002, 2010 Sam Williams
%% Copyright (c) 2010 Richard M. Stallman
%% Permission is granted to copy, distribute and/or modify this
%% document under the terms of the GNU Free Documentation License,
%% Version 1.3 or any later version published by the Free Software
%% Foundation; with no Invariant Sections, no Front-Cover Texts, and
%% no Back-Cover Texts. A copy of the license is included in the
%% file called ``gfdl.tex''.


\chapter{Puddle of Freedom}

[RMS: In this chapter, I have corrected statements about facts, including facts about my thoughts and feelings, and removed some gratuitous hostility in descriptions of events.  I have preserved Williams' statements of his own impressions, except where noted.]

Ask anyone who's spent more than a minute in Richard Stallman's presence, and you'll get the same recollection: forget the long hair. Forget the quirky demeanor. The first thing you notice is the gaze. One look into Stallman's green eyes, and you know you're in the presence of a true believer.

To call the Stallman gaze intense is an understatement. Stallman's eyes don't just look at you; they look through you. Even when your own eyes momentarily shift away out of simple primate politeness, Stallman's eyes remain locked-in, sizzling away at the side of your head like twin photon beams.

Maybe that's why most writers, when describing Stallman, tend to go for the religious angle. In a 1998 \textit{Salon.com} article titled ``The Saint of Free Software,'' Andrew Leonard describes Stallman's green eyes as ``radiating the power of an Old Testament prophet.''\endnote{See Andrew Leonard, ``The Saint of Free Software,'' \textit{Salon.com} (August 1998), \url{http://www.salon.com/21st/feature/1998/08/cov_31feature.html}.} A 1999 \textit{Wired} magazine article describes the Stallman beard as ``Rasputin-like,''\endnote{See Leander Kahney, ``Linux's Forgotten Man,'' \textit{Wired News} (March 5, 1999), \url{http://www.wired.com/news/print/0,1294,18291,00.html}.} while a \textit{London Guardian} profile describes the Stallman smile as the smile of ``a disciple seeing Jesus.''\endnote{See ``Programmer on moral high ground; Free software is a moral issue for Richard Stallman believes in freedom and free software,'' \textit{London Guardian} (November 6, 1999), \url{http://www.guardian.co.uk/uk/1999/nov/06/andrewbrown}.

These are just a small sampling of the religious comparisons. To date, the most extreme comparison has to go to Linus Torvalds, who, in his autobiography -- see Linus Torvalds and David Diamond, \textit{Just For Fun: The Story of an Accidental Revolutionary} (HarperCollins Publishers, Inc., 2001): 58 -- writes, ``Richard Stallman is the God of Free Software.''

Honorable mention goes to Larry Lessig, who, in a footnote description of Stallman in his book -- see Larry Lessig, \textit{The Future of Ideas} (Random House, 2001): 270 -- likens Stallman to Moses:

\begin{quote}

\ldots as with Moses, it was another leader, Linus Torvalds, who finally carried the movement into the promised land by facilitating the development of the final part of the OS puzzle. Like Moses, too, Stallman is both respected and reviled by allies within the movement. He is [an] unforgiving, and hence for many inspiring, leader of a critically important aspect of modern culture. I have deep respect for the principle and commitment of this extraordinary individual, though I also have great respect for those who are courageous enough to question his thinking and then sustain his wrath.

\end{quote}

In a final interview with Stallman, I asked him his thoughts about the religious comparisons. ``Some people do compare me with an Old Testament prophet, and the reason is Old Testament prophets said certain social practices were wrong. They wouldn't compromise on moral issues. They couldn't be bought off, and they were usually treated with contempt.''}

Such analogies serve a purpose, but they ultimately fall short. That's because they fail to take into account the vulnerable side of the Stallman persona. Watch the Stallman gaze for an extended period of time, and you will begin to notice a subtle change. What appears at first to be an attempt to intimidate or hypnotize reveals itself upon second and third viewing as a frustrated attempt to build and maintain contact. If his personality has a touch or ``shadow'' of autism or Asperger's Syndrome, a possibility that Stallman has entertained from time to time, his eyes certainly confirm the diagnosis. Even at their most high-beam level of intensity, they have a tendency to grow cloudy and distant, like the eyes of a wounded animal preparing to give up the ghost.

My own first encounter with the legendary Stallman gaze dates back to the March, 1999, LinuxWorld Convention and Expo in San Jose, California. Billed as a ``coming out party'' for the ``Linux'' software community, the convention also stands out as the event that reintroduced Stallman to the technology media. Determined to push for his proper share of credit, Stallman used the event to instruct spectators and reporters alike on the history of the GNU Project and the project's overt political objectives.

As a reporter sent to cover the event, I received my own Stallman tutorial during a press conference announcing the release of GNOME 1.0, a free software graphic user interface. Unwittingly, I push an entire bank of hot buttons when I throw out my very first question to Stallman himself: ``Do you think GNOME's maturity will affect the commercial popularity of the Linux operating system?''

``I ask that you please stop calling the operating system Linux,'' Stallman responds, eyes immediately zeroing in on mine. ``The Linux kernel is just a small part of the operating system. Many of the software programs that make up the operating system you call Linux were not developed by Linus Torvalds at all. They were created by GNU Project volunteers, putting in their own personal time so that users might have a free operating system like the one we have today. To not acknowledge the contribution of those programmers is both impolite and a misrepresentation of history. That's why I ask that when you refer to the operating system, please call it by its proper name, GNU/Linux.''

Taking the words down in my reporter's notebook, I notice an eerie silence in the crowded room. When I finally look up, I find Stallman's unblinking eyes waiting for me. Timidly, a second reporter throws out a question, making sure to use the term ``GNU/Linux'' instead of Linux. Miguel de Icaza, leader of the GNOME project, fields the question. It isn't until halfway through de Icaza's answer, however, that Stallman's eyes finally unlock from mine. As soon as they do, a mild shiver rolls down my back. When Stallman starts lecturing another reporter over a perceived error in diction, I feel a guilty tinge of relief. At least he isn't looking at me, I tell myself.

For Stallman, such face-to-face moments would serve their purpose. By the end of the first LinuxWorld show, most reporters know better than to use the term ``Linux'' in his presence, and Wired.com is running a story comparing Stallman to a pre-Stalinist revolutionary erased from the history books by hackers and entrepreneurs eager to downplay the GNU Project's overly political objectives.\endnote{See Leander Kahney (1999).} Other articles follow, and while few reporters call the operating system GNU/Linux in print, most are quick to credit Stallman for launching the drive to build a free software operating system 15 years before.

I won't meet Stallman again for another 17 months. During the interim, Stallman will revisit Silicon Valley once more for the August, 1999 LinuxWorld show. Although not invited to speak, Stallman does manage to deliver the event's best line. Accepting the show's Linus Torvalds Award for Community Service -- an award named after Linux creator Linus Torvalds -- on behalf of the Free Software Foundation, Stallman wisecracks, ``Giving the Linus Torvalds Award to the Free Software Foundation is a bit like giving the Han Solo Award to the Rebel Alliance.''

This time around, however, the comments fail to make much of a media dent. Midway through the week, Red Hat, Inc., a prominent GNU/Linux vendor, goes public. The news merely confirms what many reporters such as myself already suspect: ``Linux'' has become a Wall Street buzzword, much like ``e-commerce'' and ``dot-com'' before it. With the stock market approaching the Y2K rollover like a hyperbola approaching its vertical asymptote, all talk of free software or open source as a political phenomenon falls by the wayside.

Maybe that's why, when LinuxWorld follows up its first two shows with a third LinuxWorld show in August, 2000, Stallman is conspicuously absent.

My second encounter with Stallman and his trademark gaze comes shortly after that third LinuxWorld show. Hearing that Stallman is going to be in Silicon Valley, I set up a lunch interview in Palo Alto, California. The meeting place seems ironic, not only because of his absence from the show but also because of the overall backdrop. Outside of Redmond, Washington, few cities offer a more direct testament to the economic value of proprietary software. Curious to see how Stallman, a man who has spent the better part of his life railing against our culture's predilection toward greed and selfishness, is coping in a city where even garage-sized bungalows run in the half-million-dollar price range, I make the drive down from Oakland.

I follow the directions Stallman has given me, until I reach the headquarters of Art.net, a nonprofit ``virtual artists collective.'' Located in a hedge-shrouded house in the northern corner of the city, the Art.net headquarters are refreshingly run-down. Suddenly, the idea of Stallman lurking in the heart of Silicon Valley doesn't seem so strange after all.

I find Stallman sitting in a darkened room, tapping away on his gray laptop computer. He looks up as soon as I enter the room, giving me a full blast of his 200-watt gaze. When he offers a soothing ``Hello,'' I offer a return greeting. Before the words come out, however, his eyes have already shifted back to the laptop screen.

``I'm just finishing an article on the spirit of hacking,'' Stallman says, fingers still tapping. ``Take a look.''

I take a look. The room is dimly lit, and the text appears as greenish-white letters on a black background, a reversal of the color scheme used by most desktop word-processing programs, so it takes my eyes a moment to adjust. When they do, I find myself reading Stallman's account of a recent meal at a Korean restaurant. Before the meal, Stallman makes an interesting discovery: the person setting the table has left six chopsticks instead of the usual two in front of Stallman's place setting. Where most restaurant goers would have ignored the redundant pairs, Stallman takes it as challenge: find a way to use all six chopsticks at once. Like many software hacks, the successful solution is both clever and silly at the same time. Hence Stallman's decision to use it as an illustration.

As I read the story, I feel Stallman watching me intently. I look over to notice a proud but child-like half smile on his face. When I praise the essay, my comment barely merits a raised eyebrow.

``I'll be ready to go in a moment,'' he says.

Stallman goes back to tapping away at his laptop. The laptop is gray and boxy, not like the sleek, modern laptops that seemed to be a programmer favorite at the recent LinuxWorld show. Above the keyboard rides a smaller, lighter keyboard, a testament to Stallman's aging hands. During the mid 1990s, the pain in Stallman's hands became so unbearable that he had to hire a typist. Today, Stallman relies on a keyboard whose keys require less pressure than a typical computer keyboard.

Stallman has a tendency to block out all external stimuli while working. Watching his eyes lock onto the screen and his fingers dance, one quickly gets the sense of two old friends locked in deep conversation.

The session ends with a few loud keystrokes and the slow disassembly of the laptop.

``Ready for lunch?'' Stallman asks.

We walk to my car. Pleading a sore ankle, Stallman limps along slowly. Stallman blames the injury on a tendon in his left foot. The injury is three years old and has gotten so bad that Stallman, a huge fan of folk dancing, has been forced to give up all dancing activities. ``I love folk dancing intensely,'' Stallman laments. ``Not being able to dance has been a tragedy for me.''

Stallman's body bears witness to the tragedy. Lack of exercise has left Stallman with swollen cheeks and a pot belly that was much less visible the year before. You can tell the weight gain has been dramatic, because when Stallman walks, he arches his back like a pregnant woman trying to accommodate an unfamiliar load.

The walk is further slowed by Stallman's willingness to stop and smell the roses, literally. Spotting a particularly beautiful blossom, he strokes the innermost petals against his nose, takes a deep sniff, and steps back with a contented sigh.

``Mmm, rhinophytophilia,'' he says, rubbing his back.\endnote{At the time, I thought Stallman was referring to the flower's scientific name. Months later, I would learn that \textit{rhinophytophilia} was in fact a humorous reference to the activity -- i.e., Stallman's sticking his nose into a flower and enjoying the moment -- presenting it as the kinky practice of nasal sex with plants. For another humorous Stallman flower incident, visit: \url{http://www.stallman.org/articles/texas.html}.}

The drive to the restaurant takes less than three minutes. Upon recommendation from Tim Ney, former executive director of the Free Software Foundation, I have let Stallman choose the restaurant. While some reporters zero in on Stallman's monk-like lifestyle, the truth is, Stallman is a committed epicure when it comes to food. One of the fringe benefits of being a traveling missionary for the free software cause is the ability to sample delicious food from around the world. ``Visit almost any major city in the world, and chances are Richard knows the best restaurant in town,'' says Ney. ``Richard also takes great pride in knowing what's on the menu and ordering for the entire the table.''  (If they are willing, that is.)

For today's meal, Stallman has chosen a Cantonese-style dim sum restaurant two blocks off University Avenue, Palo Alto's main drag. The choice is partially inspired by Stallman's recent visit to China, including a stop in Hong Kong, in addition to Stallman's personal aversion to spicier Hunanese and Szechuan cuisine. ``I'm not a big fan of spicy,'' Stallman admits.

We arrive a few minutes after 11 a.m. and find ourselves already subject to a 20-minute wait. Given the hacker aversion to lost time, I hold my breath momentarily, fearing an outburst. Stallman, contrary to expectations, takes the news in stride.

``It's too bad we couldn't have found somebody else to join us,'' he tells me. ``It's always more fun to eat with a group of people.''

During the wait, Stallman practices a few dance steps. His moves are tentative but skilled. We discuss current events. Stallman says his only regret about not attending LinuxWorld was missing out on a press conference announcing the launch of the GNOME Foundation. Backed by Sun Microsystems and IBM, the foundation is in many ways a vindication for Stallman, who has long championed that free software and free-market economics need not be mutually exclusive. Nevertheless, Stallman remains dissatisfied by the message that came out.

``The way it was presented, the companies were talking about Linux with no mention of the GNU Project at all,'' Stallman says.

Such disappointments merely contrast the warm response coming from overseas, especially Asia, Stallman notes. A quick glance at the Stallman 2000 travel itinerary bespeaks the growing popularity of the free software message. Between recent visits to India, China, and Brazil, Stallman has spent 12 of the last 115 days on United States soil. His travels have given him an opportunity to see how the free software concept translates into different languages of cultures.

``In India many people are interested in free software, because they see it as a way to build their computing infrastructure without spending a lot of money,'' Stallman says. ``In China, the concept has been much slower to catch on. Comparing free software to free speech is harder to do when you don't have any free speech. Still, the level of interest in free software during my last visit was profound.''

The conversation shifts to Napster, the San Mateo, California software company, which has become something of a media cause célèbre in recent months. The company markets a controversial software tool that lets music fans browse and copy the music files of other music fans. Thanks to the magnifying powers of the Internet, this so-called ``peer-to-peer'' program has evolved into a de facto online jukebox, giving ordinary music fans a way to listen to MP3 music files over the computer without paying a royalty or fee, much to record companies' chagrin.

Although based on proprietary software, the Napster system draws inspiration from the long-held Stallman contention that once a work enters the digital realm -- in other words, once making a copy is less a matter of duplicating sounds or duplicating atoms and more a matter of duplicating information -- the natural human impulse to share a work becomes harder to restrict. Rather than impose additional restrictions, Napster execs have decided to take advantage of the impulse. Giving music listeners a central place to trade music files, the company has gambled on its ability to steer the resulting user traffic toward other commercial opportunities.

The sudden success of the Napster model has put the fear in traditional record companies, with good reason. Just days before my Palo Alto meeting with Stallman, U.S. District Court Judge Marilyn Patel granted a request filed by the Recording Industry Association of America for an injunction against the file-sharing service. The injunction was subsequently suspended by the U.S. Ninth District Court of Appeals, but by early 2001, the Court of Appeals, too, would find the San Mateo-based company in breach of copyright law,\endnote{See Cecily Barnes and Scott Ard, ``Court Grants Stay of Napster Injunction,'' \textit{News.com} (July 28, 2000), \url{http://news.cnet.com/news/0-1005-200-2376465.html}.} a decision RIAA spokesperson Hillary Rosen would later proclaim a ``clear victory for the creative content community and the legitimate online marketplace.''\endnote{See ``A Clear Victory for Recording Industry in Napster Case,'' RIAA press release (February 12, 2001), \url{http://www.riaa.com/PR_story.cfm?id=372}.}

For hackers such as Stallman, the Napster business model is troublesome in different ways. The company's eagerness to appropriate time-worn hacker principles such as file sharing and communal information ownership, while at the same time selling a service based on proprietary software, sends a distressing mixed message. As a person who already has a hard enough time getting his own carefully articulated message into the media stream, Stallman is understandably reticent when it comes to speaking out about the company. Still, Stallman does admit to learning a thing or two from the social side of the Napster phenomenon.

``Before Napster, I thought it might be [sufficient] for people to privately redistribute works of entertainment,'' Stallman says. ``The number of people who find Napster useful, however, tells me that the right to redistribute copies not only on a neighbor-to-neighbor basis, but to the public at large, is essential and therefore may not be taken away.''

No sooner does Stallman say this than the door to the restaurant swings open and we are invited back inside by the host. Within a few seconds, we are seated in a side corner of the restaurant next to a large mirrored wall.

The restaurant's menu doubles as an order form, and Stallman is quickly checking off boxes before the host has even brought water to the table. ``Deep-fried shrimp roll wrapped in bean-curd skin,'' Stallman reads. ``Bean-curd skin. It offers such an interesting texture. I think we should get it.''

This comment leads to an impromptu discussion of Chinese food and Stallman's recent visit to China. ``The food in China is utterly exquisite,'' Stallman says, his voice gaining an edge of emotion for the first time this morning. ``So many different things that I've never seen in the U.S., local things made from local mushrooms and local vegetables. It got to the point where I started keeping a journal just to keep track of every wonderful meal.''

The conversation segues into a discussion of Korean cuisine. During the same June, 2000, Asian tour, Stallman paid a visit to South Korea. His arrival ignited a mini-firestorm in the local media thanks to a Korean software conference attended by Microsoft founder and chairman Bill Gates that same week. Next to getting his photo above Gates's photo on the front page of the top Seoul newspaper, Stallman says the best thing about the trip was the food. ``I had a bowl of naeng myun, which is cold noodles,'' says Stallman. ``These were a very interesting feeling noodle. Most places don't use quite the same kind of noodles for your naeng myun, so I can say with complete certainty that this was the most exquisite naeng myun I ever had.''

The term ``exquisite'' is high praise coming from Stallman. I know this, because a few moments after listening to Stallman rhapsodize about naeng myun, I feel his laser-beam eyes singeing the top of my right shoulder.

``There is the most exquisite woman sitting just behind you,'' Stallman says.

I turn to look, catching a glimpse of a woman's back. The woman is young, somewhere in her mid-20s, and is wearing a white sequined dress. She and her male lunch companion are in the final stages of paying the check. When both get up from the table to leave the restaurant, I can tell without looking, because Stallman's eyes suddenly dim in intensity.

``Oh, no,'' he says. ``They're gone. And to think, I'll probably never even get to see her again.''

After a brief sigh, Stallman recovers. The moment gives me a chance to discuss Stallman's reputation vis-à-vis the fairer sex. The reputation is a bit contradictory at times. A number of hackers report Stallman's predilection for greeting females with a kiss on the back of the hand.\endnote{See Mae Ling Mak, ``A Mae Ling Story'' (December 17, 1998), \url{http://crackmonkey.org/pipermai l/crackmonkey/1998-December/001777.html}.

So far, Mak is the only person I've found willing to speak on the record in regard to this practice, although I've heard this from a few other female sources. Mak, despite expressing initial revulsion at it, later managed to put aside her misgivings and dance with Stallman at a 1999 LinuxWorld show.} A May 26, 2000 \textit{Salon.com} article, meanwhile, portrays Stallman as a bit of a hacker lothario. Documenting the free software-free love connection, reporter Annalee Newitz presents Stallman as rejecting traditional family values, telling her, ``I believe in love, but not monogamy.''\endnote{See Annalee Newitz, ``If Code is Free Why Not Me?'', \textit{Salon.com} (May 26, 2000), \url{http://www.salon.com/tech/feature/2000/05/26/free_love/print.html}.}

Stallman lets his menu drop a little when I bring this up. ``Well, most men seem to want sex and seem to have a rather contemptuous attitude towards women,'' he says. ``Even women they're involved with. I can't understand it at all.''

I mention a passage from the 1999 book \textit{Open Sources} in which Stallman confesses to wanting to name the GNU kernel after a girlfriend at the time. The girlfriend's name was Alix, a name that fit perfectly with the Unix developer convention of putting an ``x'' at the end names of operating systems and kernels -- e.g., ``Linux.'' Alix was a Unix system administrator, and had suggested to her friends, ``Someone should name a kernel after me.'' So Stallman decided to name the GNU kernel ``Alix'' as a surprise for her. The kernel's main developer renamed the kernel ``Hurd,'' but retained the name ``Alix'' for part of it.  One of Alix's friends noticed this part in a source snapshot and told her, and she was touched.  A later redesign of the Hurd eliminated that part.\endnote{See Richard Stallman, ``The GNU Operating System and the Free Software Movement,'' \textit{Open Sources} (O'Reilly \& Associates, Inc., 1999): 65. [RMS: Williams interpreted this vignette as suggesting that I am a hopeless romantic, and that my efforts were meant to impress some as-yet-unidentified woman.  No MIT hacker would believe this, since we learned quite young that most women wouldn't notice us, let alone love us, for our programming.  We programmed because it was fascinating.  Meanwhile, these events were only possible because I had a thoroughly identified girlfriend at the time.  If I was a romantic, at the time I was neither a hopeless romantic nor a hopeful romantic, but rather temporarily a successful one.

On the strength of that naive interpretation, Williams went on to compare me to Don Quijote.

For completeness' sake, here's a somewhat inarticulate quote from the first edition: ``I wasn't really trying to be romantic. It was more of a teasing thing. I mean, it was romantic, but it was also teasing, you know? It would have been a delightful surprise.'']}

For the first time all morning, Stallman smiles. I bring up the hand kissing. ``Yes, I do do that,'' Stallman says. ``I've found it's a way of offering some affection that a lot of women will enjoy. It's a chance to give some affection and to be appreciated for it.''

Affection is a thread that runs clear through Richard Stallman's life, and he is painfully candid about it when questions arise. ``There really hasn't been much affection in my life, except in my mind,'' he says. Still, the discussion quickly grows awkward. After a few one-word replies, Stallman finally lifts up his menu, cutting off the inquiry.

``Would you like some shu mai?'' he asks.

When the food comes out, the conversation slaloms between the arriving courses. We discuss the oft-noted hacker affection for Chinese food, the weekly dinner runs into Boston's Chinatown district during Stallman's days as a staff programmer at the AI Lab, and the underlying logic of the Chinese language and its associated writing system. Each thrust on my part elicits a well-informed parry on Stallman's part.

``I heard some people speaking Shanghainese the last time I was in China,'' Stallman says. ``It was interesting to hear. It sounded quite different [from Mandarin]. I had them tell me some cognate words in Mandarin and Shanghainese. In some cases you can see the resemblance, but one question I was wondering about was whether tones would be similar. They're not. That's interesting to me, because there's a theory that the tones evolved from additional syllables that got lost and replaced. Their effect survives in the tone. If that's true, and I've seen claims that that happened within historic times, the dialects must have diverged before the loss of these final syllables.''

The first dish, a plate of pan-fried turnip cakes, has arrived. Both Stallman and I take a moment to carve up the large rectangular cakes, which smell like boiled cabbage but taste like potato latkes fried in bacon.

I decide to bring up the outcast issue again, wondering if Stallman's teenage years conditioned him to take unpopular stands, most notably his uphill battle since 1994 to get computer users and the media to replace the popular term ``Linux'' with ``GNU/Linux.''

``I believe [being an outcast] did help me [to avoid bowing to popular views],'' Stallman says, chewing on a dumpling. ``I have never understood what peer pressure does to other people. I think the reason is that I was so hopelessly rejected that for me, there wasn't anything to gain by trying to follow any of the fads. It wouldn't have made any difference. I'd still be just as rejected, so I didn't try.''

Stallman points to his taste in music as a key example of his contrarian tendencies. As a teenager, when most of his high school classmates were listening to Motown and acid rock, Stallman preferred classical music. The memory leads to a rare humorous episode from Stallman's middle-school years. Following the Beatles' 1964 appearance on the Ed Sullivan Show, most of Stallman's classmates rushed out to purchase the latest Beatles albums and singles. Right then and there, Stallman says, he made a decision to boycott the Fab Four.

``I liked some of the pre-Beatles popular music,'' Stallman says. ``But I didn't like the Beatles. I especially disliked the wild way people reacted to them. It was like: who was going to have a Beatles assembly to adulate the Beatles the most?''

When his Beatles boycott failed to take hold, Stallman looked for other ways to point out the herd-mentality of his peers. Stallman says he briefly considered putting together a rock band himself dedicated to satirizing the Liverpool group.

``I wanted to call it Tokyo Rose and the Japanese Beetles.''

Given his current love for international folk music, I ask Stallman if he had a similar affinity for Bob Dylan and the other folk musicians of the early 1960s. Stallman shakes his head. ``I did like Peter, Paul and Mary,'' he says. ``That reminds me of a great filk.''

When I ask for a definition of ``filk,'' Stallman explains that the term is used in science fiction fandom to refer to the writing of new lyrics for songs.  (In recent decades, some filkers write melodies too.)  Classic filks include ``On Top of Spaghetti,'' a rewrite of ``On Top of Old Smokey,'' and ``Yoda,'' filk-master ``Weird'' Al Yankovic's Star Wars-oriented rendition of the Kinks tune, ``Lola.''

Stallman asks me if I would be interested in hearing the filk. As soon as I say yes, Stallman's voice begins singing in an unexpectedly clear tone, using the tune of ``Blowin' in the Wind'':

\begin{verse}
How much wood could a woodchuck chuck,\\
If a woodchuck could chuck wood?\\
How many poles could a polak lock,\\
If a polak could lock poles?\\
How many knees could a negro grow,\\
If a negro could grow knees?\\
The answer, my dear,\\
is stick it in your ear.\\
The answer is, stick it in your ear\ldots
\end{verse}

The singing ends, and Stallman's lips curl into another child-like half smile. I glance around at the nearby tables. The Asian families enjoying their Sunday lunch pay little attention to the bearded alto in their midst.\endnote{For Stallman's own filks, visit \url{http://www.stallman.org/doggerel.html}. To hear Stallman singing ``The Free Software Song,'' visit \url{http://www.gnu.org/music/free-software-song.html}.} After a few moments of hesitation, I finally smile too.

``Do you want that last cornball?'' Stallman asks, eyes twinkling. Before I can screw up the punch line, Stallman grabs the corn-encrusted dumpling with his two chopsticks and lifts it proudly. ``Maybe I'm the one who should get the cornball,'' he says.

The food gone, our conversation assumes the dynamics of a normal interview. Stallman reclines in his chair and cradles a cup of tea in his hands. We resume talking about Napster and its relation to the free software movement. Should the principles of free software be extended to similar arenas such as music publishing? I ask.

``It's a mistake to transfer answers from one thing to another,'' says Stallman, contrasting songs with software programs. ``The right approach is to look at each type of work and see what conclusion you get.''

When it comes to copyrighted works, Stallman says he divides the world into three categories. The first category involves ``functional'' works -- e.g., software programs, dictionaries, and textbooks. The second category involves works that might best be described as ``testimonial'' -- e.g., scientific papers and historical documents. Such works serve a purpose that would be undermined if subsequent readers or authors were free to modify the work at will.  It also includes works of personal expression -- e.g., diaries, journals, and autobiographies. To modify such documents would be to alter a person's recollections or point of view, which Stallman considers ethically unjustifiable.  The third category includes works of art and entertainment.

Of the three categories, the first should give users the unlimited right to make modified versions, while the second and third should regulate that right according to the will of the original author. Regardless of category, however, the freedom to copy and redistribute noncommercially should remain unabridged at all times, Stallman insists. If that means giving Internet users the right to generate a hundred copies of an article, image, song, or book and then email the copies to a hundred strangers, so be it. ``It's clear that private occasional redistribution must be permitted, because only a police state can stop that,'' Stallman says. ``It's antisocial to come between people and their friends. Napster has convinced me that we also need to permit, must permit, even noncommercial redistribution to the public for the fun of it. Because so many people want to do that and find it so useful.''

When I ask whether the courts would accept such a permissive outlook, Stallman cuts me off.

``That's the wrong question,'' he says. ``I mean now you've changed the subject entirely from one of ethics to one of interpreting laws. And those are two totally different questions in the same field. It's useless to jump from one to the other. How the courts would interpret the existing laws is mainly in a harsh way, because that's the way these laws have been bought by publishers.''

The comment provides an insight into Stallman's political philosophy: just because the legal system currently backs up businesses' ability to treat copyright as the software equivalent of land title doesn't mean computer users have to play the game according to those rules. Freedom is an ethical issue, not a legal issue. ``I'm looking beyond what the existing laws are to what they should be,'' Stallman says. ``I'm not trying to draft legislation. I'm thinking about what should the law do? I consider the law prohibiting the sharing of copies with your friend the moral equivalent of Jim Crow. It does not deserve respect.''

The invocation of Jim Crow prompts another question. How much influence or inspiration does Stallman draw from past political leaders? Like the civil-rights movement of the 1950s and 1960s, his attempt to drive social change is based on an appeal to timeless values: freedom, justice, and fair play.

Stallman divides his attention between my analogy and a particularly tangled strand of hair. When I stretch the analogy to the point where I'm comparing Stallman with Dr. Martin Luther King, Jr., Stallman, after breaking off a split end and popping it into his mouth, cuts me off.

``I'm not in his league, but I do play the same game,'' he says, chewing.

I suggest Malcolm X as another point of comparison. Like the former Nation of Islam spokesperson, Stallman has built up a reputation for courting controversy, alienating potential allies, and preaching a message favoring self-sufficiency over cultural integration.

Chewing on another split end, Stallman rejects the comparison. ``My message is closer to King's message,'' he says. ``It's a universal message. It's a message of firm condemnation of certain practices that mistreat others. It's not a message of hatred for anyone. And it's not aimed at a narrow group of people. I invite anyone to value freedom and to have freedom.''

Many criticize Stallman for rejecting handy political alliances; some psychologize this and describe it as a character trait. In the case of his well-publicized distaste for the term ``open source,'' the unwillingness to participate in recent coalition-building projects seems understandable. As a man who has spent the last two decades stumping on the behalf of free software, Stallman's political capital is deeply invested in the term. Still, comments such as the ``Han Solo'' comparison at the 1999 LinuxWorld have only reinforced Stallman's reputation, amongst those who believe virtue consists of following the crowd, as a disgruntled mossback unwilling to roll with political or marketing trends.

``I admire and respect Richard for all the work he's done,'' says Red Hat president Robert Young, summing up Stallman's paradoxical political conduct. ``My only critique is that sometimes Richard treats his friends worse than his enemies.''

[RMS: The term ``friends'' only partly fits people such as Young, and companies such as Red Hat.  It applies to some of what they did, and do: for instance, Red Hat contributes to development of free software, including some GNU programs.  But Red Hat does other things that work against the free software movement's goals -- for instance, its versions of GNU/Linux contain nonfree software.  Turning from deeds to words, referring to the whole system as ``Linux'' is unfriendly treatment of the GNU Project, and promoting ``open source'' instead of ``free software'' rejects our values.  I could work with Young and Red Hat when we were going in the same direction, but that was not often enough to make them possible allies.]

Stallman's reluctance to ally the free software movement with other political causes is not due to lack of interest in them.  Visit his offices at MIT, and you instantly find a clearinghouse of left-leaning news articles covering civil-rights abuses around the globe. Visit his personal web site, \url{stallman.org}, and you'll find attacks on the Digital Millennium Copyright Act, the War on Drugs, and the World Trade Organization.  Stallman explains, ``We have to be careful of entering the free software movement into alliances with other political causes that substantial numbers of free software supporters might not agree with.  For instance, we avoid linking the free software movement with any political party because we do not want to drive away the supporters and elected officials of other parties.''

Given his activist tendencies, I ask, why hasn't Stallman sought a larger voice? Why hasn't he used his visibility in the hacker world as a platform to boost his political voice?  [RMS: But I do -- when I see a good opportunity.  That's why I started \url{stallman.org}.]

Stallman lets his tangled hair drop and contemplates the question for a moment. [RMS:  My quoted response doesn't fit that question.  It does fit a different question, ``Why do you focus on free software rather than on the other causes you believe in?''  I suspect the question I was asked was more like that one.]

``I hesitate to exaggerate the importance of this little puddle of freedom,'' he says. ``Because the more well-known and conventional areas of working for freedom and a better society are tremendously important. I wouldn't say that free software is as important as they are. It's the responsibility I undertook, because it dropped in my lap and I saw a way I could do something about it. But, for example, to end police brutality, to end the war on drugs, to end the kinds of racism we still have, to help everyone have a comfortable life, to protect the rights of people who do abortions, to protect us from theocracy, these are tremendously important issues, far more important than what I do. I just wish I knew how to do something about them.''

Once again, Stallman presents his political activity as a function of personal confidence. Given the amount of time it has taken him to develop and hone the free software movement's core tenets, Stallman is hesitant to believe he can advance the other causes he supports.

``I wish I knew how to make a major difference on those bigger issues, because I would be tremendously proud if I could, but they're very hard and lots of people who are probably better than I am have been working on them and have gotten only so far,'' he says. ``But as I see it, while other people were defending against these big visible threats, I saw another threat that was unguarded. And so I went to defend against that threat. It may not be as big a threat, but I was the only one there [to oppose it].''

Chewing a final split end, Stallman suggests paying the check. Before the waiter can take it away, however, Stallman pulls out a white-colored dollar bill and throws it on the pile. The bill looks so clearly counterfeit, I can't help but pick it up and read it. Sure enough, it did not come from the US Mint. Instead of bearing the image of a George Washington or Abe Lincoln, the bill's front side bears the image of a cartoon pig. Instead of the United States of America, the banner above the pig reads, ``Untied Status of Avarice.'' The bill is for zero dollars,\endnote{RMS: Williams was mistaken to call this bill ``counterfeit.'' It is legal tender, worth zero dollars for payment of any debt.  Any U.S. government office will convert it into zero dollars' worth of gold.} and when the waiter picks up the money, Stallman makes sure to tug on his sleeve.

``I added an extra zero to your tip,'' Stallman says, yet another half smile creeping across his lips.

The waiter, uncomprehending or fooled by the look of the bill, smiles and scurries away.

``I think that means we're free to go,'' Stallman says.

\theendnotes
\setcounter{endnote}{0}
