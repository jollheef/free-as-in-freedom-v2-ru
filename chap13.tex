%% Copyright (c) 2002, 2010 Sam Williams
%% Copyright (c) 2010 Richard M. Stallman
%% Permission is granted to copy, distribute and/or modify this
%% document under the terms of the GNU Free Documentation License,
%% Version 1.3 or any later version published by the Free Software
%% Foundation; with no Invariant Sections, no Front-Cover Texts, and
%% no Back-Cover Texts. A copy of the license is included in the
%% file called ``gfdl.tex''.


\chapter{Битва продолжается}

Времени не под силу излечить все раны Ричарда Столлмана, но всё же время -- хороший союзник.

Прошло 4 года после публикации \enquote{Собора и Базара}. Столлман всё так же раздражён критикой Реймонда. Всё так же ворчит он по поводу вознесения Торвальдса до самого известного хакера в мире. Ричард вспоминает футболку, которая стала популярной на Linux-мероприятиях в районе 1999 года. Футболку, оформленную в стиле \enquote{Звёздных войн}, где Торвальдс размахивает световым мечом подобно Люку Скайуокеру, а лицо Столлмана приделано к дроиду R2D2. Столлмана раздражает этот рисунок -- он изображает Торвальдса не только его закадычным другом, но и лидером сообщества свободного ПО. Хотя сам Торвальдс, между прочим, от этой роли отказывается. \enquote{Какая ирония, -- печально замечает Столлман, -- ведь на самом деле Торвальдс отказывается поднимать меч. Он приковывает к себе всеобщее внимание, из-за чего все видят его символом движения, и при этом не хочет сражаться. Что тут хорошего?}

С другой стороны, именно нежелание Торвальдса \enquote{поднять меч} оставило за Столлманом репутацию непримиримого стража хакерской этики. Пусть за последние годы произошло много плохого для Ричарда и его движения, но ведь и хорошего тоже! Пусть успех GNU/Linux оттеснил Столлмана на периферию всеобщего внимания, пусть люди считают, что они используют \enquote{Linux}, но Столлман и не думает сдавать позиции. Он тихо и незаметно наращивает своё политическое влияние. Только за период с января 2000 года по декабрь 2001 года он посещает 6 континентов и, что особенно важно, такие страны как Индия и Китай -- страны, где свободное ПО это едва ли не жизненная необходимость.

Универсальная общественная лицензия GNU (GPL) тоже приносит неплохие плоды. Летом 2000 года Столлман и фонд свободного ПО одерживают 2 крупные победы. В июле норвежская компания Trolltech лицензирует свой мощный графический фреймворк Qt под GPL. Спустя несколько недель компания Sun Microsystems, которая ранее использовала открытый код, не участвуя при этом в разработке, наконец-то добавила ослабленную версию GPL (LGPL) к собственной Sun Industry Standards Source License (SISSL) для офисного пакета OpenOffice. Такое двойное лицензирование обеспечило почву для грядущего перевода OpenOffice \endnote{Из-за проблем с торговыми марками Sun пришлось сменить название пакета на нелепое \enquote{OpenOffice.org}} в разряд свободных программ.

Шаг компании Trolltech -- результат длительных усилий проекта GNU. Несвобода Qt доставляла серьёзные проблемы, потому что это был единственный полноценный графический фреймворк в мире GNU/Linux. На его основе создавалась свободная среда рабочего стола KDE. Trolltech разрешила свободным проектам вроде KDE бесплатно использовать Qt, но это не решало проблем. Ведь люди, для которых была важна свобода софта, не могли использовать Qt по моральным соображениям. Столлман понимал, что очень многие хотят графический рабочий стол в GNU/Linux, и это желание может легко пересилить стремление иметь полностью свободную операционную систему. Если люди начнут устанавливать и использовать KDE в GNU/Linux, они автоматически начнут использовать и несвободный Qt. Это разрушило бы всё, за что боролся проект GNU.

Столлман не стал сидеть сложа руки, для решения проблемы он запустил целых 2 параллельных проекта. Первый проект -- GNOME, свободная среда рабочего стола GNU. Второй -- Harmony, совместимая свободная версия фреймворка Qt. Если GNOME понравится людям, то надобность в KDE отпадёт, но если люди всё-таки продолжат выбирать KDE, то можно будет вытащить второй козырь -- Harmony, который освободит KDE от привязки к несвободному Qt. В любом случае люди получат полностью свободную систему с графическим рабочим столом.

Усилия Столлмана оказываются успешными, руководство Trolltech чувствует себя неуютно. Они решают выпустить Qt под собственной свободной лицензией QPL. Однако Столлман обращает внимание на то, что совмещение в одной программе кода под лицензиями QPL и GPL неизбежно нарушает одну из них. В итоге руководство Trolltech сдаётся и применяет к Qt двойное лицензирование -- под QPL и GPL одновременно. Это победа.

С освобождением Qt становится ненужным Harmony, и разработчики закрывают проект, тем более, что до готовности ему далеко. GNOME же полюбился многим пользователем и продолжает развиваться как основной графический рабочий стол GNU.

Компания Sun принимает правила игры фонда свободного ПО. На O'Reilly Open Source Conference 1999 года соучредитель и главный научный сотрудник Sun Microsystems Билл Джой ещё защищает фирменную лицензию \enquote{коллективного исходного кода}. Она разрешает пользователям копировать и редактировать код программ Sun, но продавать эти копии без согласования с Sun нельзя. Из-за этого ограничения лицензия не может считаться свободной, и даже не соответствует принципам открытого кода. Проходит год, и вице-президент Sun Microsystems Марко Бёррис на той же сцене провозглашает новый шаг компании в отношении лицензирования OpenOffice -- офисного пакета, разработанного, в частности, для GNU/Linux.

\enquote{Я хочу сказать всего три буквы, -- говорит Бёррис, -- GPL}.

По его словам, решение компании не связано со Столлманом. Оно вызвано наблюдениями за жизнью и развитием GPL-программ. \enquote{Суть в том, что разные продукты привлекают разные сообщества, и выбирая разные лицензии, можно привлекать к программе разную аудиторию, -- говорит Бёррис, -- и совершенно очевидно, что аудитория OpenOffice имеет сильное пересечение с сообществом GPL}.\endnote{Марко Бёррис, интервью июля 2000 года).} К сожалению, эта победа оказалась неполной, потому что OpenOffice рекомендует использовать несвободные дополнения.

Всё это говорит о недооценённой силе GPL и о политической гениальности человека, благодаря которому эта лицензия появилась. \enquote{Ни один юрист в мире не смог бы создать GPL такой, какая она есть, -- утверждает Эбен Моглен, профессор права Колумбийского университета и главный юрисконсульт фонда свободного ПО, -- но она создана и работает. И работает она благодаря философии дизайна Ричарда}.

Бывший профессиональный программист Моглен работает со Столлманом ещё с 1990 года, когда тот попросил его о частной юридической помощи. Моглен, которому также довелось оказывать поддержку Филиппу Циммерману в его судебных тяжбах с федеральным правительством касательно программы шифрования PGP, с готовностью откликнулся на просьбу.\endnote{Больше информации об этой истории можно найти у Стивена Леви в книге \textit{КРИПТО: как криптографы победили правительство, отстояв приватность для эпохи цифровых технологий}.}

\enquote{Я сказал Ричарду, что использую Emacs каждый божий день, и что мне придётся очень долго работать на него, чтобы воздать ему должное}.

Моглен, наверное, лучше всех знает о том, как хакерская философия Столлмана переходила в область права. Он говорит, что подход Ричарда к программному коду и юридическому языку во многом одинаков. \enquote{Как юрист, я могу сказать, что в идее удаления всех ошибок из юридического документа не особенно много смысла, -- объясняет Моглен, -- в каждом юридическом деле есть неопределённости, и адвокаты стараются обратить их в пользу своего клиента. У Ричарда же полностью противоположная цель. Он хочет устранить любые неопределённости, что в принципе невозможно. Невозможно создать одну лицензию, которая предусматривала бы все обстоятельства во всех правовых системах мира. Но если вы бы на это пошли, вам стоило бы идти его путём. Его элегантность и простота почти достигают означенной цели. Добавьте немного адвокатуры, и получите то что нужно}.

Иногда Моглена обвиняют в том, что он стал локомотивом радикализма Столлмана, и он понимает разочарование людей, которые могли бы стать союзниками при более мягких условиях. \enquote{Ричард -- это тот человек, что не идёт на компромиссы в базовых вопросах. Ему нелегко вертеть словами и прятаться за двусмысленностями, чего общество часто требует от людей}.

Моглен не только оказывает поддержку фонду свободного ПО, но и помогает другим ответчикам по делам об авторском праве. Например, он участвовал в процессе Дмитрия Склярова и распространителей программы для дешифровки DVD-дисков deCSS.

Скляров, будучи программистом в российской компании, написал программу для взлома цифровой защиты электронных книг Adobe. В его стране это было законно, потому что законов об авторском праве, подобных американским, в России просто не было. Но когда Скляров приехал на конференцию в США, его арестовали. Столлман с большой охотой участвовал в акциях протеста против Adobe, а фонд свободного ПО объявил Закон о защите авторских прав в цифровом тысячелетии (DMCA) \enquote{цензурой программного обеспечения}. От защиты самой программы Склярова и её распространения, впрочем, пришлось откреститься, потому что она была несвободна и незаконна в США. Столлман только осудил правительство США за запрет deCSS. Моглен же выступил непосредственным адвокатом обвиняемых.

После решения фонда свободного ПО не участвовать в подобных историях Моглен начинает ценить упрямство Столлмана. \enquote{За все эти годы я много раз приходил к Ричарду и говорил: нам нужно сделать это, нам нужно сделать то, вот здесь стратегический момент, вот тут хороший ход, вот это мы просто обязаны сделать. И Ричард всегда отвечал: нам не нужно делать ничего, просто подожди, и всё устаканится}.

\enquote{И знаете что? -- добавляет Моглен. -- Так и случалось раз за разом}.

Подобные комментарии диссонируют с самооценкой Столлмана. \enquote{Я не очень хорош в подобных играх, -- отвечает он тем, кто считает его великим стратегом, -- я не умею смотреть в будущее и предвидеть действия других людей. Мой подход заключается в том, чтобы сосредоточиться на фундаменте идей, сделать его настолько прочным, насколько это возможно.\hspace{0.01in}}

Растущая популярность GPL и её сила притяжения -- лучшие свидетельства тому, что фундамент удался на славу. Столлман, конечно, никогда не был единственным, кто делает свободное ПО, но этическую основу свободного ПО создал именно он. Неважно, выбирают современные программисты GPL или что-то другое, комфортно они себя чувствуют в рамках философии свободного ПО или нет -- величайшее наследие Столлмана заключается в том, что у них вообще есть выбор.

Обсуждение его наследия сейчас выглядит, конечно, преждевременным. Столлман ещё вполне может добавить что-то внушительное в этот список. Тем не менее, огромный импульс движения за свободное ПО подталкивает к изучению жизни Столлмана вне постоянных битв на поле программной индустрии и ещё более суровых исторических событий.

К чести Ричарда, он отказывается обсуждать заранее свою эпитафию. \enquote{У меня никогда не получалось детально продумывать своё будущее, -- говорит Столлман, -- я просто говорил себе: \enquote{Что ж, буду бороться. Посмотрим, к чему это приведёт}.\hspace{0.01in}}

Нет сомнений в том, что воюя за себя, Столлман оттолкнул многих людей -- тех людей, что могли бы стать великими, если бы Столлман воевал за них. Это тоже свидетельство прозрачной этики Ричарда, в адрес которой даже его политические противники высказывают уважение. Однако напряжённая противоречивость между Столлманом-идеологом и Столлманом-хакером всё же заставляет поневоле задаваться вопросом: как люди будут смотреть на Столлмана, когда им не будет мешать его личность?

В первых черновиках этой книги я назвал этот вопрос \enquote{вопросом 100 лет}. Надеясь достичь объективного представления о Столлмане и его делах, я попросил разных светил индустрии ПО вывести себя за рамки настоящего и поставить на место человека, который живёт через 100 лет после нас и изучает историю свободного софта. С нынешней позиции легко видеть сходство Столлмана с американцами прошлого, жизнь которых была весьма маргинальна, но со временем обрела значительную ценность. На ум сразу приходит Генри Дэвид Торо, философ-трансценденталист и автор эссе \textit{Гражданское неповиновение}, а также Джон Мьюр, основатель Sierra Club и прародитель современного экологического движения. Без труда, впрочем, можно проследить и сходство с Вильямом Дженнингсом Брайаном, также известным как \enquote{великий простой человек} -- лидером популистского движения, врагом монополий и человеком, что затерялся во времени, несмотря на всё своё тогдашнее влияние.

Хоть Столлман и не первый, кто считал программное обеспечение общественным достоянием, ему всё же гарантирована сноска в будущих учебниках истории, хотя бы из-за GPL. Учитывая этот факт, есть смысл проанализировать наследие Ричарда Столлмана в отрыве от настоящего времени. Будут программисты 2102 года использовать GPL, или к тому времени эта лицензия уже затеряется в забвении? Будет ли термин \enquote{свободное ПО} казаться таким же странным, каким нам сегодня кажется \enquote{бесплатное серебро}, или он окажется пугающе пророческим?

Прогнозирование будущего -- рискованная игра. Сам Столлман отказывается обсуждать, что будет через 100 лет, и как будут размышлять люди будущего. Он предпочитает задаваться вопросом: \enquote{Что мы можем сделать сейчас ради лучшего будущего?}. Но многие люди охотно играют в эту игру.

\enquote{Через сто лет Ричард и ещё пара человек заслужат куда большего, чем простую сноску, -- говорит Моглен, -- на них будут смотреть, как на тех, кто творил историю}.

Под этой \enquote{парой человек} скрывается Джон Гилмор, что внёс огромный вклад в различные свободные проекты и основал Фонд электронных рубежей, и Теодор (Тед) Нельсон, автор книги \textit{Литературные машины} 1982 года. Моглен говорит, что Столлман, Гилмор и Нельсон -- исторические фигуры, чьи заслуги не пересекаются друг с другом. Он отдаёт должное Нельсону, который, как считается, придумал термин \enquote{гипертекст}, чтобы подчеркнуть трудности в вопросах владения информацией в цифровую эпоху. Гилмор и Столлман, в свою очередь, заслуживают признания за выявление негативных эффектов контроля над информацией, и за создание организаций для борьбы с этими эффектами. Однако деятельность Столлмана Моглен считает более личностной и менее политичной.

\enquote{Уникальность Ричарда в том, что он уже в самом начале понял, к каким этическим последствиям приведёт несвободное ПО, -- говорит Моглен, -- это во многом связано с его личностью, и люди, когда рассказывают о Столлмане, часто изображают это как побочный или даже негативный эффект его жизни}.

Гилмор, который называет своё включение в почётный список между непредсказуемым Нельсоном и вспыльчивым Столлманом чем-то вроде \enquote{сомнительной славы}, тем не менее, вторит аргументам Моглена. Вот что он пишет:

\begin{quote}
Я думаю, что работы Столлмана будут стоять на одном уровне с работами Томаса Джефферсона; он довольно ясный писатель, и принципы его прозрачны\ldots Обретёт ли Ричард такое же влияние, как Джефферсон, зависит от того, станут ли через сто лет абстракции, что мы называем \enquote{гражданскими правами}, важнее абстракций, что мы называем \enquote{программным обеспечением} или \enquote{технически наложенными ограничениями}.
\end{quote}

Гилмор напоминает, что есть ещё одна часть столлмановского наследия, которую ни в коем случае нельзя упускать из виду -- коллективная модель разработки программ, впервые реализованная в рамках проекта GNU. Эта модель, хоть и имеет недостатки, стала стандартом для индустрии разработки ПО. И эта модель, наверное, важнее и влиятельнее, чем проект GNU, лицензия GPL и все программы, разработанные Столлманом.

\begin{quote}
До появления интернета было довольно сложно наладить сотрудничество между удалёнными друг от друга командами разработчиков, даже если эти команды прекрасно знали и доверяли друг другу. Ричард же организовал коллективную разработку программ, в частности -- одиночными добровольцами, которые редко встречались друг с другом. Ричард не создавал инструменты для такой разработки (протокол TCP, почтовые рассылки, утилиты diff и patch, архивы tar, системы контроля версий RCS и CVS), но он использовал весь их потенциал для формирования нового типа социальных групп: коллектива программистов, которые могут работать сообща, невзирая на географию.
\end{quote}

Столлман, впрочем, считает, что эта похвала упускает суть его работы. \enquote{Он подчёркивает, что метод разработки важнее свободы, и это созвучно идеям сторонников открытого кода, но не сторонников свободного ПО. Если будущие поколения будут видеть проект GNU именно в таком свете, то, боюсь, это будет мир, где разработчики держат пользователей в неволе, лишь иногда позволяя им помогать делать работу программистов, не снимая при этом цепей}.

Лоуренс Лессиг, профессор права Стэнфордского университета и автор книги \textit{Будущее идей}, также оптимистичен. Лессиг и многие его коллеги рассматривают GPL как основной оплот так называемых \enquote{цифровых общин}, как основу огромной агломерации принадлежащих им программ, сетей и стандартов, благодаря которым произошёл взрывной рост интернета. Лессиг не пытается связать Столлмана с такими пионерами интернета, как Ванневар Буш, Винтон Серф и Джозеф Карл Робнетт Ликлайдер, которые убеждали общество смотреть на компьютерные технологии более глобально. Он считает, что влияние Столлмана более лично, самосозерцательно и, наконец, уникально:

\begin{quote}
Столлман сменил тон дискуссий с обсуждения того, что есть, на обсуждение того, что должно быть. Он помог людям увидеть, что на самом деле поставлено на кон, и создал инструмент для продвижения своих идеалов\ldots Поэтому я не знаю, как поместить его в один ряд с Серфом или Ликлайдером. Это новация другого рода. Она заключает в себе не какой-то новый тип кода или создание возможности для появления интернета. Она заключает в себе новое восприятие, благодаря которому люди увидели, что некоторые аспекты интернета таят в себе огромную ценность. Не думаю, что кто-нибудь когда-нибудь делал что-то подобное.
\end{quote}

Конечно, не все видят наследие Столлмана высеченным в камне. Сторонник открытого кода Эрик Реймонд считает, что лидерская роль Столлмана значительно поблекла с 1996 года, и теперь, глядя в хрустальный шар на 2102 год, видит нечто противоречивое:

\begin{quote}
Я думаю, что созданные Столлманом культурные объекты вроде GPL, Emacs и GCC будут восприниматься как революционные работы, фундамент информационного мира. Но также я думаю, что история будет не столь доброжелательна к некоторым идеям и концепциям, которыми оперирует РМС, и будет совсем недоброжелательна к его личной склонности к территориальному поведению и его замашкам лидера культа.
\end{quote}

Столлман тоже видит некоторую противоречивость в будущем:

\begin{quote}
То, что история скажет о проекте GNU через двадцать лет, зависит от того, кто победит в битве за свободу использования общественных знаний. Если мы проиграем, мы будем простой сноской. Если победим, то всё равно неясно, будут ли люди понимать роль операционной системы GNU -- если они будут продолжать думать, что пользуются \enquote{Линуксом}, то у них сложится ложная картина о том, что, как и почему происходило.

Но даже если мы победим, то знания и представления людей через сто лет будут зависеть от того, кто в то время будет иметь политическую власть.
\end{quote}

Столлман проводит свою историческую аналогию, приведя в пример Джона Брауна, воинствующего аболициониста, которого по одну сторону линии Мейсона-Диксона считали героем, а по другую -- безумцем.

Восстание Джона Брауна так и не состоялось, но во время судебного разбирательства он фактически мотивировал нацию на отмену рабства. В период Гражданской войны его считали героем. Следующие 100 лет, особенно в начале XX века, учебники истории называли его сумасшедшим. В эпоху законной расовой сегрегации, несмотря на весь её позор, США частично приняли интерпретацию Юга, и в учебники истории попало много неправды о Гражданской войне и связанных с нею событиях.

Такое сравнение показывает, что сам Ричард считает свою текущую работу второстепенной, а свою репутацию -- двойственной. Конечно, трудно представить, чтобы репутация Столлмана была так же опозорена, как репутация Брауна в период после Реконструкции. Ричард, несмотря на его метафоры и аналогии на военную тему, вряд ли сделал что-то, что вдохновило бы людей на насилие. Тем не менее, можно без труда представить себе будущее, в котором идеи Столлмана обратятся в прах. \endnote{РМС: Далее Сэм Вильямс пишет следующее: \enquote{При разработке свободного ПО возникает не массовое движение, а набор небольших коллективов, что противостоят искушениям собственничества}, и это не соответствует действительности. С самого первого объявления о проекте GNU я призываю людей вливаться в наши ряды. Движение за свободное ПО стремится стать массовым, и единственный вопрос здесь -- достаточно ли у него сторонников, чтобы называться \enquote{массовым}. В 2009 году фонд свободного ПО насчитывает более 3000 участников, платящих взносы, и более 20 тысяч подписчиков на ежемесячную новостную рассылку.}

С другой стороны, воля Столлмана может сделать его наследие поистине бессмертным. Моглен пристально наблюдал за Ричардом целое десятилетие, и теперь возражает всем, кто считает личность Столлмана контрпродуктивной, а созданные им культурные объекты -- чем-то побочным. Моглен говорит, что без этой личности никаких таких культурных объектов не было бы вообще. Вот что говорит он, вспоминая свою работу секретарём Верховного суда:

\begin{quote}
Слушай, величайшим человеком, на которого я когда-либо работал, был Тергуд Маршалл, судья Верховного суда. Я знаю, как он стал великим человеком. Я знаю, как ему удалось изменить мир. Сравнивать его со Столлманом было бы рискованно, потому что невозможно себе представить более разных людей: Тергуд Маршалл был очень социальным человеком, пусть и среди изгоев, но всё-таки социальным. Его талантом было жить в обществе. Но он тоже был законченным и самодостаточным. Столлман отличается от него во всех отношениях, кроме одного -- он такой же законченный и самодостаточный, скроенный из звёздного вещества.
\end{quote}

Пытаясь развернуть этот образ, Моглен вспоминает весну 2000 года. Биржевой успех VA Linux всё ещё будоражит деловую прессу, и связанные со свободным ПО темы всё ещё плавают в новостях. Вокруг бушует ураган проблем и историй, каждую из которых можно обсуждать бесконечно, но Моглен садится обедать со Столлманом и ощущает спокойствие, словно в глазу урагана. Целый час они обсуждают одну тему: укрепление GPL.

\enquote{Мы сидим и говорим о проблемах в Восточной Европе, о проблемах владения данными, которые начали угрожать свободному ПО, -- вспоминает Моглен, -- и у меня возникает сторонняя мысль: кем мы выглядим сейчас в глазах окружающих? Сидят два маленьких бородатых анархиста и что-то планируют. Ричард, конечно, вытаскивает колтуны из волос и бросает их в суп, и вообще ведёт себя как обычно. Любой, кто слышит разговор, думает, что мы сумасшедшие. Но я-то знаю, что вот здесь, за этим столом, творится революция. И этот человек -- тот, кто её делает, и тот, кто сделал её возможной}.

Этот момент дал лучшее понимание простоты Столлмана.

\enquote{Забавно, -- продолжает Моглен, -- я говорю ему: \enquote{Знаешь, Ричард, мы с тобой, наверное, единственные парни, которые нисколько не заработали на этой революции}. А потом вытаскиваю кошелёк и плачу за обед, потому что знаю, что у него денег нет}.\endnote{РМС: Я никогда не отказываюсь от того, чтобы люди оплатили за меня обед, потому что моё достоинство не базируется на собирании чеков. Но я уверен, что деньги на обед у меня тогда были. Конечно, мой доход, половину которого составляют выступления, намного меньше зарплаты профессора права, но я не бедный человек.}

\theendnotes
\setcounter{endnote}{0}
