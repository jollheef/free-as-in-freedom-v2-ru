%% Copyright (c) 2002, 2010 Sam Williams
%% Copyright (c) 2010 Richard M. Stallman
%% Permission is granted to copy, distribute and/or modify this
%% document under the terms of the GNU Free Documentation License,
%% Version 1.3 or any later version published by the Free Software
%% Foundation; with no Invariant Sections, no Front-Cover Texts, and
%% no Back-Cover Texts. A copy of the license is included in the
%% file called ``gfdl.tex''.

\chapter{Святитель Игнуциус}

Гавайский центр высокопроизводительных вычислений Мауи расположился в одноэтажном здании на пыльных красных холмах чуть выше города Кихеи. Среди многомиллионных пейзажей и мультимиллионных объектов гольф-клуба \enquote{Серебрянный меч} центр выглядит настоящей бессмыслицей. От стерильной коробки здания Техносквера и научных центров Аргонна, Лос-Аламоса, Нью-Мексико центр отделяют тысячи миль, и кажется, что учёные здесь больше времени проводят загорая под солнцем, нежели трудясь над своими исследовательскими проектами.

Это впечатление верно лишь наполовину. Хотя местные научные сотрудники не упускают возможности отдохнуть на чудесных пляжах, они также уделяют большое внимание своей работе. Согласно веб-сайту \url{Top500.org}, который отслеживает 500 самых мощных суперкомпьютеров мира, здешний суперкомпьютер IBM SP Power3 выдавал на тот момент 837 миллиардов операций с плавающей точкой в секунду, что делало его одним из 25 самых мощных компьютеров планеты. Этой машиной совместно управляли Гавайский университет и ВВС США, деля её процессорные такты между исследованиями физики высокотемпературной плазмы и нуждами военной логистики.

В общем, центр высокопроизводительных вычислений Мауи -- уникальное место, в котором гармонично сочетаются высокоинтеллектуальная атмосфера научного мира и расслабленная атмосфера Гавайских островов. На сайте центра образца 2000 года это выражалось в красноречивом слогане: \enquote{Вычисления в раю}.

И меньше всего вы ожидаете встретить в этом месте Ричарда Столлмана, который стоит перед окном в кабинете научного сотрудника и любуется видом канала Мауи. \enquote{Слишком много солнца}, -- бормочет Ричард. Он -- посланник из одного компьютерного рая в другой, и должен выполнить свою миссию, даже если это причиняет боль его глазам, привыкшим к полумраку комнат и тёмным цветам компьютерных экранов.

Когда я прихожу в конференц-зал, он уже битком набит людьми. Половой баланс аудитории чуть улучшился по сравнению с нью-йоркской речью -- здесь где-то 85\% мужчин и 15\% женщин. Около половины слушателей одеты в брюки цвета хаки и футболки с логотипами. Остальные выглядят очень по-гавайски -- яркие рубашки с цветочным рисунком и лица с глубоким оттенком охры. Единственное, что выдаёт в них гиков -- вещи в их руках: сотовые телефоны Nokia, карманные компьютеры Palm Pilot и ноутбуки Sony VAIO.

Столлман в своей простой синей футболке, коричневых брюках из полиэстера и белых носках на фоне всей этой публики выглядит белой вороной. Люминесцентные лампы подчёркивают нездоровую бледность его кожи, голодающей по солнцу. \endnote{РМС: Убеждение, что кожа может \enquote{голодать по солнцу}, или что бледность \enquote{нездорова} -- не соответствует реальности и очень опасно. Отсутствие солнца не причинит вам вреда, если в вашем организме достаточно витамина D. А вот постоянное пребывание под ярким солнцем может повредить вашу кожу или даже убить вас.} Его большой бороды и гривы волос достаточно, чтобы взмокла шея самого хладнокровного гавайца. Ему не хватает только надписи \enquote{континентал} на лбу, чтобы выглядеть максимально инородно. [RMS: Он так говорит, будто выглядеть отлично от других -- это что-то плохое.]

Пока Ричард слоняется по передней части зала, несколько слушателей, одетых в футболки с логотипом местной группы пользователей FreeBSD, возятся с настройкой видеокамеры и звукового оборудования. FreeBSD -- свободное ответвление системы BSD, академической версии Unix 70-х годов, и по совместительству -- основной конкурент GNU/Linux. Но хакеры, вне зависимости от используемых ими систем, записывают и документируют речи Столлмана с таким же пылом и воодушевлением, какой был у легендарной армии поклонников рок-группы Grateful Dead. Как заявили руководители местной группы пользователей FreeBSD, они хотят быть уверены, что коллеги-программисты в Гамбурге, Мумбае и Новосибирске не пропустят ни одного последнего откровения Ричарда Мэттью Столлмана.

Сравнение с Grateful Dead вполне уместно. Рассказывая о деловых возможностях модели свободного ПО, Столлман нередко приводит в пример эту рок-группу. Grateful Dead отказались мешать поклонникам записывать свои выступления и свободно раздавать эти записи. Благодаря этому решению Grateful Dead стали больше, чем просто рок-группой. Они стали центром очень большого и преданного сообщества, посвящённого их творчеству. Такая горячая поддержка позволила группе вообще не заключать контракты со звукозаписывающими компаниями, и жить исключительно за счёт музыкальных туров и живых выступлений. На гастролях 1994 года Grateful Dead собрали 52 миллиона долларов на одних только входных билетах. \endnote{\enquote{Grateful Dead Time Capsule: 1985-1995 North American Tour Grosses,} \url{http://www.dead101.com/1197.htm}.}

Немногие компании-разработчики программного обеспечения достигают таких успехов, так что фактор преданности сообщества стал одной из причин роста популярности свободных лицензий. Компании вроде IBM, Sun Microsystem, Hewlett-Packard решили, что публикация исходных кодов -- хороший шаг, который поможет им обрести такую же преданную армию поклонников. Эти компании решили следовать если не духу, то хотя бы букве столлмановского манифеста свободного ПО. Называя GPL \textit{Великой хартией вольностей} индустрии высоких технологий, колумнист ZDNet Эван Лейбович высказал мнение, что рост внимания к различным аспектам проекта GNU -- это не просто мода. \enquote{Это социальный сдвиг, позволяющий пользователям вернуть себе власть над своим будущим, -- пишет он, -- подобно тому, как \textit{Великая хартия вольностей} давала права британским подданным, GPL обеспечивает права и свободы потребителей от лица пользователей программ}.\endnote{Evan Leibovitch, \enquote{Who's Afraid of Big Bad Wolves,} \textit{ZDNet Tech Update} (December 15, 2000), \url{http://www.zdnet.com/news/whos-afraid-of-the-big-bad-wolves/298394}.}

Фактор преданности сообщества помогает понять, почему 40 с лишним программистов собрались в этом зале, хотя могли бы провести это время за научными исследованиями или чтением самоучителей по виндсёрфингу.

Столлман не тратит время на ознакомительные вступления, как было в Нью-Йорке. Как только команда пользователей FreeBSD заканчивает возиться с оборудованием, Ричард просто делает шаг вперёд и начинает говорить, прекратив все разговоры в зале.

\enquote{Когда в обществе обсуждают правила использования программного обеспечения, подавляющее большинство участников оказываются работниками софтверных компаний, -- начинает Столлман, -- и вопрос они рассматривают исключительно с точки зрения корысти: какие правила можно навязать людям, чтобы они платили побольше денег? В 70-х годах мне посчастливилось оказаться в коллективе программистов, которые делились программами. И поэтому я всегда смотрю на вопрос с другой точки зрения: какие правила пошли бы на пользу обществу в целом и отдельному человеку в частности? Конечно, я пришёл к совершенно иному ответу}.

И снова Ричард быстро пробегает историю с лазерным принтером Xerox, чтобы разыграть тот же эпизод с театральным указыванием пальцем в некоторых слушателей. Также он уделяет пару минут названию GNU/Linux.

\enquote{Люди иногда говорят мне: \enquote{Неужели вопрос заслуг настолько важен, чтобы уделять ему столько внимания? Ведь важнее всего то, что работа сделана, а не признание, почести и всё такое}. Ну да, эти слова были бы мудрыми, будь они правильными. Ведь работа заключается не в том, чтобы создать операционную систему, а в том, чтобы сделать свободными пользователей компьютеров. А для этого мы должны добиться того, чтобы с компьютерами можно было свободно делать всё, что хочется. Именно такова цель проекта GNU}.\endnote{В рамках повествования я не стал углубляться в столлмановское определение свободы программного обеспечения. На веб-сайте проекта GNU перечислены 4 основных компонента этой свободы:

\begin{itemize}
  \item Свобода запускать программу в любых целях как вам захочется (Свобода 0).
  \item Свобода изучать исходный код программы и менять его для своих целей (Свобода 1).
  \item Свобода раздавать копии программы, чтобы помочь ближнему (Свобода 2).
  \item Свобода раздавать изменённые вами версии программы ради блага общества (Свобода 3).
\end{itemize}

Более полную информацию можно найти в статье \enquote{Определение свободного программного обеспечения} на странице \url{http://www.gnu.org/philosophy/free-sw.html}.}

\enquote{Нам предстоит ещё много работы}, -- добавляет Столлман.

Кому-то в аудитории уже известны эти речи, кому-то они кажутся немного заумными. Один из слушателей в футболке начинает дремать, Ричард прерывается и просит кого-нибудь растормошить заснувшего.

\enquote{Мне как-то раз сказали, что у меня очень успокаивающий голос, и спросили, не пробовал ли я себя в целительстве, -- говорит он, немедленно вызвав смех аудитории, -- наверное, имелось в виду, что я могу помочь вам погрузиться в сладкий сон. Некоторым из вас это действительно не помешало бы, и я не буду возражать против этого. Если вам нужно поспать -- обязательно поспите}.

Речь завершается кратким обсуждением патентов на программное обеспечение. Они всё сильнее беспокоят и представителей индустрии ПО, и членов сообщества свободного софта. Подобно Napster, патенты на софт показывают нелепость попыток расширить законы и концепции, созданные для мира физических предметов, на бесконечную вселенную информационных технологий.

Авторское право и патентное право работают по-разному и приводят к разным последствиям в сфере программного обеспечения. Авторские права на программу принадлежат её разработчику, они регламентируют копирование и редактирование исходного кода. Но авторское право не работает в отношении идей. Это значит, что разработчик волен реализовать по-своему в своей программе те функции и возможности, что он увидел в других программах, защищённых авторским правом. Также он может, хоть это и неимоверно трудно, декодировать бинарные файлы программы, чтобы подсмотреть какие-нибудь идеи и алгоритмы, и в изменённом виде перенести их в свою программу. Этот подход называется \enquote{обратной разработкой} или \enquote{реверс-инжинирингом}.

Патенты -- совсем другое дело. Согласно ведомству по патентам и товарным знакам США, компании и отдельные люди могут патентовать новаторские или хотя бы неизвестные ведомству компьютерные идеи. В теории, это позволило бы патентообладателям открывать свои технологии в обмен на 20-летнее монопольное положение на рынке. На практике же, раскрытие технологии не особенно полезно для общества, потому что принцип работы программы зачастую понятен без описания, а в остальных случаях его можно узнать через обратную разработку. При этом патентное право запрещает создавать программы с функциональностью, которая аналогична запатентованной.

В индустрии ПО, где 20 лет -- это порой целый жизненный цикл рыночной ниши, патенты становятся стратегическим оружием. Компании вроде Apple и Microsoft некогда бились за авторские права и дизайн различных технологий, но теперь технологические гиганты используют патенты, чтобы застолбить конкретные приложения и бизнес-модели. Хорошо известный и весьма печальный пример -- попытка Amazon запатентовать онлайн-покупки в 1 клик. Также большинство компаний использует патенты на софт в качестве оружия обороны. Они заключают между собой сделки с кросс-лицензированием, создавая таким образом баланс и снижая напряжённость в отношениях. Но можно вспомнить и примеры, как производители ПО в области шифрования и обработки изображений патентами давят конкурентов. Так, свободным проектам запрещено использовать некоторые возможности отрисовки шрифтов, потому что они защищены патентами Apple.

Ситуация с патентами в сфере программ, по мнению Столлмана, особенно хорошо показывает, что хакерам нужно всегда оставаться начеку. Она говорит и о том, что политические преимущества свободного ПО куда важнее технических и потребительских. Ричард считает, что нужно обращать максимум внимания не на производительность и стоимость свободных операционных систем вроде GNU/Linux и FreeBSD, хотя по этим параметрам они уже обошли несвободных конкурентов. Главное здесь -- свобода пользователей и разработчиков.

Позиция Ричарда идёт вразрез с позицией сообщества -- евангелисты и просто сторонники открытого ПО в его продвижении делают ставку именно на практические выгоды в ущерб философским и политическим ценностям. Их главные аргументы крутятся вокруг превосходства хакерской коллективной модели разработки, а не вокруг необходимости свободных лицензий для защиты интересов пользователей. Суть этой аргументации сводится к мощи коллегиальной экспертной оценки: открытый исходный код позволяет эффективнее выявлять и исправлять ошибки, что делает системы GNU/Linux или FreeBSD куда более качественными для среднего пользователя.

Нельзя сказать, что термин \enquote{открытый исходный код} политически бесполезен. Его сторонники могут привести как минимум 2 аргумента в пользу именно такой формулировки. Во-первых, слово \enquote{свободный} многие предприниматели толкуют, в первую очередь, как \enquote{бесплатный}, что не совсем соответствует реальности. Во-вторых, акцент на открытости обращает внимание компаний на практическое, технологическое преимущество свободного ПО, а не на отвлечённых этических ценностях. Эрик Реймонд, один из основателей Open Source Initiative и один из главных сторонников использования термина \enquote{открытый код}, изложил свой отказ следовать политическим идеям Столлмана в статье 1999 года с красноречивым названием \enquote{Заткнись и покажи им код}:

\begin{quote}
Риторика РМС очень соблазнительна для людей вроде нас. Мы, хакеры -- мыслители и идеалисты, которые легко откликаются на апелляции к \enquote{принципам}, \enquote{правам}, \enquote{свободе}. Даже если мы не согласны с некоторыми пунктами политической программы Столлмана, нам хотелось бы, чтобы его риторика действовала на людей; нам кажется, что она должна действовать, и мы приходим в замешательство, когда эта риторика оставляет равнодушными 95\% населения, которые не настолько \enquote{в теме}, как мы. \endnote{Eric Raymond, \enquote{Shut Up and Show Them the Code}, (June 28, 1999), \url{http://www.catb.org/~esr/writings/shut-up-and-show-them.html}.}
\end{quote}

Реймонд обращает внимание, что среди этих 95\% населения -- менеджеры предприятий, инвесторы и обычные пользователи, которые своей многочисленностью определяют направления развития рынка ПО. Он утверждает: если не заинтересовать и не завоевать расположение всех этих людей, то программисты со своей идеологией обречены плестись в хвосте:

\begin{quote}
Когда РМС настаивает, что мы говорим о \enquote{правах пользователей компьютеров}, он создаёт очень опасный соблазн повторно наступить на старые грабли и потерпеть неудачу. Мы должны отказаться от такой риторики -- не потому, что она неправильна, а потому, что она в контексте программного обеспечения звучит убедительно только для таких как мы, но не для остальных людей. Людей вне хакерской культуры она сбивает с толку и отталкивает.\endnote{\textit{Ibid.}}
\end{quote}

Вот что возражает на это Столлман:

\begin{quote}
Реймонд пытается объяснить причины нашей старой неудачи, но никакой неудачи не было. Наше дело велико, и нам следует пройти большой путь, но не меньший путь мы уже прошли.

Реймонд слишком пессимистичен касательно ценностей не-хакеров. Многие не-хакеры больше озабочены политическими вопросами, нежели техническими преимуществами, на которые делают упор сторонники открытого кода. Среди таких людей встречаются даже лидеры стран, хоть и нечасто.

Именно этические идеалы свободного софта, а не \enquote{более высокое качество}, подтолкнули президентов Эквадора и Бразилии перевести свои правительства на свободное ПО. Они не гики, но понимают концепцию свободы.
\end{quote}

Но главный минус риторики вокруг открытого кода в том, что она ослабляет и подрывает политическую позицию. Многие пользователи довольствуются открытыми и бесплатными программами, они не видят причин для перехода на свободный софт. Да, открытые программы дают пользователям какую-то свободу, но это не полноценная свобода, и пользователи не получают должного опыта, чтобы распознавать и ценить свободу, а потому могут легко её лишиться. Что, например, произойдёт, когда развитие свободной программы упрётся в какой-нибудь патент?

Большинство сторонников открытого кода столь же радикальны в отношении патентов на ПО, что и Столлман, если не больше того. Равно как и большинство разработчиков собственнических программ -- ведь патенты угрожают и им тоже. Но Ричард обращает внимание, что позиция движения за свободное ПО в отношении ограничительной роли патентов несколько отличается от позиции сторонников открытого кода.

\enquote{Мы ограничены в развитии программ не потому, что нам недостаёт способностей и умений, -- говорит Столлман, -- а потому, что нас лишают права развивать программы. Кто-то запрещает нам служить на благо общества. Что же случится, когда пользователи столкнутся с этим? Если они прислушивались к сторонникам открытого кода, которые сулили более высокую функциональность благодаря открытости, то у них будут все основания сказать: \enquote{Вы не исполнили того, что обещали. Ваша программа не более функциональна, в ней не хватает вот этой запатентованной возможности. Вы нам солгали}. Но если пользователи принимают позицию движения за свободное ПО, то они скажут: \enquote{Да как они смеют лишать нас этой функции и нашей свободы?}. Такая позиция поможет нам выстоять против любых патентных атак}.

В том, как Столлман излагает свою политическую позицию, трудно увидеть что-то заумное или отталкивающее. Да, сам Ричард может показаться кому-то неприятным, несимпатичным, но его аргументация ясна и логична. Один из слушателей спрашивает: не лишают ли себя сторонники свободного ПО последних достижений технологического прогресса, отвергая собственнические программы? Столлман отвечает на это с позиции личного мировоззрения. \enquote{Я считаю, что свобода важнее технического прогресса, -- говорит он, -- я всегда предпочту свободную программу более продвинутой несвободной программе, потому что не хочу жертвовать своей свободой. Мой принцип таков: если я не могу этим поделиться, то я это не приму}.

Человек, который отождествляет этику с религией, может воспринять такой ответ в религиозном свете. Но если иудей, соблюдающий кошер, или мормон, избегающий алкоголя, просто повинуются заповедям, то Столлман отстаивает свою свободу. Его объяснения показывают рациональность его позиции: собственнические программы отбирают вашу свободу, так что если вы хотите её сохранить, вам следует отказаться от таких программ.

Ричард придаёт своему выбору свободных программ вместо собственнических окрас личной убеждённости, к которой надеется приобщить других людей. Но, в отличие от евангелистов, он не пытается вбить эту убеждённость в головы слушателей. Люди и без того редко уходят с выступлений Ричарда, потому что хотят узнать о праведном программном обеспечении.

Словно пытаясь закрепить эффект, Столлман проделывает необычный ритуал. Он вытаскивает из продуктового пакета чёрную мантию и надевает её, после чего извлекает оттуда же блестящий коричневый диск от компьютера, и водружает его себе на голову подобно нимбу. По аудитории расползаются испуганные смешки.

\enquote{Я -- святитель ИГНУциус из Церкви Емаксовой, -- произносит Ричард, подняв правую руку, -- и я благословляю твой компьютер, сын мой}.

\begin{figure}[ht] \centering
  \includegraphics{stignucius}
  \caption{Столлман в одеянии святителя ИГНУциуса. Фото сделано Стианом Эйкеланном в норвежском Бергене 19 февраля 2009 года.}
\end{figure}

За считанные секунды смех переходит в аплодисменты. Компьютерный диск на голове Столлмана ловит лучи света, создавая идеальное гало. Ричард в этот момент -- словно живая православная икона.

\enquote{Сначала Emacs был текстовым редактором, -- объясняет Столлман, -- но со временем стал образом жизни для многих и религией для некоторых. Мы называем эту религию Церковью Емаксовой}.

Эта сценка -- беспечная самопародия, юмористический укол в адрес тех людей, что считают компьютерный аскетизм Столлмана скрытой формой религиозного фанатизма. Также это своевременная разрядка атмосферы. Надев мантию и нимб, Ричард словно говорит аудитории: \enquote{Смеяться это нормально. Я знаю, что я странный}.  [РМС: Смеяться над кем-то из-за его странности -- это хамство, которому нет оправданий. Но я надеюсь, что людей смешит моя комедийная сценка со святителем ИГНУциусом.]

Впоследствии, обсуждая образ святителя ИГНУциуса, Столлман скажет, что придумал его в 1996 году, спустя многие годы после создания Emacs, но до появления термина \enquote{открытый исходный код}, который обострил борьбу за лидерство в хакерском сообществе. Тогда Ричард хотел, чтобы эта \enquote{пародия на себя} показала людям, что он упрям, но не фанатичен, как многие думают. Позже некоторые соперники Столлмана стали использовать этот образ, чтобы поднять собственную репутацию, как это сделал Эрик Реймонд в своём интервью сайту Linux.com в 1999 году:

\begin{quote}
Когда я говорю, что РМС подбирает свои слова и действия, я не хочу обвинить его в неискренности. Я имею в виду, что у него, как и у всякого хорошего публичного деятеля, есть актёрская жилка. Иногда он делает это осознанно -- вы когда-нибудь видели его в образе святителя ИГНУциуса, благословляющего программы с компьютерным диском на голове? В основном же он делает это бессознательно -- добивается такого баланса в раздражении людей, чтобы удерживать внимание публики, но при этом не напугать и не оттолкнуть её. \endnote{\enquote{Guest Interview: Eric S. Raymond,} \textit{Linux.com} (May 18, 1999).}
\end{quote}

Столлман не соглашается с этой аналитикой Реймонда. \enquote{Это всего лишь мой способ подшутить над собой, -- говорит он, -- если другие видят в нём что-то большее, то они видят лишь отражение их собственного мышления, а не моего}.

Однако Ричард признаётся, что бывает провокатором. \enquote{Серьёзно? -- сказал он однажды. -- Я обожаю быть в центре внимания}. Чтобы научиться получше провоцировать публику, Столлман даже как-то раз вступил в Toastmasters -- организацию, которая помогает людям обрести навыки публичных выступлений. У него есть чувство сцены, которому могли бы позавидовать многие артисты театра, и он ощущает родство с водевилями прошлого. Через несколько дней после речи Столлмана в вычислительном центре Мауи, я вспоминаю его номер на LinuxWorld 1999 года и спрашиваю, не чувствует ли он себя подобно Граучо Марксу -- одиночкой, не желающим вступать в какой-либо клуб. \endnote{РМС: Вильямс неверно понимает известные слова Граучо, истолковывая их в ключе психологии личности, тогда как это был укол в сторону антисемитизма многих клубов, которые отказывались принимать его в свои ряды. Я сам не понимал этого, пока мама не объяснила мне. Вильямс и я росли, когда эта нетерпимость уже ушла в тень, и потому у людей не стало необходимости прятать критику нетерпимости в юморе, как это делал Граучо.} Ричард мгновенно отвечает: \enquote{Нет, но я во многом восхищаюсь Граучо Марксом, и он, безусловно, во многом вдохновил меня. Хотя меня также вдохновил и Харпо, его брат}.

Влияние Граучо Маркса очевидно проявляется в любви Столлмана к каламбурам. С другой стороны, слабость к каламбурам и игре слов -- общая черта многих хакеров. А вот совершенно невозмутимая подача каламбуров -- это, наверное, чисто \enquote{граучовая} черта характера Ричарда. Когда замечаешь, как Столлман каламбурит без малейшего намёка на улыбку, задаёшься вопросом: кто над кем смеётся больше -- аудитория над ним, или наоборот?

Явление святителя ИГНУциуса аудитории в центре высокопроизводительных вычислений Мауи, кажется, снимает все сомнения. Столлман, конечно, не стэндап-комик, но без труда веселит полный зал инженеров. \enquote{Пребывание святителем Церкви Емаксовой не требует безбрачия, но требует нравственной чистоты, -- вещает он, -- вы должны изгнать со своих компьютеров зло собственнических систем, чтобы установить чистейше и совершеннейше свободные системы. И в лоно их вы должны ставить только свободные программы. Если вы примете этот обет и будете следовать ему, тогда вы тоже станете святителем Церкви Емаксовой, и обретёте нимб над головой}.

Сценка со святителем ИГНУциусом заканчивается шуткой для посвящённых. В большинстве Unix-систем в качестве текстового редактора используется конкурент Emacs -- программа vi, разработанная бывшим студентом Калифорнийского университета Беркли и главным исследователем Sun Microsystems Биллом Джоем. И прежде, чем снять с головы \enquote{нимб}, Столлман вышучивает программу-соперницу его детища. \enquote{Иногда люди спрашивают меня, грешно ли в Церкви Емаксовой принимать vi, -- говорит он, -- так вот, принимать свободную версию vi это не грех, это искупление. Удачного хакерства}. \endnote{Приобщение к Церкви Емаксовой несколько упростилось с 2001 года. Теперь пользователи могут присоединиться к ней, прочитав Символ Веры: \enquote{Нет системы, кроме GNU, и Linux -- ядро её}. Иногда Столлман упоминает религиозную церемонию Фубар-мицва, Великий Раскол между различными версиями Emacs, а также культ Непорочной Девы Емаксовой (то есть, любого новичка, который ещё не освоился в редакторе Emacs). Кроме того, \enquote{vi vi vi} был объявлен Редактором Зверя.}

После этого проходит недолгий сеанс вопросов-ответов, и слушатели собираются вокруг Столлмана. Некоторые просят автографы. \enquote{Я подпишу это, -- говорит Столлман, принимая от одной женщины распечатку лицензии GNU GPL, -- но только если вы пообещаете говорить GNU/Linux вместо Linux, и скажете всем своим друзьям говорить так}.

Этот эпизод подтверждает: в отличие от многих других сценических артистов и публичных деятелей, Столлман не \enquote{выключается}. Если не считать образ святителя ИГНУциуса, на сцене и вне сцены Ричард -- один и тот же человек. Тем же вечером, во время ужина, когда один программист что-то скажет о программах с \enquote{открытым кодом}, Столлман упрекнёт его в ответ: \enquote{Вы имеете в виду свободные программы. Так это называется правильно}.

Во время вопросов-ответов Столлман иногда позволяет себе менторский тон. \enquote{Многие говорят, мол, давайте сначала наприглашаем людей в наше сообщество, а потом уже будем учить их свободе. Может, это и эффективная стратегия, только вот у нас полным-полно людей, которые готовы приглашать других, и почти не найти людей, которые готовы учить их свободе}.

В итоге, как говорит Ричард, получается что-то вроде мегаполиса в стране третьего мира. \enquote{Миллионы приезжают в ваш город и ютятся в трущобах, и никто не утруждает себя следующим шагом: вытащить эти миллионы из трущоб. Если вы считаете, что продвигать свободный софт -- это хорошая идея, то, пожалуйста, перейдите сразу ко второму шагу. Потому что с желающими помочь на первом шаге у нас и так нет проблем}.

\enquote{Перейдите ко второму шагу} означает, что именно свобода, а не приобщение и вовлечение -- главная цель движения за свободное ПО. Тот, кто надеется преобразовать индустрию собственнических программ изнутри, выглядит круглым дураком. \enquote{Менять систему изнутри опасно, -- говорит Столлман, -- если вы работаете не на уровне Горбачёва, вас нейтрализуют}.

Поднимаются руки. Ричард указывает на человека в футболке. \enquote{Как вы предлагаете бороться с коммерческим шпионажем без патентов?}

\enquote{На самом деле, это два несвязанных между собой вопроса}, -- говорит Столлман.

\enquote{Но если кто-то хочет украсть у компании код...}

Столлман отшатывается, будто от перцового спрея. \enquote{Погодите! Украсть? Извините, но в этой фразе столько предубеждения, что я могу ответить только: я отвергаю это предубеждение}. Затем он переходит к сути вопроса. \enquote{Компании, создающие несвободное ПО и другие подобные вещи, многое хранят в секрете, и это вряд ли изменится. Раньше, даже в 80-е годы, программисты и не подозревали о существовании патентов на программы, не обращали на них внимания. Случалось так, что они публиковали интересные идеи и утаивали конкретику, если не хотели присоединяться к движению за свободное ПО. Теперь идеи покрываются патентами, а конкретика так же остаётся в тайне. Так что патенты не имеют никакого значения в любом случае}.

\enquote{Но если они не влияют на опубликование...} -- вмешивается другой слушатель, и Ричард перебивает его.

\enquote{Но они влияют! Сам факт опубликования запатентованной идеи говорит обществу, что она закрыта для свободного использования на ближайшие 20 лет. Что в этом хорошего, чёрт побери? Кроме того, патент составляется таким образом, чтобы описать идею как можно более расплывчато и трудно для понимания, так что бесполезно пытаться выжать из него какую-то информацию. Единственное, на что годятся патенты -- сообщать вам плохие новости о том, что вы больше не можете сделать}.

Аудитория замолкает. Речь началась в 3:15, а сейчас уже около 5 часов -- конец рабочего дня, и многие слушатели ёрзают на своих местах, готовясь рвануться к долгожданным выходным. Столлман оглядывает зал, чувствует повисшую в воздухе усталость, и поспешно закругляется. \enquote{Что ж, похоже, что мы закончили}, -- говорит он, чтобы подстегнуть возможных желающих задать последний вопрос. Когда никто не поднимает руки, Ричард произносит свою коронную финальную фразу.

\enquote{Удачного хакерства}.

\theendnotes
\setcounter{endnote}{0}
