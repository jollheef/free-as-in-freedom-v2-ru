%% Copyright (c) 2002, 2010 Sam Williams
%% Copyright (c) 2010 Richard M. Stallman
%% Permission is granted to copy, distribute and/or modify this
%% document under the terms of the GNU Free Documentation License,
%% Version 1.3 or any later version published by the Free Software
%% Foundation; with no Invariant Sections, no Front-Cover Texts, and
%% no Back-Cover Texts. A copy of the license is included in the
%% file called ``gfdl.tex''.

\chapter{Impeach God}


Although their relationship was fraught with tension, Richard Stallman would inherit one noteworthy trait from his mother: a passion for progressive politics.

It was an inherited trait that would take several decades to emerge, however. For the first few years of his life, Stallman lived in what he now admits was a ``political vacuum.''\endnote{See Michael Gross, ``Richard Stallman: High School Misfit, Symbol of Free Software, MacArthur-certified Genius'' (1999).} Like most Americans during the Eisenhower age, the Stallman family spent the Fifties trying to recapture the normalcy lost during the wartime years of the 1940s.

``Richard's father and I were Democrats but happy enough to leave it at that,'' says Lippman, recalling the family's years in Queens. ``We didn't get involved much in local or national politics.''

That all began to change, however, in the late 1950s when Alice divorced Daniel Stallman. The move back to Manhattan represented more than a change of address; it represented a new, independent identity and a jarring loss of tranquility.

``I think my first taste of political activism came when I went to the Queens public library and discovered there was only a single book on divorce in the whole library,'' recalls Lippman. ``It was very controlled by the Catholic church, at least in Elmhurst, where we lived. I think that was the first inkling I had of the forces that quietly control our lives.''

Returning to her childhood neighborhood, Manhattan's Upper West Side, Lippman was shocked by the changes that had taken place since her departure to Hunter College a decade and a half before. The skyrocketing demand for post-war housing had turned the neighborhood into a political battleground. On one side stood the pro-development city-hall politicians and businessmen hoping to rebuild many of the neighborhood's blocks to accommodate the growing number of white-collar workers moving into the city. On the other side stood the poor Irish and Puerto Rican tenants who had found an affordable haven in the neighborhood.

At first, Lippman didn't know which side to choose. As a new resident, she felt the need for new housing. As a single mother with minimal income, however, she shared the poorer tenants' concern over the growing number of development projects catering mainly to wealthy residents. Indignant, Lippman began looking for ways to combat the political machine that was attempting to turn her neighborhood into a clone of the Upper East Side.

Lippman says her first visit to the local Democratic party headquarters came in 1958. Looking for a day-care center to take care of her son while she worked, she had been appalled by the conditions encountered at one of the city-owned centers that catered to low-income residents. ``All I remember is the stench of rotten milk, the dark hallways, the paucity of supplies. I had been a teacher in private nursery schools. The contrast was so great. We took one look at that room and left. That stirred me up.''

The visit to the party headquarters proved disappointing, however. Describing it as ``the proverbial smoke-filled room,'' Lippman says she became aware for the first time that corruption within the party might actually be the reason behind the city's thinly disguised hostility toward poor residents. Instead of going back to the headquarters, Lippman decided to join up with one of the many clubs aimed at reforming the Democratic party and ousting the last vestiges of the Tammany Hall machine. Dubbed the Woodrow Wilson/FDR Reform Democratic Club, Lippman and her club began showing up at planning and city-council meetings, demanding a greater say.

``Our primary goal was to fight Tammany Hall, Carmine DeSapio and his henchman,''\endnote{Carmine DeSapio holds the dubious distinction of being the first Italian-American boss of Tammany Hall, the New York City political machine. For more information on DeSapio and the politics of post-war New York, see John Davenport, ``Skinning the Tiger: Carmine DeSapio and the End of the Tammany Era,'' \textit{New York Affairs} (1975): 3:1.} says Lippman. ``I was the representative to the city council and was very much involved in creating a viable urban-renewal plan that went beyond simply adding more luxury housing to the neighborhood.''

Such involvement would blossom into greater political activity during the 1960s. By 1965, Lippman had become an ``outspoken'' supporter for political candidates like William Fitts Ryan, a Democrat elected to Congress with the help of reform clubs and one of the first U.S. representatives to speak out against the Vietnam War.

It wasn't long before Lippman, too, was an outspoken opponent of U.S. involvement in Indochina. ``I was against the Vietnam War from the time Kennedy sent troops,'' she says. ``I had read the stories by reporters and journalists sent to cover the early stages of the conflict. I really believed their forecast that it would become a quagmire.''

Such opposition permeated the Stallman-Lippman household. In 1967, Lippman remarried. Her new husband, Maurice Lippman, a major in the Air National Guard, resigned his commission to demonstrate his opposition to the war. Lippman's stepson, Andrew Lippman, was at MIT and temporarily eligible for a student deferment. Still, the threat of induction should that deferment disappear, as it eventually did, made the risk of U.S. escalation all the more immediate. Finally, there was Richard who, though younger, faced the prospect of being drafted as the war lasted into the 1970s.

``Vietnam was a major issue in our household,'' says Lippman. ``We talked about it constantly: what would we do if the war continued, what steps Richard or his stepbrother would take if they got drafted. We were all opposed to the war and the draft. We really thought it was immoral.''

For Stallman, the Vietnam War elicited a complex mixture of emotions: confusion, horror, and, ultimately, a profound sense of political impotence. As a kid who could barely cope in the mild authoritarian universe of private school, Stallman experienced a shiver whenever the thought of Army boot camp presented itself. He did not think he could get through it and emerge sane.

``I was devastated by the fear, but I couldn't imagine what to do and didn't have the guts to go demonstrate,'' recalls Stallman, whose March 16th birthday earned him a low number in the dreaded draft lottery.  This did not affect him immediately, since he had a college deferment, one of the last before the U.S. stopped granting them; but it would affect him in a few years. ``I couldn't envision moving to Canada or Sweden. The idea of getting up by myself and moving somewhere. How could I do that? I didn't know how to live by myself. I wasn't the kind of person who felt confident in approaching things like that.''

Stallman says he was impressed by the family members who did speak out. Recalling a sticker, printed and distributed by his father, likening the My Lai massacre to similar Nazi atrocities in World War II, he says he was ``excited'' by his father's gesture of outrage. ``I admired him for doing it,'' Stallman says. ``But I didn't imagine that I could do anything. I was afraid that the juggernaut of the draft was going to destroy me.''

However, Stallman says he was turned off by the tone and direction of much of that movement. Like other members of the Science Honors Program, he saw the weekend demonstrations at Columbia as little more than a distracting spectacle.\endnote{Chess, another Columbia Science Honors Program alum, describes the protests as ``background noise.'' ``We were all political,'' he says, ``but the SHP was important. We would never have skipped it for a demonstration.''} Ultimately, Stallman says, the irrational forces driving the anti-war movement became indistinguishable from the irrational forces driving the rest of youth culture. Instead of worshiping the Beatles, girls in Stallman's age group were suddenly worshiping firebrands like Abbie Hoffman and Jerry Rubin. To a kid already struggling to comprehend his teenage peers, slogans like ``make love not war'' had a taunting quality. Stallman did not want to make war, at least not in Southeast Asia, but nobody was inviting him to make love either.

``I didn't like the counter culture much,'' Stallman recalls. ``I didn't like the music. I didn't like the drugs. I was scared of the drugs. I especially didn't like the anti-intellectualism, and I didn't like the prejudice against technology. After all, I loved a computer. And I didn't like the mindless anti-Americanism that I often encountered. There were people whose thinking was so simplistic that if they disapproved of the conduct of the U.S. in the Vietnam War, they had to support the North Vietnamese. They couldn't imagine a more complicated position, I guess.''

Such comments underline a trait that would become the key to Stallman's own political maturation. For Stallman, political confidence was directly proportionate to personal confidence. By 1970, Stallman had become confident in few things outside the realm of math and science. Nevertheless, confidence in math gave him enough of a foundation to examine the extremes of the anti-war movement in purely logical terms.  Doing so, Stallman found the logic wanting. Although opposed to the war in Vietnam, Stallman saw no reason to disavow war as a means for defending liberty or correcting injustice.

In the 1980s, a more confident Stallman decided to make up for his past inactivity by participating in mass rallies for abortion rights in Washington DC.  ``I became dissatisfied with my earlier self for failing in my duty to protest the Vietnam War,'' he explains.

In 1970, Stallman left behind the nightly dinnertime conversations about politics and the Vietnam War as he departed for Harvard. Looking back, Stallman describes the transition from his mother's Manhattan apartment to life in a Cambridge dorm as an ``escape.'' At Harvard, he could go to his room and have peace whenever he wanted it. Peers who watched Stallman make the transition, however, saw little to suggest a liberating experience.

``He seemed pretty miserable for the first while at Harvard,'' recalls Dan Chess, a classmate in the Science Honors Program who also matriculated at Harvard. ``You could tell that human interaction was really difficult for him, and there was no way of avoiding it at Harvard. Harvard was an intensely social kind of place.''

To ease the transition, Stallman fell back on his strengths: math and science. Like most members of the Science Honors Program, Stallman breezed through the qualifying exam for Math 55, the legendary ``boot camp'' class for freshman mathematics ``concentrators'' at Harvard. Within the class, members of the Science Honors Program formed a durable unit. ``We were the math mafia,'' says Chess with a laugh. ``Harvard was nothing, at least compared with the SHP.''

To earn the right to boast, however, Stallman, Chess, and the other SHP alumni had to get through Math 55. Promising four years worth of math in two semesters, the course favored only the truly devout. ``It was an amazing class,'' says David Harbater, a former ``math mafia'' member and now a professor of mathematics at the University of Pennsylvania. ``It's probably safe to say there has never been a class for beginning college students that was that intense and that advanced. The phrase I say to people just to get it across is that, among other things, by the second semester we were discussing the differential geometry of Banach manifolds. That's usually when their eyes bug out, because most people don't start talking about Banach manifolds until their second year of graduate school.''

Starting with 75 students, the class quickly melted down to 20 by the end of the second semester. Of that 20, says Harbater, ``only 10 really knew what they were doing.'' Of that 10, 8 would go on to become future mathematics professors, 1 would go on to teach physics.

``The other one,'' emphasizes Harbater, ``was Richard Stallman.''

Seth Breidbart, a fellow Math 55 classmate, remembers Stallman distinguishing himself from his peers even then.

``He was a stickler in some very strange ways,'' says Breidbart. There is a standard technique in math which everybody does wrong. It's an abuse of notation where you have to define a function for something and what you do is you define a function and then you prove that it's well defined. Except the first time he did and presented it, he defined a relation and proved that it's a function. It's the exact same proof, but he used the correct terminology, which no one else did. That's just the way he was.''

It was in Math 55 that Richard Stallman began to cultivate a reputation for brilliance. Breidbart agrees, but Chess, whose competitive streak refused to yield, says the realization that Stallman might be the best mathematician in the class didn't set in until the next year. ``It was during a class on Real Analysis,'' says Chess, now a math professor at Hunter College. ``I actually remember in a proof about complex valued measures that Richard came up with an idea that was basically a metaphor from the calculus of variations. It was the first time I ever saw somebody solve a problem in a brilliantly original way.''

For Chess, it was a troubling moment. Like a bird flying into a clear glass window, it would take a while to realize that some levels of insight were simply off limits.

``That's the thing about mathematics,'' says Chess. ``You don't have to be a first-rank mathematician to recognize first-rate mathematical talent. I could tell I was up there, but I could also tell I wasn't at the first rank. If Richard had chosen to be a mathematician, he would have been a first-rank mathematician.''\endnote{Stallman doubts this, however. ``One of the reasons I moved from math and physics to programming is that I never learned how to discover anything new in the former two.  I only learned to study what others had done.  In programming, I could do something useful every day.''}

For Stallman, success in the classroom was balanced by the same lack of success in the social arena. Even as other members of the math mafia gathered to take on the Math 55 problem sets, Stallman preferred to work alone. The same went for living arrangements. On the housing application for Harvard, Stallman clearly spelled out his preferences. ``I said I preferred an invisible, inaudible, intangible roommate,'' he says. In a rare stroke of bureaucratic foresight, Harvard's housing office accepted the request, giving Stallman a one-room single for his freshman year.

Breidbart, the only math-mafia member to share a dorm with Stallman that freshman year, says Stallman slowly but surely learned how to interact with other students. He recalls how other dorm mates, impressed by Stallman's logical acumen, began welcoming his input whenever an intellectual debate broke out in the dining club or dorm commons.

``We had the usual bull sessions about solving the world's problems or what would be the result of something,'' recalls Breidbart. ``Say somebody discovers an immortality serum. What do you do? What are the political results? If you give it to everybody, the world gets overcrowded and everybody dies. If you limit it, if you say everyone who's alive now can have it but their children can't, then you end up with an underclass of people without it. Richard was just better able than most to see the unforeseen circumstances of any decision.''

Stallman remembers the discussions vividly. ``I was always in favor of immortality,'' he says. ``How else would we be able to see what the world is like 200 years from now?'' Curious, he began asking various acquaintances whether they would want immortality if offered it. ``I was shocked that most people regarded immortality as a bad thing.'' Many said that death was good because there was no use living a decrepit life, and that aging was good because it got people prepared for death, without recognizing the circularity of the combination.

Although perceived as a first-rank mathematician and first-rate informal debater, Stallman shied away from clear-cut competitive events that might have sealed his brilliant reputation. Near the end of freshman year at Harvard, Breidbart recalls how Stallman conspicuously ducked the Putnam exam, a prestigious test open to math students throughout the U.S. and Canada. In addition to giving students a chance to measure their knowledge in relation to their peers, the Putnam served as a chief recruiting tool for academic math departments. According to campus legend, the top scorer automatically qualified for a graduate fellowship at any school of his choice, including Harvard.

Like Math 55, the Putnam was a brutal test of merit. A six-hour exam in two parts, it seemed explicitly designed to separate the wheat from the chaff. Breidbart, a veteran of both the Science Honors Program and Math 55, describes it as easily the most difficult test he ever took. ``Just to give you an idea of how difficult it was,'' says Breidbart, ``the top score was a 120, and my score the first year was in the 30s. That score was still good enough to place me 101st in the country.''

Surprised that Stallman, the best student in the class, had skipped the test, Breidbart says he and a fellow classmate cornered him in the dining common and demanded an explanation. ``He said he was afraid of not doing well,'' Breidbart recalls.

Breidbart and the friend quickly wrote down a few problems from memory and gave them to Stallman. ``He solved all of them,'' Breidbart says, ``leading me to conclude that by not doing well, he either meant coming in second or getting something wrong.''

Stallman remembers the episode a bit differently. ``I remember that they did bring me the questions and it's possible that I solved one of them, but I'm pretty sure I didn't solve them all,'' he says. Nevertheless, Stallman agrees with Breidbart's recollection that fear was the primary reason for not taking the test. Despite a demonstrated willingness to point out the intellectual weaknesses of his peers and professors in the classroom, Stallman hated and feared the notion of head-to-head competition -- so why not just avoid it?

``It's the same reason I never liked chess,'' says Stallman. ``Whenever I'd play, I would become so consumed by the fear of making a single mistake and losing that I would start making stupid mistakes very early in the game. The fear became a self-fulfilling prophecy.''  He avoided the problem by not playing chess.

Whether such fears ultimately prompted Stallman to shy away from a mathematical career is a moot issue. By the end of his freshman year at Harvard, Stallman had other interests pulling him away from the field. Computer programming, a latent fascination throughout Stallman's high-school years, was becoming a full-fledged passion. Where other math students sought occasional refuge in art and history classes, Stallman sought it in the computer-science laboratory.

For Stallman, the first taste of real computer programming at the IBM New York Scientific Center had triggered a desire to learn more. ``Toward the end of my first year at Harvard school, I started to have enough courage to go visit computer labs and see what they had. I'd ask them if they had extra copies of any manuals that I could read.'' Taking the manuals home, Stallman would examine the machine specifications to learn about the range of different computer designs.

One day, near the end of his freshman year, Stallman heard about a special laboratory near MIT. The laboratory was located on the ninth floor of a building in Tech Square, the mostly-commercial office park MIT had built across the street from the campus. According to the rumors, the lab itself was dedicated to the cutting-edge science of artificial intelligence and boasted the cutting-edge machines and software to match.

Intrigued, Stallman decided to pay a visit.

The trip was short, about 2 miles on foot, 10 minutes by train, but as Stallman would soon find out, MIT and Harvard can feel like opposite poles of the same planet. With its maze-like tangle of interconnected office buildings, the Institute's campus offered an aesthetic yin to Harvard's spacious colonial-village yang.  Of the two, the maze of MIT was much more Stallman's style. The same could be said for the student body, a geeky collection of ex-high school misfits known more for its predilection for pranks than its politically powerful alumni.

The yin-yang relationship extended to the AI Lab as well. Unlike Harvard computer labs, there was no grad-student gatekeeper, no clipboard waiting list for terminal access, no atmosphere of ``look but don't touch.'' Instead, Stallman found only a collection of open terminals and robotic arms, presumably the artifacts of some AI experiment. When he encountered a lab employee, he asked if the lab had any spare manuals it could loan to an inquisitive student. ``They had some, but a lot of things weren't documented,'' Stallman recalls. ``They were hackers, after all,'' he adds wryly, referring to hackers' tendency to move on to a new project without documenting the last one.

Stallman left with something even better than a manual: A job.  His first project was to write a PDP-11 simulator that would run on a PDP-10. He came back to the AI Lab the next week, grabbing an available terminal, and began writing the code.

Looking back, Stallman sees nothing unusual in the AI Lab's willingness to accept an unproven outsider at first glance. ``That's the way it was back then,'' he says. ``That's the way it still is now. I'll hire somebody when I meet him if I see he's good. Why wait? Stuffy people who insist on putting bureaucracy into everything really miss the point. If a person is good, he shouldn't have to go through a long, detailed hiring process; he should be sitting at a computer writing code.''

To get a taste of ``bureaucratic and stuffy,'' Stallman need only visit the computer labs at Harvard. There, access to the terminals was doled out according to academic rank. As an undergrad, Stallman sometimes had to wait for hours. The waiting wasn't difficult, but it was frustrating. Waiting for a public terminal, knowing all the while that a half dozen equally usable machines were sitting idle inside professors' locked offices, seemed the height of irrational waste. Although Stallman continued to pay the occasional visit to the Harvard computer labs, he preferred the more egalitarian policies of the AI Lab. ``It was a breath of fresh air,'' he says. ``At the AI Lab, people seemed more concerned about work than status.''

Stallman quickly learned that the AI Lab's first-come, first-served policy owed much to the efforts of a vigilant few. Many were holdovers from the days of Project MAC, the Department of Defense-funded research program that had given birth to the first time-share operating systems. A few were already legends in the computing world. There was Richard Greenblatt, the lab's in-house Lisp expert and author of MacHack, the computer chess program that had once humbled AI critic Hubert Dreyfus. There was Gerald Sussman, original author of the robotic block-stacking program HACKER. And there was Bill Gosper, the in-house math whiz already in the midst of an 18-month hacking bender triggered by the philosophical implications of the computer game LIFE.\endnote{See Steven Levy, \textit{Hackers} (Penguin USA [paperback], 1984): 144.\\Levy devotes about five pages to describing Gosper's fascination with LIFE, a math-based software game first created by British mathematician John Conway. I heartily recommend this book as a supplement, perhaps even a prerequisite, to this one.}

Members of the tight-knit group called themselves ``hackers.'' Over time, they extended the ``hacker'' description to Stallman as well. In the process of doing so, they inculcated Stallman in the ethical traditions of the ``hacker ethic.'' In their fascination with exploring the limits of what they could make a computer do, hackers might sit at a terminal for 36 hours straight if fascinated with a challenge. Most importantly, they demanded access to the computer (when no one else was using it) and the most useful information about it. Hackers spoke openly about changing the world through software, and Stallman learned the instinctual hacker disdain for any obstacle that prevented a hacker from fulfilling this noble cause. Chief among these obstacles were poor software, academic bureaucracy, and selfish behavior.

Stallman also learned the lore, stories of how hackers, when presented with an obstacle, had circumvented it in creative ways. This included various ways that hackers had opened professors' offices to ``liberate'' sequestered terminals. Unlike their pampered Harvard counterparts, MIT faculty members knew better than to treat the AI Lab's limited stock of terminals as private property. If a faculty member made the mistake of locking away a terminal for the night, hackers were quick to make the terminal accessible again -- and to remonstrate with the professor for having mistreated the community.  Some hackers did this by picking locks (``lock hacking''), some by removing ceiling tiles and climbing over the wall.  On the 9th floor, with its false floor for the computers' cables, some spelunked under it.  ``I was actually shown a cart with a heavy cylinder of metal on it that had been used to break down the door of one professor's office,''\endnote{Gerald Sussman, an MIT faculty member and hacker whose work at the AI Lab predates Stallman's, disputes this story. According to Sussman, the hackers never broke any doors to retrieve terminals.} Stallman says.

The hackers' insistence served a useful purpose by preventing the professors from egotistically obstructing the lab's work.  The hackers did not disregard people's particular needs, but insisted that these be met in ways that didn't obstruct everyone else.  For instance, professors occasionally said they had something in their offices which had to be protected from theft.  The hackers responded, ``No one will object if you lock your office, although that's not very friendly, as long as you don't lock away the lab's terminal in it.''

Although the academic people greatly outnumbered the hackers in the AI Lab, the hacker ethic prevailed. The hackers were the lab staff and students who had designed and built parts of the computers, and written nearly all the software that users used.  They kept everything working, too.  Their work was essential, and they refused to be downtrodden.  They worked on personal pet projects as well as features users had asked for, but sometimes the pet projects revolved around improving the machines and software even further. Like teenage hot-rodders, most hackers viewed tinkering with machines as its own form of entertainment.

Nowhere was this tinkering impulse better reflected than in the operating system that powered the lab's central PDP-10 computer. Dubbed ITS, short for the Incompatible Time Sharing system, the operating system incorporated the hacking ethic into its very design. Hackers had built it as a protest to Project MAC's original operating system, the Compatible Time Sharing System, CTSS, and named it accordingly. At the time, hackers felt the CTSS design too restrictive, limiting programmers' power to modify and improve the program's own internal architecture if needed. According to one legend passed down by hackers, the decision to build ITS had political overtones as well. Unlike CTSS, which had been designed for the IBM 7094, ITS was built specifically for the PDP-6. In letting hackers write the system themselves, AI Lab administrators guaranteed that only hackers would feel comfortable using the PDP-6. In the feudal world of academic research, the gambit worked. Although the PDP-6 was co-owned in conjunction with other departments, AI researchers soon had it to themselves.  Using ITS and the PDP-6 as a foundation, the Lab had been able to declare independence from Project MAC shortly before Stallman's arrival.\endnote{\textit{Ibid.}}

By 1971, ITS had moved to the newer but compatible PDP-10, leaving the PDP-6 for special stand-alone uses. The AI PDP-10 had a very large memory for 1971, equivalent to a little over a megabyte; in the late 70s it was doubled.   Project MAC had bought two other PDP-10s; all were located on the 9th floor, and they all ran ITS.  The hardware-inclined hackers designed and built a major hardware addition for these PDP-10s, implementing paged virtual memory, a feature lacking in the standard PDP-10.\endnote{I apologize for the whirlwind summary of ITS' genesis, an operating system many hackers still regard as the epitome of the hacker ethos. For more information on the program's political significance, see Simson Garfinkel, \textit{Architects of the Information Society: Thirty-Five Years of the Laboratory for Computer Science at MIT} (MIT Press, 1999).}

As an apprentice hacker, Stallman quickly became enamored with ITS. Although forbidding to some non-hackers, ITS boasted features most commercial operating systems wouldn't offer for years (or even to this day), features such as multitasking, applying the debugger immediately to any running program, and full-screen editing capability. 

``ITS had a very elegant internal mechanism for one program to examine another,'' says Stallman, recalling the program. ``You could examine all sorts of status about another program in a very clean, well-specified way.''  This was convenient
not only for debugging, but also for programs to start, stop or control other programs.

Another favorite feature would allow the one program to freeze another program's job cleanly, between instructions. In other operating systems, comparable operations might stop the program in the middle of a system call, with internal status that the user could not see and that had no well-defined meaning. In ITS, this feature made sure that monitoring the step-by-step operation of a program was reliable and consistent.

``If you said, `Stop the job,' it would always be stopped in user mode. It would be stopped between two user-mode instructions, and everything about the job would be consistent for that point,'' Stallman says. ``If you said, `Resume the job,' it would continue properly. Not only that, but if you were to change the (explicitly visible) status of the job and continue it, and later change it back, everything would be consistent. There was no hidden status anywhere.''

Starting in September 1971, hacking at the AI Lab had become a regular part of Stallman's weekly school schedule. From Sunday through Friday, Stallman was at Harvard. As soon as Friday afternoon arrived, however, he was on the subway, heading down to MIT for the weekend. Stallman usually made sure to arrive well before the ritual food run. Joining five or six other hackers in their nightly quest for Chinese food, he would jump inside a beat-up car and head across the Harvard Bridge into nearby Boston. For the next hour or so, he and his hacker colleagues would discuss everything from ITS to the internal logic of the Chinese language and pictograph system. Following dinner, the group would return to MIT and hack code until dawn, or perhaps go to Chinatown again at 3 a.m.

Stallman might stay up all morning hacking, or might sleep Saturday morning on a couch. On waking he would hack some more, have another Chinese dinner, then go back to Harvard.  Sometimes he would stay through Sunday as well.  These Chinese dinners were not only delicious; they also provided sustenance lacking in the Harvard dining halls, where on the average only one meal a day included anything he could stomach. (Breakfast did not enter the count, since he didn't like most breakfast foods and was normally asleep at that hour.)

For the geeky outcast who rarely associated with his high-school peers, it was a heady experience to be hanging out with people who shared the same predilection for computers, science fiction, and Chinese food. ``I remember many sunrises seen from a car coming back from Chinatown,'' Stallman would recall nostalgically, 15 years after the fact in a speech at the Swedish Royal Technical Institute. ``It was actually a very beautiful thing to see a sunrise, 'cause that's such a calm time of day. It's a wonderful time of day to get ready to go to bed. It's so nice to walk home with the light just brightening and the birds starting to chirp; you can get a real feeling of gentle satisfaction, of tranquility about the work that you have done that night.''\endnote{See Richard Stallman, ``RMS lecture at KTH (Sweden),'' (October 30, 1986), \url{http://www.gnu.org/philosophy/stallman-kth.html}.}

The more Stallman hung out with the hackers, the more he adopted the hacker world view. Already committed to the notion of personal liberty, Stallman began to infuse his actions with a sense of communal duty. When others violated the communal code, Stallman was quick to speak out. Within a year of his first visit, Stallman was the one opening locked offices to recover the sequestered terminals that belonged to the lab community as a whole. In true hacker fashion, Stallman also sought to make his own personal contribution to the art. One of the most artful door-opening tricks, commonly attributed to Greenblatt, involved bending a stiff wire into several right angles and attaching a strip of tape to one end. Sliding the wire under the door, a hacker could twist and rotate the wire so that the tape touched the inside doorknob. Provided the tape stuck, a hacker could turn the doorknob by pulling the handle formed from the outside end of the wire.

When Stallman tried the trick, he found it hard to execute. Getting the tape to stick wasn't always easy, and twisting the wire in a way that turned the doorknob was similarly difficult. Stallman thought about another method: sliding away ceiling tiles to climb over the wall. This always worked, if there was a desk
to jump down onto, but it generally covered the hacker in itchy fiberglass.  Was there a way to correct that flaw?  Stallman considered an alternative approach. What if, instead of slipping a wire under the door, a hacker slid away two ceiling panels and reached over the wall with a wire?

Stallman took it upon himself to try it out. Instead of using a wire, Stallman draped out a long U-shaped loop of magnetic tape with a short U of adhesive tape attached sticky-side-up at the base. Reaching across over the door jamb, he dangled the tape until it looped under the inside doorknob. Lifting the tape until the adhesive stuck, he then pulled on one end of the tape, thus turning the doorknob. Sure enough, the door opened. Stallman had added a new twist to the art of getting into a locked room.

``Sometimes you had to kick the door after you turned the door knob,'' says Stallman, recalling a slight imperfection of the new method. ``It took a little bit of balance to pull it off while standing on a chair on a desk.''

Such activities reflected a growing willingness on Stallman's part to speak and act out in defense of political beliefs. The AI Lab's spirit of direct action had proved inspirational enough for Stallman to break out of the timid impotence of his teenage years. Opening up an office to free a terminal wasn't the same as taking part in a protest march, but it was effective in a way that most protests weren't: it solved the problem at hand.

By the time of his last years at Harvard, Stallman was beginning to apply the whimsical and irreverent lessons of the AI Lab back at school.

``Did he tell you about the snake?'' his mother asks at one point during an interview. ``He and his dorm mates put a snake up for student election. Apparently it got a considerable number of votes.''

The snake was a candidate for election within Currier House, Stallman's dorm, not the campus-wide student council. Stallman does remember the snake attracting a fair number of votes, thanks in large part to the fact that both the snake and its owner both shared the same last name. ``People may have voted for it because they thought they were voting for the owner,'' Stallman says. ``Campaign posters said that the snake was `slithering for' the office. We also said it was an `at large' candidate, since it had climbed into the wall through the ventilating unit a few weeks before and nobody knew where it was.''

Stallman and friends also ``nominated'' the house master's 3-year-old son. ``His platform was mandatory retirement at age seven,'' Stallman recalls. Such pranks paled in comparison to the fake-candidate pranks on the MIT campus, however. One of the most successful fake-candidate pranks was a cat named Woodstock, which actually managed to outdraw most of the human candidates in a campus-wide election. ``They never announced how many votes Woodstock got, and they treated those votes as spoiled ballots,'' Stallman recalls. ``But the large number of spoiled ballots in that election suggested that Woodstock had actually won. A couple of years later, Woodstock was suspiciously run over by a car. Nobody knows if the driver was working for the MIT administration.'' Stallman says he had nothing to do with Woodstock's candidacy, ``but I admired it.''\endnote{In an email shortly after this book went into its final edit cycle, Stallman says he drew political inspiration from the Harvard campus as well. ``In my first year of Harvard, in a Chinese History class, I read the story of the first revolt against the Qin dynasty,'' he says.  (That's the one whose cruel founder burnt all the books and was buried with the terra cotta warriors.)  ``The story is not reliable history, but it was very moving.''}

At the AI Lab, Stallman's political activities had a sharper-edged tone. During the 1970s, hackers faced the constant challenge of faculty members and administrators pulling an end-run around ITS and its hacker-friendly design. ITS allowed anyone to sit down at a console and do anything at all, even order the system to shut down in five minutes. If someone ordered a shutdown with no good reason, some other user canceled it. In the mid-1970s some faculty members (usually those who had formed their attitudes elsewhere) began calling for a file security system to limit access to their data.   Other operating systems had such features, so those faculty members had become accustomed to living under security, and to the feeling that it was protecting them from something dangerous. But the AI Lab, through the insistence of Stallman and other hackers, remained a security-free zone.

Stallman presented both ethical and practical arguments against adding security. On the ethical side, Stallman appealed to the AI Lab community's traditions of intellectual openness and trust. On the practical side, he pointed to the internal structure of ITS, which was built to foster hacking and cooperation rather than to keep every user under control.  Any attempt to reverse that design would require a major overhaul. To make it even more difficult, he used up the last empty field in each file's descriptor for a feature to record which user had most recently changed the file. This feature left no place to store file security information, but it was so useful that nobody could seriously propose to remove it.

``The hackers who wrote the Incompatible Timesharing System decided that file protection was usually used by a self-styled system manager to get power over everyone else,'' Stallman would later explain. ``They didn't want anyone to be able to get power over them that way, so they didn't implement that kind of a feature. The result was, that whenever something in the system was broken, you could always fix it'' (since access control did not stand in your way).\endnote{See Richard Stallman (1986).}

Through such vigilance, hackers managed to keep the AI Lab's machines security-free. In one group at the nearby MIT Laboratory for Computer Sciences, however, security-minded faculty members won the day. The DM group installed its first password system in 1977. Once again, Stallman took it upon himself to correct what he saw as ethical laxity. Gaining access to the software code that controlled the password system, Stallman wrote a program to decrypt the encrypted passwords that the system recorded.  Then he started an email campaign, asking users to choose the null string as their passwords. If the user had chosen ``starfish,'' for example, the email message looked something like this:

\begin{quote}
I see you chose the password ``starfish''. I suggest that you switch to the password ``carriage return'', which is what I use. It's easier to type, and also opposes the idea of passwords and security.
\end{quote}

The users who chose ``carriage return'' -- that is, users who simply pressed the Enter or Return button, entering a blank string instead of a unique password -- left their accounts accessible to the world at large, just as all accounts had been, not long before. That was the point: by refusing to lock the shiny new locks on their accounts, they ridiculed the idea of having locks. They knew that the weak security implemented on that machine would not exclude any real intruders, and that this did not matter, because there was no reason to be concerned about intruders, and that no one wanted to intrude anyway, only to visit.

Stallman, speaking in an interview for the 1984 book \textit{Hackers}, proudly noted that one-fifth of the LCS staff accepted this argument and employed the null-string password.\endnote{See Steven Levy, \textit{Hackers} (Penguin USA [paperback], 1984): 417.}

Stallman's null-string campaign, and his resistance to security in general, would ultimately be defeated. By the early 1980s, even the AI Lab's machines were sporting password security systems. Even so, it represented a major milestone in terms of Stallman's personal and political maturation. Seen in the context of Stallman's later career, it represents a significant step in the development of the timid teenager, afraid to speak out even on issues of life-threatening importance, into the adult activist who would soon turn needling and cajoling into a full-time occupation.

In voicing his opposition to computer security, Stallman drew on many of the key ideas that had shaped his early life: hunger for knowledge, distaste for authority, and frustration over prejudice and secret rules that rendered some people outcasts. He would also draw on the ethical concepts that would shape his adult life: responsibility to the community, trust, and the hacker spirit of direct action. Expressed in software-computing terms, the null string represents the 1.0 version of the Richard Stallman political worldview -- incomplete in a few places but, for the most part, fully mature.

Looking back, Stallman hesitates to impart too much significance to an event so early in his hacking career. ``In that early stage there were a lot of people who shared my feelings,'' he says. ``The large number of people who adopted the null string as their password was a sign that many people agreed that it was the proper thing to do. I was simply inclined to be an activist about it.''

Stallman does credit the AI Lab for awakening that activist spirit, however. As a teenager, Stallman had observed political events with little idea as to how he could do or say anything of importance. As a young adult, Stallman was speaking out on matters in which he felt supremely confident, matters such as software design, responsibility to the community, and individual freedom. ``I joined this community which had a way of life which involved respecting each other's freedom,'' he says. ``It didn't take me long to figure out that that was a good thing. It took me longer to come to the conclusion that this was a moral issue.''

Hacking at the AI Lab wasn't the only activity helping to boost Stallman's esteem. At the start of his junior year at Harvard, Stallman began participating in a recreational international folk dance group which had just been started in Currier House. He was not going to try it, considering himself incapable of dancing, but a friend pointed out, ``You don't know you can't if you haven't tried.'' To his amazement, he was good at it and enjoyed it. What started as an experiment became another passion alongside hacking and studying; also, occasionally, a way to meet women, though it didn't lead to a date during his college career.  While dancing, Stallman no longer felt like the awkward, uncoordinated 10-year-old whose attempts to play football had ended in frustration. He felt confident, agile, and alive. In the early 80s, Stallman went further and joined the MIT Folk Dance Performing Group.   Dancing for audiences, dressed in an imitation of the traditional garb of a Balkan peasant, he found being in front of an audience fun, and discovered an aptitude for being on stage which later helped him in public speaking.

Although the dancing and hacking did little to improve Stallman's social standing, they helped him overcome the sense of exclusion that had clouded his pre-Harvard life. In 1977, attending a science-fiction convention for the first time, he came across Nancy the Buttonmaker, who makes calligraphic buttons saying whatever you wish. Excited, Stallman ordered a button with the words ``Impeach God'' emblazoned on it.

For Stallman, the ``Impeach God'' message worked on many levels. An atheist since early childhood, Stallman first saw it as an attempt to start a ``second front'' in the ongoing debate on religion. ``Back then everybody was arguing about whether a god existed,'' Stallman recalls. ```Impeach God' approached the subject from a completely different viewpoint. If a god was so powerful as to create the world and yet did nothing to correct the problems in it, why would we ever want to worship such a god? Wouldn't it be more just to put it on trial?''

At the same time, ``Impeach God'' was a reference to the the Watergate scandal of the 1970s, in effect comparing a tyrannical deity to Nixon.  Watergate affected Stallman deeply. As a child, Stallman had grown up resenting authority. Now, as an adult, his mistrust had been solidified by the culture of the AI Lab hacker community. To the hackers, Watergate was merely a Shakespearean rendition of the daily power struggles that made life such a hassle for those without privilege. It was an outsized parable for what happened when people traded liberty and openness for security and convenience.

Buoyed by growing confidence, Stallman wore the button proudly. People curious enough to ask him about it received a well-prepared spiel. ``My name is Jehovah,'' Stallman would say. ``I have a secret plan to end injustice and suffering, but due to heavenly security reasons I can't tell you the workings of my plan. I see the big picture and you don't, and you know I'm good because I told you so. So put your faith in me and obey me without question. If you don't obey, that means you're evil, so I'll put you on my enemies list and throw you in a pit where the Infernal Revenue Service will audit your taxes every year for all eternity.''

Those who interpreted the spiel as a parody of the Watergate hearings only got half the message. For Stallman, the other half of the message was something only his fellow hackers seemed to be hearing. One hundred years after Lord Acton warned about absolute power corrupting absolutely, Americans seemed to have forgotten the first part of Acton's truism: power, itself, corrupts. Rather than point out the numerous examples of petty corruption, Stallman felt content voicing his outrage toward an entire system that trusted power in the first place.

``I figured, why stop with the small fry,'' says Stallman, recalling the button and its message. ``If we went after Nixon, why not go after Mr. Big? The way I see it, any being that has power and abuses it deserves to have that power taken away.''

\theendnotes
\setcounter{endnote}{0}
