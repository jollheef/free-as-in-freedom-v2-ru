%% Copyright (c) 2002, 2010 Sam Williams
%% Copyright (c) 2010 Richard M. Stallman
%% Permission is granted to copy, distribute and/or modify this
%% document under the terms of the GNU Free Documentation License,
%% Version 1.3 or any later version published by the Free Software
%% Foundation; with no Invariant Sections, no Front-Cover Texts, and
%% no Back-Cover Texts. A copy of the license is included in the
%% file called ``gfdl.tex''.

\chapter{Развенчай бога}

Напряжённые отношения с матерью не помешали Ричарду унаследовать её страсть к прогрессивным политическим идеям. Но проявилось это далеко не сразу. Первые годы его жизни были полностью свободны от политики. Как говорит сам Столлман -- он жил в \enquote{политическом вакууме}\endnote{Источник: Michael Gross, \enquote{Richard Stallman: High School Misfit, Symbol of Free Software, MacArthur-certified Genius} (1999)}. При Эйзенхауэре большинство американцев не загружали себя глобальными проблемами, а старались лишь вернуться к нормальной человеческой жизни после 40-х годов, полных мрака и жестокости. Семья Столлманов не была исключением.

\enquote{Мы с отцом Ричарда были демократами, -- вспоминает Липпман семейные годы в Квинсе, -- но почти не участвовали в местной и общенациональной политической жизни. Мы были достаточно счастливы и довольны существующим порядком вещей}.

Всё начало меняться в конце 50-х, после развода Элис и Даниэля Столлмана. Возвращение на Манхэттен было чем-то большим, нежели сменой адреса. Это было прощание со спокойным укладом жизни и переосмысление себя в новом, независимом ключе.

\enquote{Думаю, моему политическому пробуждению поспособствовал тот случай, когда я пришла в общественную библиотеку Квинса и смогла найти только одну книжку, посвящённую разводам, -- рассказывает Липпман, -- подобные темы жёстко контролировались католической церковью, по крайней мере, в Элмхерсте, где мы жили. Мне кажется, тогда у меня впервые открылись глаза на силы, контролирующие нашу жизнь}.

Когда Элис вернулась в Верхний Вест-Сайд Манхэттена, район своего детства, её потрясло то, как сильно здесь всё изменилось за прошедшие 15 лет. Бешеный послевоенный спрос на жильё превратил район в поле ожесточённых политических баталий. На одной стороне были бизнесмены-застройщики и заинтересованные чиновники, которые хотели чуть ли не полностью перестроить район, превратив его в крупный жилой массив для \enquote{белых воротничков}. Им противостояла местная ирландская и пуэрториканская беднота, которая не хотела расставаться со своим дешёвым жильём.

Поначалу Липпман не знала, какую сторону выбрать. Как новой жительнице района, ей нравилась идея о новых домах с большим количеством просторных квартир. Но в экономическом плане Элис была куда ближе к местной бедноте -- минимальный доход матери-одиночки не позволил бы ей соседствовать с офисными работниками и служащими. Все планы развития районов ориентировались на состоятельных жителей, и это возмутило Липпман. Она принялась искать способы борьбы с политической машиной, которая хотела превратить её район в близнеца Верхнего Ист-Сайда.

Но сначала надо было найти детский сад для Ричарда. Придя в местный садик для бедных семей, Элис была потрясена условиями, в которых находились дети. \enquote{Я запомнила запах скисшего молока, тёмные коридоры и крайне скудное оснащение. А ведь мне доводилось работать воспитательницей в частных детсадах. Это просто небо и земля. Меня это расстроило и толкнуло к действиям}.

На дворе стоял 1958 год. Элис направилась в местную штаб-квартиру Демократической партии, полная решимости обратить внимание на ужасные условия жизни бедноты. Однако визит этот не принёс ничего, кроме разочарования. В комнате, где от курева можно было топор вешать, Липпман стала подозревать, что враждебное отношение к бедным слоям может быть вызвано коррумпированностью политиков. Поэтому она не стала больше ходить туда. Элис решила присоединиться к одному из многочисленных политических движений, нацеленных на кардинальные реформы в Демократической партии. Вместе с другими участниками движения, которое называлось Объединением демократических реформ имени Вудро Вильсона, Липпман начала ходить на городские заседания и общественные слушания, и добиваться большего участия в политической жизни.

\enquote{Своей главной целью мы видели борьбу с Таммани-холл -- влиятельной группой внутри Демократической партии Нью-Йорка, которая в то время состояла из Кармина де Сапио и его прихвостней.\endnote{Кармину де Сапио выпала сомнительная честь стать первым итало-американским боссом политической машины Таммани-холл, которая безраздельно господствовала в муниципальном управлении. Больше информации об этой фигуре в частности и послевоенной политике Нью-Йорка вообще можно найти здесь: John Davenport, \enquote{Skinning the Tiger: Carmine DeSapio and the End of the Tammany Era,} \textit{New York Affairs} (1975): 3:1.} Я стала общественным представителем в городском совете, и активно участвовала в создании более реалистичного плана преобразования района, который не сводился бы к его простой застройке элитным жильём}, -- рассказывает Липпман.

В 60-х годах это её занятие переросло в серьёзную политическую деятельность. К 1965 году Элис уже открыто и весьма активно поддерживала политиков вроде Вильяма Фитца Райана, конгрессмена от Демократической партии, который избрался благодаря сильной поддержке таких вот движений за партийные реформы, и который одним из первых высказался против войны во Вьетнаме.

Очень скоро Элис тоже стала ярым противником политики американского правительства в Индокитае. \enquote{Я была против войны во Вьетнаме с тех самых пор, как Кеннеди послал войска, -- говорит она, -- я читала сводки и репортажи о том, что там происходит. И я была твёрдо уверена, что это вторжение затянет нас в страшную трясину}.

Это противостояние американскому правительству проникло и в семью. В 1967 году Элис повторно вышла замуж, и её новый муж, Морис Липпман, будучи майором ВВС, подал в отставку, чтобы показать своё отношение к этой войне. Его сын Эндрю Липпман учился в МТИ, и был до конца учёбы освобождён от призыва. Но в случае разрастания конфликта отсрочку могли отменить, что в итоге и произошло. Наконец, угроза висела и над Ричардом, который хоть и был ещё слишком юн для службы, но вполне мог попасть туда в дальнейшем.

\enquote{Вьетнам был главной темой разговоров в нашем доме, -- вспоминает Элис, -- мы постоянно толковали о том, что будет, если война затянется, что нам и детям нужно будет делать, если их призовут. Мы все были против войны и призыва в армию. Мы были твёрдо убеждены, что это ужасно}.

У самого Ричарда война во Вьетнаме вызывала целую бурю эмоций, где главными чувствами были растерянность, страх и осознание своего бессилия перед политической системой. Столлман едва мог смириться с довольно мягкой и ограниченной авторитарностью частной школы, а от мыслей об армейской учебной части его вовсе бросало в дрожь. Он был уверен, что не сможет пройти через это и остаться в своём уме.

\enquote{Страх буквально опустошил меня, но у меня не было ни малейших идей о том, что мне делать, я даже на демонстрацию боялся пойти, -- вспоминает Столлман о том дне рождения 16 марта, когда ему вручили страшный билет во взрослую жизнь, -- можно было уехать в Канаду или Швецию, но у меня это в голове не укладывалось. Как мне решиться на такое? Я ничего не знал о самостоятельной жизни. В этом плане я был совершенно не уверен в себе}. Конечно, ему предоставили отсрочку для учёбы в вузе -- одну из последних, потом американское правительство перестало их давать -- но эти несколько лет пройдут быстро, и что делать тогда?

Ричард помнит, как его впечатлили высказывания членов семьи на эту тему. Вспоминает плакатики, что напечатал и распространил его отец, сравнив в них массовое убийство в Сонгми с преступлениями нацистов во Вторую Мировую. Этот поступок отца не на шутку взволновал Столлмана. \enquote{Я восхищался тем, что он сделал, -- говорит Ричард, -- но сам я и представить не мог, что делать. Я боялся, что безжалостная система призыва уничтожит меня}.

По большей части, Столлмана отталкивал стиль и цели основной массы антивоенного движения. Подобно другим участникам Колумбийской программы естественнонаучных достижений, он видел в демонстрациях зрелищный отвлекающий манёвр. \endnote{Чесс, сокурсник Ричарда, называл протесты \enquote{фоновым шумом}. Он говорит: \enquote{Мы все интересовались политикой, но Колумбийская программа была намного важнее. Мы бы никогда не ушли с занятий ради демонстрации}}. В конце концов, как рассказывает Столлман, хаотичные антивоенные силы перестали отличаться от хаотичных сил подростковых субкультур. Вместо того, чтобы увлекаться Битлами, его ровесницы фанатели от политических активистов вроде Эбби Хоффмана и Джерри Рубина. Для подростка, который горячо хотел нормальных отношений с ровесниками, лозунг \enquote{занимайтесь любовью, а не войной} звучал как издёвка. Столлман нисколько не хотел воевать, но и любовью заниматься его никто не звал.

\enquote{Мне не очень нравилась контркультура, -- рассказывает Столлман, -- не нравилась популярная музыка, не нравилась мода на наркотики. Наркотиков я вообще боялся. И особенно я не любил антиинтеллектуализм, не любил предвзято негативное отношение к технологиям. В конце концов, я любил компьютеры. Также я не любил бездумную американофобию, с которой часто сталкивался. Есть люди, которые мыслят настолько примитивно, что если выступают против Вьетнамской войны, то обязательно поддерживают северных вьетнамцев. По-моему, они неспособны понять, что дело может быть несколько сложнее}.

Такие откровения выделяют ключевую для политического созревания Столлмана черту -- прямую зависимость политической активности от уверенности в тех или иных вещах. К 1970 году он уже наработал твёрдые знания в некоторых областях за пределами точных наук. Тем не менее, для анализа антивоенного движения и его крайностей Ричард использовал чистую математическую логику, и в результате пришёл к системе взглядов, которая его устроила. Хотя Столлман был против войны во Вьетнаме, он не нашёл причин отказываться от войны как средства защиты свободы или преодоления несправедливости.

В 80-х годах уже обретший уверенность Столлман участвовал в вашингтонских массовых акциях за право на аборты. По его словам, этим он старался заглушить своё теперешнее недовольство тогдашней своей гражданской пассивностью.

В 1970 году Ричард отправился в Гарвард, оставив дома долгие кухонные разговоры о политике и Вьетнамской войне. Сейчас он описывает свой переезд из манхэттенской квартиры матери в общежитие в Кембридже одним словом: бегство. В Гарварде он мог проводить сколько угодно времени в покое, просто уйдя в свою комнату. Сверстники Столлмана и не подозревали, каким ветром перемен и свободы был для него этот переезд.

\enquote{В Гарварде он выглядел очень несчастным на первых порах, -- вспоминает Дэн Чесс, сокурсник Столлмана по Колумбийской программе, который тоже поступил в Гарвард, -- можно было с полным правом сказать, что отношения с людьми были для него очень трудным делом, а в Гарварде не было ни малейшей возможности их избежать. Это место требует активной социальной жизни}.

Чтобы легче адаптироваться, Ричард налёг на свои козыри: математику и точные науки. Вместе с большей частью учеников Колумбийской программы он без особого труда прошёл квалификационный экзамен для Math 55 -- усиленного математического курса, обросшего легендами и жуткими историями, который ещё называли \enquote{казармой} и \enquote{концентрационным лагерем} для новеньких математиков Гарварда. Внутри группы выпускники \enquote{колумбийцы} сформировали прочное сообщество. \enquote{Мы были математической бандой, -- смеётся Чесс, -- по сравнению с Колумбией, экзамены Гарварда были развлечением}.

Но такую гордость ещё нужно было обосновать, пройдя через Math 55, который давал четырёхлетнюю программу за 2 семестра. Это был выбор настоящих маньяков. \enquote{Курс был что надо, -- делится впечатлениями Дэвид Харбатер, член \enquote{математической банды}, а ныне -- известный своими работами профессор математики Пенсильванского Университета, -- думаю, можно уверенно заявить, что такого мощного и продвинутого курса для новоявленных студентов нигде и никогда не было. Чтобы вы приблизительно понимали, о чём идёт речь: ко второму семестру мы уже вовсю работали с дифференциальной геометрией на банаховых многообразиях. Это полный отвал башки, потому что обычно с банаховыми многообразиями люди начинают знакомиться на втором году аспирантуры}.

В начале курса Math 55 группа насчитывала 75 студентов, но до конца второго семестра добрались лишь 20 человек. Как говорит Харбатер, только половина оставшейся группы хорошо понимали, чем они занимаются, из них 8 впоследствии стали профессорами математики, один стал преподавать физику.

\enquote{И ещё один -- был Ричард Столлман}, -- говорит Харбатер.

По словам Сета Брайдбарта, который также прошёл курс Math 55, Столлман выделялся даже на фоне этой двадцатки.

\enquote{Он не искал лёгких путей, -- рассказывает Сет, -- в математике есть общепринятый метод, который все используют неправильно. По сути, это злоупотребление формализмом. Вам нужно определить функцию чего-то там, и вот что вы делаете: задаёте функцию, а потом доказываете, что она строго определена. Ричард сделал так только раз, а потом делал наоборот -- определял соотношение и доказывал, что это функция. То есть, для нас всех это было \enquote{наоборот}, а на самом деле это и было правильное использование метода. В этом был весь Столлман}.

Именно на курсе Math 55 у Ричарда стала складываться репутация гения. Брайдбарт сразу согласился с его превосходством, но Чесс, тоже будучи сильнейшим математиком, продолжал состязаться с ним за звание лучшего математика группы, и осознал гениальность Столлмана лишь в следующем году. \enquote{Это было на матанализе, -- вспоминает Чесс, ныне профессор математики Хантерского колледжа, -- мы работали над доказательством в области комплекснозначных функций, и Ричард додумался до идеи, которая основывалась на аналогии с вариационным исчислением. Тогда я впервые увидел, как он для решения задачи может отыскать очень эффективный и совершенно неожиданный путь}.

Для Чесса это был переломный момент. Он понял, что есть уровни знания и понимания, которые ему недоступны, хотя поначалу таковыми не выглядят. Ты просто натыкаешься на прозрачную стену, как птица, которая бьётся в окно.

\enquote{В этом суть математики, -- говорит Чесс, -- вам не нужно быть математическим гением, чтобы распознать математического гения. Я мог сказать, что был где-то возле того, но в то же время я понимал, что я всё-таки не гений. Если бы Ричард выбрал стезю математика, он стал бы великим учёным мирового уровня}.\endnote{Столлман, впрочем, в этом сомневается. \enquote{Одной из причин, по которой я ушёл в программирование из математики и физики, было то, что я никогда не учился открывать и создавать в них что-то новое. Я учился только использовать то, что другие уже открыли и создали до меня. В программировании же я мог каждый день создавать что-то новое и полезное}.}

Блестящий успех Столлмана в учёбе уравновешивался беспросветной неудачей в социальной жизни. Даже когда другие члены \enquote{математической банды} кооперировались для решения заданий Math 55, Ричард работал в одиночку. То же относилось и к его быту. В заявлении на общежитие Столлман выразился предельно ясно: \enquote{Предпочитаю невидимого, неслышимого, неосязаемого соседа по комнате}. Чиновники Гарварда проявили редкую чуткость к такому пожеланию, и весь первый курс Ричард прожил в комнате один.

Брайдбарт был единственным членом \enquote{математической банды}, который на первом курсе жил в том же общежитии, что и Столлман. Он говорит, что Ричард медленно, но верно учился общению с другими студентами. Он вспоминает, как соседи по общежитию, впечатлённые интеллектом Столлмана, начали приглашать его на тусовки в столовую или комнаты, где дискутировали обо всём на свете.

\enquote{У нас были обычные такие дебаты о решении мировых проблем или о последствиях какого угодно явления, -- рассказывает Брайдбарт, -- скажем, кто-то изобретает сыворотку бессмертия. Что тогда вы будете делать? Какие политические последствия это повлечёт? Если вы раздадите её всем подряд, мир быстро переполнится людьми и погибнет. Если вы начнёте раздавать её избирательно, это сразу разделит человечество на высший и низший классы. Ричард лучше других умел анализировать ситуации и предвидеть тонкие, неочевидные последствия решений}.

Столлман хорошо помнит эти дискуссии. \enquote{Я всегда был за бессмертие, -- говорит он, -- как мы ещё сможем узнать, на что будет похож мир через 200 лет?} Заинтересовавшись, Ричард стал расспрашивать знакомых, согласятся ли они стать бессмертными, если им предложат. \enquote{Меня поразило, что большинство людей считали бессмертие чем-то плохим}. Они говорили, что смерть это неплохо, потому что нет смысла жить в старости и дряхлости, а старение -- тоже хорошо, потому что готовит человека к смерти. И даже не видели в этом логическом порочном круге ничего странного.

В общем, у Столлмана сложилась репутация первоклассного математика и сильнейшего дебатёра, причём без каких-либо усилий с его стороны. Напротив, он всячески избегал откровенно состязательных мероприятий, на которых мог бы проявить себя во всём блеске. Брайдбарт вспоминает, как на исходе первого курса Ричард уклонился от участия в Патнемском тестировании -- престижном конкурсном экзамене для студентов-математиков США и Канады. Патнем -- не только отличный способ проверить уровень своих знаний, но и реальный шанс получить работу в лучших вузах и научных центрах. По кампусу ходили слухи, что набравшему самое большое количество очков гарантирована стипендия в любом вузе страны, включая Гарвард.

Но пройти этот тест было непросто -- как и Math 55, Патнем был ориентирован на лучших из лучших в математике. Экзамен состоял из 2 частей и длился 6 часов, и даже такие ветераны Колумбийской программы и Math 55, как Брайдбарт, описывают Патнем как труднейшее математическое испытание в жизни. \enquote{Чтобы вы понимали, насколько безумно сложен этот тест, -- говорит Брайдбарт, -- приведу статистику: максимальный результат равнялся 120 баллам, а я в первый год набрал около 30 баллов, и этот показатель был достаточно хорошим, чтобы я занял 101 место по стране}.

Когда Столлман отказался пройти Патнем, это удивило всех. Брайдбарт рассказывает, как за обедом они с другими студентами насели на него, допытываясь, почему он не стал проходить тест. \enquote{Ричард ответил, что боялся сплоховать}, -- вспоминает Брайдбарт. Но когда они быстро набросали по памяти несколько задач Патнема, Столлман быстро решил их все. \enquote{Такое ощущение, что под \enquote{сплоховать} он имел в виду второе место или что-то вроде этого}, -- смеётся Брайдбарт.

Сам Ричард описывает этот эпизод немного иначе. \enquote{Да, я решил одну или две задачи из тех, что они набросали, но я точно помню, что решил не все}, -- говорит он. Но Столлман подтверждает слова Брайдбарта касательно причины отказа -- он действительно боялся этого теста. Ему было нетрудно поправлять ошибки однокурсников и преподавателей, но когда речь заходила о жёстком соревновании на звание самого умного и способного, Ричард испытывал страх вперемешку с отвращением. И если соревнования можно избежать, то почему бы не воспользоваться этой возможностью?

\enquote{Ровно по этой же причине я никогда не любил шахматы, -- объясняет Столлман, -- всякий раз, как я принимаюсь за игру, я начинаю бояться, что ошибусь и проиграю, и из-за этого я действительно делаю глупые ошибки и проигрываю. Получается самоисполняющееся пророчество, основанное на страхе}. Он избегает этой проблемы, просто не играя в шахматы.

Возможно, именно такие страхи побудили Столлмана отказаться от карьеры математика, но это спорный вопрос. К концу первого курса у Столлмана отчётливо проявились сторонние увлечения, в частности -- программирование, которое из потаённого влечения школьной поры переросло в явную страсть. Студенты-математики часто отстранялись от зубодробительной учёбы, погружаясь в другие области вроде истории или искусств. Ричард же спасался в лаборатории информатики.

Когда Столлман распробовал программирование на настоящем компьютере в Нью-Йоркском научном центре IBM, он уже не мог забыть об этом, ему хотелось программировать ещё и ещё. \enquote{К концу первого курса в Гарварде я набрался смелости и зашёл в компьютерную лабораторию посмотреть, что у них там было интересного. Я нашёл массу всяких руководств и спросил, найдутся ли у них копии или дополнительные экземпляры, которые я мог бы взять себе}. Получив руководства, Ричард принялся штудировать их, изучая спецификации различных компьютеров.

Однажды вскоре после этого Столлман услышал о специальной лаборатории возле Массачусетского технологического института. Она располагалась на 9 этаже Техносквера -- коммерческого здания МТИ, стоящего через дорогу от кампуса. По слухам, лаборатория занималась передовыми исследованиями в области искусственного интеллекта, и была напичкана новейшими компьютерами и программным обеспечением.

Крайне заинтригованный Столлман решил туда наведаться.

Далеко ехать не пришлось -- всего 10 минут поездом и 2 мили пешком, но этого хватило, чтобы попасть в совершенно другой мир. МТИ представлял собой запутанный лабиринт соединённых между собой зданий, в противовес просторному кампусу Гарварда в классическом загородном стиле. Отличались и студенты -- в МТИ словно собрались безалаберные гики со всей страны, тогда как Гарвард был полон вымуштрованных щёголей, добивающихся политического влияния.

Ощущение разительного контраста получило своё продолжение и в Лаборатории ИИ. Ничего общего с компьютерными лабораториями Гарварда: ни вахтёра на входе, ни списков очерёдности доступа к терминалам, ни музейного правила \enquote{смотри, но не трогай}. Вместо этого Ричард увидел множество свободных терминалов и роботизированных рук, которые, видимо, использовались в каком-то эксперименте с ИИ. Встретив сотрудника Лаборатории, Столлман спросил, есть ли у них какая-нибудь запасная документация, которую им не жаль было бы одолжить любознательному студенту. \enquote{Кое-что у них было, но, по большей части, они ничего не документировали, -- рассказывает Столлман, -- они, по сути, были хакерами, и делали массу вещей, не тратя время на документацию}.

Ричард получил там кое-что получше всяких руководств, а именно -- работу. Ему поручили написать эмулятор PDP-11, который запускался бы на PDP-10. На следующей неделе Столлман вернулся в Лабораторию ИИ, уселся за первый попавшийся свободный терминал, и начал писать код.

Вспоминая об этом, он не видит ничего странного в том, что Лаборатория наняла на работу непонятно кого. \enquote{Тогда это было в порядке вещей. Да и сейчас это вполне нормально, почему бы и нет? Я точно так же найму первого встречного, если пойму, что он хорош в деле. Унылые бюрократы своими тягомотными процедурами убивают всю эффективность сотрудничества, когда ты встречаешь нужного человека, и в тот же час он сидит за компьютером и пишет код}.

Такой унылой тягомотной бюрократии Столлман вдоволь наелся в компьютерной лаборатории Гарварда. Там доступ к терминалам распределялся согласно академической иерархии. Как и всякому студенту, Ричарду порой приходилось часами ждать своей очереди, в то время как многие терминалы простаивали свободными в запертых кабинетах преподавателей, и это выглядело совершенно неразумным. Столлман продолжал время от времени наведываться в компьютерные кабинеты Гарварда, но эгалитарная атмосфера Лаборатории нравилась ему намного больше. \enquote{Это был глоток свежего воздуха, -- говорит он, -- в Лаборатории ИИ людей больше заботила работа, а не возня с иерархией и званиями}.

Столлман быстро понял, что принцип Лаборатории, гласящий \enquote{кто первый пришёл -- того и терминал} сложился благодаря влиянию группы идейных работников, во многом ещё со времён проекта MAC -- финансируемой Минобороны исследовательской программы по созданию первых операционных систем разделения времени. Многие работники уже были легендами компьютерного мира. Например, Ричард Гринблатт, штатный эксперт по LISP и автор шахматной программы Mac Hack, которая в своё время отправила в утиль риторику известного критика ИИ Хьюберта Дрейфуса. Или Джеральд Сассмен, создатель блочной ИИ-программы HACKER. Ну и, конечно же, Билл Госпер, штатный гений математики, который в то время с головой ушёл в компьютерную игру LIFE и связанную с ней философию.\endnote{Steven Levy, \textit{Hackers} (Penguin USA, 1984): 144.\\Леви на протяжении нескольких страниц рассказывает об этом увлечении Госпера математической игрой LIFE, созданной британским математиком Джоном Конвеем. Я от всей души советую его книгу как дополнение к этой, может быть даже -- как обязательное дополнение}.

Члены этой дружной группы называли себя \enquote{хакерами}. Со временем они стали называть хакером и Столлмана, приобщив его к идеалам \enquote{хакерской этики}. Хакеры могли торчать за компьютером по 36 часов подряд, исследуя границы своих и компьютерных возможностей. Поэтому им нужен был постоянный доступ к свободным компьютерам и самая полная и полезная информация о них. Хакеры открыто говорили о том, как изменить мир с помощью компьютерных программ, и Ричард стал бессознательно разделять их презрение и неприязнь к любым преградам на этом благородном пути. Главными же преградами были плохие программы, бюрократия в науке и человеческий эгоизм.

Столлман выслушал местные предания о том, как хакеры творчески преодолевали всякие бюрократические препятствия, в частности -- всеми способами \enquote{высвобождали} заблокированные профессорами терминалы. Нет, здесь не процветало собственническое отношение к компьютерам, как в Гарварде. Здесь кто-нибудь мог заблокировать доступ просто по рассеянности, уходя домой вечером. Тогда хакеры спешили исправить положение и на следующий день высказать виновнику протест против такого неконструктивного поведения. Иногда приходилось \enquote{хакать замки} или проникать в запертый кабинет через фальшпотолок. Был случай, когда он обвалился вместе с незадачливым \enquote{хакером}. Столлман рассказывает, что однажды ему показали тележку с увесистым металлическим брусом, которой таранили дверь одного профессора.\endnote{Джеральд Сассмен отрицает это и говорит, что хакеры никогда не взламывали замков на дверях ради доступа к компьютерам}

Упрямство хакеров служило благой цели -- не позволяло эгоизму мешать эффективной работе Лаборатории. Хакеры не отвергали личные потребности людей, но настаивали на том, чтобы их удовлетворение не мешало работать остальным. К примеру, преподаватель мог сказать, что в его кабинете есть вещи, которые нужно защитить от кражи. На это хакеры отвечали: \enquote{Никто не возражает против того, что вы закроете свой кабинет, но будет очень недружелюбно с вашей стороны закрыть в своём кабинете терминал}.

Хотя научных работников было намного больше, чем хакеров, в Лаборатории ИИ господствовала хакерская этика. Хакерами были работники и студенты, которые возились с оборудованием и программами, а уж они-то были жизненно важны для Лаборатории. Поэтому хакеры отказывались беспрекословно подчиняться. Они тратили массу времени на свои личные проекты и всякие улучшения, о которых просили пользователи, но нередко случалось так, что личные проекты хакеров выливались в усовершенствования рабочих компьютеров и программ. Их мышление было очень похоже на подростковое: заниматься чем-то просто потому, что это интересно и весело.

Ярче всего эта задорная мотивация отражалась в операционной системе для мейнфрейма PDP-10, которую разработали тут же, в Лаборатории на замену CTSS -- системе, оставшейся от проекта MAC. Аббревиатура CTSS расшифровывалась как \enquote{совместимая система разделения времени}, а система, написанная хакерами, получила название ITS -- \enquote{несовместимая система разделения времени}. CTSS не нравилась хакерам из-за своей архитектуры -- возможности модификации и расширения были сильно ограничены. ITS была, по сути, молчаливым протестом против таких ограничений. Местные хакерские предания говорили и о политических мотивах создания ITS. Дело в том, что ITS разработали конкретно под PDP-6, тогда как CTSS предназначалась для IBM 7094. Начальство Лаборатории осознанно позволило хакерам создать систему для PDP-6, да ещё и такую систему, которой нормально пользоваться могли лишь сами хакеры. Это был неплохой ход конём -- PDP-6 Лаборатория ИИ делила с другими отделами, но после создания ITS они пользовались этой машиной всё реже и реже, пока и фактически, и формально PDP-6 не перешёл в полное распоряжение Лаборатории. Благодаря ITS и PDP-6 в единоличном владении, Лаборатория перестала зависеть от проекта MAC ещё до прибытия Столлмана.\endnote{\textit{Ibid.}}

К 1971 году ITS перевели на новую машину -- PDP-10. Она была совместима с PDP-6, которую оставили для специальных и личных нужд. В этой PDP-10 было очень много памяти по тем временам -- больше 1 мегабайта, и в конце 70-х годов её объём удвоили. В рамках проекта MAC купили ещё 2 PDP-10, их установили также на 9 этаже здания, и на обоих компьютерах вскоре воцарилась ITS. Хакеры, которые занимались оборудованием, разработали и встроили в эти машины механизм страничной организации виртуальной памяти, которого не было в стандартном PDP-10. \endnote{Прошу прощения за этот беглый обзор истории системы ITS, которую многие хакеры считали чистым воплощением хакерской идеологии. Больше информации о политическом значении этой системы можно найти здесь: Simson Garfinkel, \textit{Architects of the Information Society: Thirty-Five Years of the Laboratory for Computer Science at MIT} (MIT Press, 1999)}.

Будучи учеником хакеров и впитывая их идеалы, Столлман буквально влюбился в ITS. Эта система имела ряд недоступных для не-хакеров возможностей, которых не было ни в одной коммерческой системе тех лет: многозадачность, отладка любой программы в режиме реального времени, редактирование в полноэкранном режиме.

\enquote{В ITS был встроен очень изящный механизм, позволяющий одной программе исследовать другую, -- вспоминает Столлман, -- вы могли быстро и точно узнать полное состояние любой программы без кучи грязных и утомительных трюков}. Это было удобно не только для отладки, но и просто для управления процессами.

Ещё одна любимая хакерами функция -- атомарная заморозка любого процесса. В других операционных системах подобные функции могли остановить процесс прямо посреди системного вызова или какой-нибудь другой инструкции, когда внутреннее состояние процесса остаётся неизвестным для пользователя. В ITS остановка выполнялась гарантированно между инструкциями, что делало пошаговый анализ работы программы очень надёжным и эффективным.

Вот как это описывает Столлман: \enquote{Если вы отдавали команду остановить процесс, он останавливался, во-первых, только в пользовательском режиме, а во-вторых -- только в тот момент, когда выполнение одной инструкции завершилось, а следующей -- ещё не началось. Если вы приказывали процессу продолжить работу, он продолжал работать правильно и предсказуемо. Если вы меняли что-то в остановленном процессе, запускали его дальше, а потом снова останавливали и возвращали всё обратно -- всё действительно возвращалось обратно и работало как ни в чём ни бывало. Полная согласованность и никаких скрытых сущностей}.

Начиная с сентября 1971 года, хакерство в Лаборатории ИИ стало постоянной частью недельного расписания Столлмана. С воскресенья по пятницу Ричард был в Гарварде, но уже вечером пятницы он отправлялся в МТИ. После нескольких часов работы за компьютером хакеры вспоминали о том, что неплохо бы поесть. Они прыгали в потрёпанное авто и ехали через Гарвардский мост в Бостон, где колесили по ночному городу в поисках китайской еды. В это время компания обсуждала всё на свете, начиная операционными системами, и заканчивая внутренней логикой китайского языка. Поужинав, они возвращались в Лабораторию, где копались в компьютерах и программах до глубокой ночи, и часа в 3-4 утра снова отправлялись за едой. Спать ложились только на рассвете.

Ричард иногда возвращался в Гарвард аж в воскресенье, но чаще -- к вечеру субботы, после того, как отсыпался на диване, ещё немного возился с компьютерами и обедал китайской едой. Эти китайские блюда были не только вкусными, но и сытными, чего нельзя было сказать о питании в столовой Гарварда, где только раз в день Столлман мог нормально поесть (во время завтрака он обычно ещё спал).

После многих лет жизни изгоем в школе времяпровождение с людьми, которые так же любили компьютеры, научную фантастику и китайскую еду, буквально пьянило Ричарда. \enquote{Я помню восходы солнца над кварталами, сквозь которые мы ехали на машине из Чайна-тауна, -- ностальгирует Столлман, -- заря -- это очень красивое зрелище ещё и потому, что раннее утро -- самое спокойное время суток. В такой момент хорошо идти домой под пение птиц или ложиться спать, когда душа полна спокойного, нежного удовлетворения от ночной работы}\endnote{Источник: Richard Stallman, \enquote{RMS lecture at KTH (Sweden)}, (October 30, 1986), \url{http://www.gnu.org/philosophy/stallman-kth.html}}.

Чем дольше Столлман тусовался с хакерами, тем сильнее он проникался их мировоззрением. На его преданность идее личной свободы стали накладываться соображения ответственности перед обществом. Ричард в числе первых протестовал против нарушений коллективных норм и правил. В первое время только он открывал запертые двери кабинетов с терминалами. Как настоящий хакер, Столлман старался возвести эти занятия в ранг искусства. Одно из оригинальных хакерских приспособлений, которое обычно приписывается Гринблатту, позволяло без шума и пыли открыть почти любую запертую дверь. Это была жёсткая проволока, изогнутая под прямым углом в нескольких местах. На один конец проволоки прикреплялась клейкая лента. Хакер просовывал проволоку под дверь и ворочал ею так, чтобы лента приклеилась к дверной ручке, после чего оставалось поворотом потянуть её вниз.

Столлман попробовал эту штуку в деле и нашёл её очень неудобной. Приклеить ленту к ручке было непросто, как и поворачивать проволоку, чтобы тянуть ручку вниз. Ричард подумал о другом способе: отодвинуть ячейки фальшпотолка и пролезть в запертый кабинет. Но и здесь были трудности. Например, в кабинете могло не оказаться стола в пределах досягаемости, чтобы безопасно спрыгнуть. Да и ползание за фальшпотолком покрывало хакера налётом стекловолокна, от которого всё дико чесалось. Можно ли как-то избежать этих неприятностей? Столлман решил совместить два способа: вместо того, чтобы совать проволоку под дверь, можно было отодвинуть ячейку фальшпотолка прямо у двери, перегнуться через стену и орудовать проволокой сверху.

Экспериментальную проверку нового способа Ричард взял на себя. Вместо проволоки он использовал длинную магнитную ленту в форме буквы U, к концу которой прикрепил короткую клейкую ленту. Перегнувшись через стену и манипулируя лентой, он быстро приклеил короткую ленту к ручке, после чего потянул за один конец длинной ленты, и дверь открылась. Так Столлман добавил новый приём в хакерский арсенал способов \enquote{освобождения терминалов}. Его недостаток был лишь в том, что дверь иногда нужно было немного пнуть, чтобы она открылась.

Подобное поведение говорило о растущей готовности Столлмана отстаивать свои идеи не только на словах, но и на деле. Дух Лаборатории, который отдавал предпочтение действию перед словами, достаточно воодушевил Ричарда, чтобы вытащить его из робкой пассивности подросткового периода. Взломать кабинет, чтобы освободить терминал -- это, конечно, не то же самое, что принять участие в акции протеста, но у этого действа было неоспоримое преимущество: проблема решалась здесь и сейчас собственными руками. Это было прекрасное воплощение политического принципа прямого действия.

В последние годы учёбы в Гарварде Столлман и там начал использовать всякие изощрённые и беспардонные приёмы в духе Лаборатории.

\enquote{Он вам рассказывал о змее? -- спросила Элис Липпман на одном из интервью, -- Ричард и его товарищи выдвинули змею в кандидаты на студенческих выборах. И она вроде даже набрала немало голосов}.

Змея была кандидатом на выборах в Карриер-Хаус, общежитии Столлмана. Она действительно оказалась популярным кандидатом, потому что никто не знал, что это змея -- её владелец дал ей своё имя и фамилию. \enquote{Люди думали, что голосуют за реального студента, -- рассказывает Столлман, -- мы ещё наделали предвыборных плакатов, в которых говорилось, что кандидат \enquote{не витает в облаках, а прочно держится за землю}, что это \enquote{свободный самовыдвиженец}, потому что змея вылезла из вентиляции за несколько недель до этого}.

Также они выдвинули кандидатом 3-летнего сына управляющего общежитием. \enquote{Его программа включала выход на пенсию в 7 лет}, -- вспоминает Ричард. Однако в Гарварде эти розыгрыши не принимали драматичного оборота. В МТИ же выдвинутый студентами фальшивый кандидат -- кот Вудсток -- скорее всего, даже победил в выборах, обойдя всех кандидатов-людей. \enquote{Официально не говорилось о том, сколько людей проголосовало за Вудстока, такие бюллетени посчитали испорченными, и их аномально большое количество наталкивает на подозрения, что Вудсток всё-таки победил. Через пару лет Вудстока сбила машина. До сих пор неизвестно, работал ли водитель на администрацию МТИ}. Ричард говорит, что не имел никакого отношения к проделке с Вудстоком, но восхищался ею. \endnote{Когда эта книга была в последней стадии редактирования, Столлман написал электронное письмо, в котором признавал влияние Гарварда на формирование его мировоззрения. \enquote{На первом курсе мы проходили историю восстания против династии Цинь, против того жестокого правителя, что сжёг все книги и был похоронен с Терракотовой армией. Конечно, история с котом не идёт ни в какое сравнение, но всё-таки она очень занимательна}}.

Основная часть политической активности Столлмана приходилась на Лабораторию ИИ. В 70-е годы там развернулась нешуточная борьба между хакерами, сотрудниками факультета и должностными лицами. Хакерский дизайн ITS шёл вразрез с потребностями научных сотрудников и администраторов, потому что не предусматривал никакой системы прав доступа. Любой мог сделать на компьютере что угодно, например, дать команду на выключение, и любой же мог отменить её. В середине 70-х многие преподаватели, особенно из тех, что недавно пришли в Лабораторию, начали требовать систему разграничения доступа к файлам, чтобы обезопасить свои данные. Другие операционные системы тех времён имели такую функциональность, и многие сотрудники факультета привыкли к этому чувству защищённости. Но Лаборатория ИИ по настоянию Столлмана и других хакеров продолжала оставаться территорией, свободной от всякой защищённости.

Ричард выдвигал и этические, и практические аргументы против внедрения систем безопасности. В этическом плане Столлман апеллировал к интеллектуальной традиции коллектива Лаборатории, которая основывалась на открытости и доверии. Практические доводы упирали на глубинное устройство ITS, которое было заточено под совместное использование и прозрачный доступ ко всем программам и данным. Любые попытки внести сюда толику безопасности и разграничений потребовали бы полностью переделать систему. Чтобы исчерпать все возможности модификации системы, Ричард сделал так, чтобы единственный всё ещё свободный файловый дескриптор начал хранить метку о пользователе, который последним редактировал файл. Таким образом, не осталось никаких возможностей добавить в файловую систему метки безопасности, и в то же время изменение Ричарда оказалось настолько полезным, что никто не стал требовать его удаления.

\enquote{Хакеры, которые создали ITS, считали, что система разграничения прав доступа используется самоназванными администраторами для ущемления других пользователей, -- объяснял позже Столлман, -- они не хотели, чтобы кто-то имел над ними власть, поэтому даже не брались за реализацию такой системы. Благодаря этому всякий раз, как в ITS что-то ломалось, вы могли это без труда починить, потому что контроль доступа не мешал вам}\endnote{Источник: Richard Stallman (1986).}

Таким образом хакеры отстояли status quo Лаборатории. Но в других лабораториях соображения безопасности одержали верх. В 1977 году в расположенной рядом Лаборатории информатики МТИ внедрили парольную систему доступа. Столлман решил исправить это этическое недоразумение, и написал программу для дешифровки пользовательских паролей. Затем он начал рассылать по электронной почте примерно такие сообщения:

\begin{quote}
Я смотрю, вы используете пароль \enquote{starfish}. Предлагаю вам изменить его на пароль в виде возврата каретки, который использую я. Его проще и быстрее набрать, и он не противоречит принятой концепции безопасности.
\end{quote}

Пользователи, которые выбирали возврат каретки в качестве пароля -- то есть, простое нажатие на соответствующую клавишу вместо уникальной текстовой строки -- возвращали всеобщий доступ к своим аккаунтам, который исчез при внедрении парольной системы доступа. В этом был смысл акции: использовать вырожденную форму пароля, чтобы высмеять саму концепцию использования паролей. Хакеры знали, что механизмы безопасности в этих операционных системах были очень слабыми и никак не могли помешать реальным злоумышленникам. Тогда какой смысл закрывать доступ добропорядочным сотрудникам, которым понадобились какие-то данные?

Давая интервью для книги \textit{\enquote{Хакеры}} 1984 года, Столлман с гордостью заметил, что пятая часть всех сотрудников Лаборатории информатики приняла его предложение и перешла на пустой пароль.\endnote{Источник: Steven Levy, \textit{Hackers} (Penguin USA [paperback], 1984): 417.}

Нуль-парольная кампания Столлмана и хакерское сопротивление мерам безопасности будут побеждены в конечном итоге. Уже в начале 80-х годов на всех компьютерах МТИ, даже в Лаборатории ИИ, появились системы контроля доступа с паролями и прочими механизмами безопасности. Но сопротивление сыграло важную роль в философском и политическом созревании Ричарда. Этот период был переходной формой в эволюции робкого подростка, который боится что-то сделать даже когда речь идёт о его судьбе и жизни, в матёрого активиста, для которого критика, высмеивание и активное противодействие со склонением многих людей на свою сторону -- обычное дело.

В своей риторике против систем компьютерной безопасности Ричард во многом опирался на характерные черты своей юности: жажду знаний, отвращение к власти, досаду на предрассудки и скрытые правила, которые делали некоторых людей изгоями. С другой стороны, здесь ощущалось влияние и недавно приобретённых идей вроде ответственности перед обществом, человеческого доверия и хакерского принципа прямого действия. Если использовать терминологию программистов, нуль-парольная кампания была результатом работы Ричарда Столлмана версии 1.0 -- ещё далеко не законченной политической фигуры, но уже более-менее сформированной.

Сам же Ричард, вспоминая своё студенчество, не придаёт большого значения тем событиям. \enquote{В тот период многие люди разделяли мои взгляды, -- говорит он, -- в той же истории с пустым паролем немало людей откликнулось на мою инициативу. Мои действия не встречали серьёзного сопротивления и осуждения, так что не стоит считать их настоящей борьбой, в которой закаляется характер и мировоззрение}.

Однако Столлман отдаёт должное Лаборатории ИИ, считая, что именно она пробудила в нём дух сопротивления. Подростком он лишь наблюдал происходящее, не имея ни малейшего представления о том, как повлиять на события. В юношестве он уже стал высказываться по тем вопросам, в которых выработал твёрдую, уверенную позицию. \enquote{Я влился в коллектив Лаборатории, где процветало уважение к свободе личности, -- говорит Ричард, -- и мне не нужно было время, чтобы понять, насколько это хорошо. Мне нужно было время, чтобы понять, что это не данность, а один из вариантов решения моральной проблемы}.

Уверенность Столлмана в себе прокачивалась не только хакерством в Лаборатории ИИ. В начале первого курса Ричард записался в развлекательную группу танцев народов мира, которую организовали в Карриер-Хаус. Вообще-то у него даже мыслей не было туда идти -- он считал себя совершенно неспособным танцевать, но друг уговорил его, сказав: \enquote{Как ты можешь быть уверен, что неспособен на это, если даже не пробовал?}. И -- о, чудо! -- Ричард не только смог танцевать, но и получил от этого огромное удовольствие! Сомнительный эксперимент превратился в ещё одну страсть Столлмана, подобную хакерству и учёбе, также это был действенный способ познакомиться с девушками, хотя за всё время обучения он ни разу не ходил на свидания. Танцуя, Ричард уже не ощущал себя неловким 10-летним мальчиком, чья попытка поиграть в футбол окончилась полным провалом. Он ощущал себя уверенным в себе, ловким и живым. В начале 80-х Столлман пошёл дальше и присоединился к ансамблю народных танцев МТИ. Танцуя перед залом в традиционном костюме балканского крестьянина, он веселился от души и попутно тренировал в себе способность находиться на сцене  перед большим количеством людей, что потом пригодилось ему для публичных выступлений.

Хотя танцы и хакерство вряд ли повысили популярность Столлмана среди сверстников, они помогли ему избавиться от чувства отчуждённости, которое отравляло его жизнь до Гарварда. В 1977 году на научно-фантастическом конвенте он встретил Пуговичную Нэнси -- она делала очень стильные пуговицы с любой каллиграфической надписью по желанию. Восхищённый Столлман заказал пуговицу со словами \enquote{Развенчай бога}.

Для Столлмана этот лозунг имел несколько смысловых слоёв. Будучи атеистом с малых лет, Ричард словами \enquote{Развенчай бога} словно открыл \enquote{Второй фронт} в противостоянии с религиозностью. \enquote{Тогда все увлечённо спорили, существует ли бог, -- вспоминает Столлман, -- и это \enquote{Развенчай бога} представило вопрос в совершенно ином свете. Есть бог или нет -- не столь важно. Куда важнее -- зачем он нам нужен. Если бог настолько могуч, что смог создать мир, но при этом не делает ничего для его улучшения, то зачем нам поклоняться такому богу? Разве этого недостаточно, чтобы предать его суду?}

В то же время, \enquote{Развенчай бога} отсылал к Уотергейтскому скандалу 70-х годов, сравнивая Никсона с божеством-тираном. Уотергейт глубоко поразил Столлмана. С самого детства власть вызывала у него негодование и отвращение. В юношестве это отношение укрепилось под влиянием коллектива Лаборатории ИИ. Для хакеров Уотергейт был поистине шекспировской пьесой с ожесточённой борьбой за власть, от которой в жизни простых людей появлялась масса хлопот. Это была длинная притча о том, что произошло, когда люди продали свободу и открытость за безопасность и комфорт.

Полный воодушевления, Столлман открыто и гордо носил пуговицу. Достаточно любопытные для вопросов люди получали хорошо поставленный номер. \enquote{Меня зовут Иегова, -- вещал Ричард, -- у меня есть тайный план, как положить конец несправедливости и страданиям, но небесные правила безопасности запрещают мне рассказывать, что это за план и как он работает. Я вижу общую картину, а ты -- нет, я хорош и велик, потому что я тебе так сказал. Так что верь в меня и повинуйся без вопросов. Если ты откажешься -- значит, ты злой, и я внесу тебя в список своих врагов и сброшу в бездну, где Адская налоговая служба будет вечно проверять твои выплаты}.

Те, кто истолковывали этот номер в контексте Уотергейта, видели только половину смысла. Сам Ричард вложил в него ещё и то, что понимали, казалось, только его коллеги-хакеры. Меньше века прошло после предупредительных слов лорда Актона о том, что абсолютная власть развращает абсолютно, и американцы, казалось, забыли первую часть его очевидного изречения: всякая власть развращает сама по себе. Вместо того, чтобы указывать на многочисленные примеры мелкой коррупции, Столлман считал нужным возмущаться всей системой, которая на первое место ставила доверие к власти.

\enquote{Я понял, что бесполезно ловить мелкую рыбёшку. Если мы пришли за Никсоном, то почему бы теперь не прийти за Большим Братом? Я глубоко убеждён, что всякий, кто злоупотребляет властью, заслуживает того, чтобы у него эту власть отняли силой}.

\theendnotes
\setcounter{endnote}{0}
