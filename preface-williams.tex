%% Copyright (c) 2002, 2010 Sam Williams
%% Copyright (c) 2010 Richard M. Stallman
%% Permission is granted to copy, distribute and/or modify this
%% document under the terms of the GNU Free Documentation License,
%% Version 1.3 or any later version published by the Free Software
%% Foundation; with no Invariant Sections, no Front-Cover Texts, and
%% no Back-Cover Texts. A copy of the license is included in the
%% file called ``gfdl.tex''.

\chapter{Предисловие от Сэма Вильямса}

Этим летом исполнилось 10 лет той переписке по электронной почте,
что дала начало книге \enquote{Крестовый
поход Ричарда Столлмана за свободное программное обеспечение} и её
доработанной версии: \enquote{Ричард Столлман и революция свободного
программного обеспечения}, которую вы держите в руках.

Не стоит и говорить, что за эти 10 лет многое изменилось в этом мире.

Книгу эту я задумывал в эпоху триумфа Америки, и её суть была совсем
не созвучна победному маршу технокапиталистической модели.
Герой этой книги подобно Иеремии пытается обратить внимание
разработчиков ПО не на экономический потенциал компьютерных
программ, а на этические ценности и обязательства. Он старается донести
до людей, что уход от интеллектуальной и культурной свободы к
выгодному собственничеству дорого обойдётся в будущем.

Теперь же настроения в обществе прямо противоположны -- умы
захватили сомнения в Системе, и потому интересно посмотреть,
какие идеи не потеряли актуальности за эти 10 лет.

Я уже не так пристально слежу за состоянием дел в индустрии ПО,
как раньше, но одно очевидно даже для меня: люди легко отказываются
от личных прав и свобод ради того, чтобы пользоваться всякими
классными и удобными технологиями, и чтобы просто быть модными.

В этом я убедился ещё несколько лет назад, когда случилось то, что
можно назвать \enquote{iPod-эффектом}: музыкальная индустрия и простые
слушатели стали сходить с ума по плееру Apple iPod и сервису iTunes,
при их огромном количестве ограничений. Позже эта история повторилась
с планшетом Apple iPad и электронной книгой Amazon Kindle. Мода на
все эти гаджеты привела к тектоническим сдвигам в индустрии.

Ричарду Столлману, кстати, не понравилось, что я в предисловии как будто
рекламирую несвободные устройства, так что для баланса я скажу
вам про парочку интернет-сайтов, где подробно разбирают недостатки
этих и других продуктов:  \url{DefectiveByDesign.org} и \url{BadVista.org}.

Не имеет значения, о каком именно бренде идёт речь. Люди привыкли
ассоциировать бренды с неким качеством, и этот стереотип побороть
очень трудно. Даже несмотря на многие неудачи и провалы корпораций,
в том числе и экономические.

Десяток лет назад было обычным делом поучаствовать в конференции,
где старожилы программирования вроде Ричарда Столлмана говорили
об альтернативном будущем. И теперь я убедился, куда завела нас,
тогдашних новичков в журналистике, гордость владения новейшими
инструментами вроде Microsoft Word, PowerPoint, Internet Explorer. Все
эти продукты были удобными, заманчивыми пристанищами для тех, кто
работал на компьютере, и благодаря популярности они разрослись в
огромный город программной экосистемы персональных компьютеров.

Город этот похож на антиутопическую диктатуру, где движение, развитие,
жизнь строго регламентированы и возможны лишь в указанных сверху
местах. Неудивительно, что хакеры не прекращают критически ворчать
в адрес структурных недостатков Windows, диктаторского надзора Apple
над их магазином приложений, постоянно меняющегося определения \enquote{зла}
у Google. С каждым годом всё сильнее разрастается \enquote{цифровая реальность},
где пользователи служат гоббсовой корпократии. Наверное, потому что
свободное ПО подкидывает изрядное количество проблем, которые
заставляют пользователей скрежетать зубами.

Как бы то ни было, развитие софта с его бесконечным бегом на шаг
впереди пользовательских вкусов и потребностей всё сильнее толкает
разработчиков к наживе. Нельзя сказать, что хакерская этика умерла или
даже сильно ослабела. Я просто думаю, что вряд ли программист,
который добивается для заказчика высочайших показателей, будет
сидеть в своём Порше и размышлять о том, как развивались бы его
программы без корпоративных кандалов.

Впрочем, десятки и сотни миллионов человек пользуются многими
свободными программами, а какая-то их часть использует только
свободное ПО. Однако понятия вроде \enquote{программное обеспечение}
или \enquote{компьютер} уходят из потребительского лексикона. Нынешние
мобильные телефоны сравнимы с ноутбуками десятилетней давности,
но многие ли при покупке телефона думают о свободе комплектующих
или операционной системы? Единственное, что интересует большинство
-- приложения, которые можно установить, надёжность работы сети, и
самое важное -- тарифные планы.

Заставить современного потребителя взглянуть на софт и оборудование
не с точки зрения удобства и функциональности, а с точки зрения свободы
-- если не невозможно, то довольно сложно.

И вот на такой пессимистичной ноте можно задаться вопросом: зачем
тогда вообще нужна эта книга и кого она может заинтересовать?

Я могу предложить целых две причины.

Первая причина -- личная. В эпилоге книги рассказывается, что я и
Ричард расстались на не совсем тёплой ноте незадолго до публикации.
Виноват в этом, по большей части, я. Довольно плотно поработав с ним,
чтобы быть уверенным, что моя книга не противоречит ценностям
свободного ПО -- с гордостью, кстати, сообщаю, что это одна из первых
книг, что использует GNU Free Documentation License -- я резко оборвал
сотрудничество, когда настала пора редактирования и Ричард прислал
длиннющий список исправлений.

Хоть я умею виртуозно прятаться за своими принципами вроде
авторской независимости и журналистской объективности, я не стал
ходить к издателю и умолять его о дополнительном времени. Ведь
О'Рейли и так сделал мне огромную уступку -- позволил выбрать GFDL
в качестве лицензии. Наверное, он бы согласился и с огромным потоком
исправлений перед самой публикацией, но я не стал испытывать удачу.

По прошествии нескольких лет у меня появилось ещё одно оправдание
этому шагу: сама лицензия GFDL, которая позволяет любому читателю
редактировать книгу и распространять свой вариант. Как однажды
сказал Эрнест Хемингуэй: \enquote{Первый черновик -- всегда дерьмо}.
Если Столлман и другие хакеры сообщества свободного ПО считали
мою книгу первым черновиком, то что ж -- по крайней мере, я избавил
их от этой неблагодарной работы.

Теперь же я могу только одобрить множество правок Ричарда, чтобы
остаться верным себе. На самом деле, я их даже приветствую. Видя, что
активность Ричарда и не думает снижаться, я надеюсь только, что его
правки включат в себя множество документальных дополнений.

Прежде чем перейти ко второй причине, хочу отметить, что работа над
этой книгой имела приятный побочный эффект -- возобновление нашего
со Столлманом общения по электронной почте. Я снова могу получать
удовольствие от его острого как бритва стиля изложения.

Те, кто когда-либо дискутировал со Столлманом в переписке, найдут
показательным и забавным следующий случай. Рецензенты часто критиковали
главу, где описывается, как мы едем через Кихеи на Мауи (\enquote{Краткое
путешествие по аду хакера}) -- мол, она выбивается из общей канвы
и выглядит неуместной. Я сказал Ричарду, что он может вырезать эту
главу, но заметил, что она даёт неплохое метафорическое описание
его характера и вообще хакерского мышления.

К моему удивлению, Столлман со мной согласился. Его больше заботили
отдельные неточности в тексте. Например, я устами Ричарда выразил, что
ведущий нас местный житель \enquote{специально} ведёт нас такой дорогой,
что придало его поведению ноту злого умысла. К сожалению, я мог
положиться лишь на записи -- довольно ненадёжный источник. Из-за
этого я исказил формулировки Ричарда, сделав их более жёсткими.

Вопросы у Столлмана возникли и к слову \enquote{долбаный}. Опять же, я не
мог обратиться к аудиозаписи диктофона, но сказал Ричарду, что меня
тогда впечатлила его лексика, напомнив мне о его нью-йоркских корнях,
так что я уверен в использованном слове.

На следующий день пришло ответное письмо, которое заставило меня
перечитать предыдущее сообщение. Оказалось, что Столлман спорит
не с самим словом, а с той формой, которую я использовал в книге.

\enquote{Сомневаюсь, что я назвал бы чью-то ошибку \enquote{долбаной}, -- писал
мне Ричард, -- мне несвойственно так выражаться. Я бы скорее сказал:
\enquote{долбанутая ошибка}. Поэтому я думаю, что слова искажены}.

Шах и мат.

Вторая причина существования этой книги и чьего-либо интереса к
ней возвращает нас к началу предисловия -- разница между началом
века и его вторым десятилетием. Буду честен: мои взгляды сильно
поменялись после 11 сентября 2001 года, так что скоро после выхода
первой версии книги я почти совсем перестал следить за Столлманом
и движением за свободное ПО. Хотя основные тенденции и проблемы
индустрии ПО всё-таки достигали моего внимания, основная масса
событий скрылась от меня, подобно подводной части айсберга.

[РМС: Атаки 11 сентября 2001 года почти не упоминаются в книге,
 но я думаю, что стоит дать краткий комментарий на эту тему. Хотя
 многие думают и заявляют, что эти теракты \enquote{изменили всё}, на
 самом деле, они не изменили ничего, в том плане, что во власти
 всё так же сидят негодяи, которые ненавидят наши свободы. Но
 теперь они могут ссылаться на \enquote{террористов} всякий раз, когда
 хотят ввести законы, отбирающие наши права. Подробнее об
 этом можно почитать в политических заметках на моём сайте
 \url{stallman.org}.]

Это весьма плачевно, потому что перед моими глазами тот самый
XXI век, который я считал когда-то светлым будущим. Но сейчас
его уместно назвать веком бега на месте. Я объясню, почему.

Мы находимся в необычной точке исторического развития, где
возможность изменить мир, невзирая на весь цинизм и пессимизм,
действительно доступна каждому. Но в этой доступности каждому
и кроется подвох. Если в прошлом можно было добиться перемен,
завоевав расположение нескольких влиятельных фигур, то теперь
реформатор должен придать нужное направление целому полю
всевозможных идей, настроений и интересов самых разных людей.
То есть, эффективный реформатор должен обладать в наше время
титанической волей и упорством, готовностью десятилетиями
голосить в пустыне, а также массой знаний и творческими талантами,
чтобы создать идеи, способные обыграть Систему в её же игре.

И Ричард Столлман подходит под все эти критерии.

Кто-то может оплакивать насмерть забюрократизированное будущее,
в котором любая проблема требует собраний и заседаний комитетов
и подкомитетов, чтобы только наметить возможные пути её решения.
Я же вижу кое-что другое. Я вижу, как реальность настолько чутко
реагирует на личности и небольшие группы людей, что находится
самозваный актёр, дерзнувший испробовать эту чуткость для того,
чтобы изменить мир.

И если вы один из тех, кто надеется, что XXI век будет намного более
человечным, чем XX век, то -- добро пожаловать в битву. Эпиграф к
первой главе намекает на мою надежду, что эта книга станет эпосом
интернет-эры, построенным вокруг странной героической фигуры.

Так что я хочу закончить это предисловие так же, как закончил
эпилог -- предложением присоединиться к улучшению книги. Текст
лицензии GFDL расскажет вам о ваших правах отредактировать книгу
или даже создать производную версию. Если хотите, можете выслать
свои правки мне или Ричарду. Его контакты вы найдёте на сайте
фонда свободного ПО.

А теперь -- желаю удачи и удовольствия от чтения книги!

\vspace{0.5in}
\noindent Сэм Вильямс\\
\noindent Статен-Айленд, США
