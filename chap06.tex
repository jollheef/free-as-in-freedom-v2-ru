%% Copyright (c) 2002, 2010 Sam Williams
%% Copyright (c) 2010 Richard M. Stallman
%% Permission is granted to copy, distribute and/or modify this
%% document under the terms of the GNU Free Documentation License,
%% Version 1.3 or any later version published by the Free Software
%% Foundation; with no Invariant Sections, no Front-Cover Texts, and
%% no Back-Cover Texts. A copy of the license is included in the
%% file called ``gfdl.tex''.

\chapter{Коммуна Emacs}

Лаборатория ИИ в 70-х годах была особенным местом, в этом сходились все. Здесь проходили передовые исследования, здесь работали сильнейшие специалисты, так что в компьютерном мире Лаборатория была постоянно на слуху. А её хакерская культура и мятежный дух создавали вокруг неё ореол священного места. Только когда из Лаборатории ушли многие учёные и \enquote{рок-звёзды программирования}, хакеры ощутили всю мифологичность и эфемерность того мира, в котором они жили.

\enquote{Лаборатория была для нас чем-то вроде Эдема, -- рассказывает Столлман в статье \textit{Forbes} 1998 года, -- никому даже в голову не приходило отгородиться от других сотрудников вместо того, чтобы работать сообща}.\footnote{Josh McHugh, \enquote{For the Love of Hacking,} \textit{Forbes} (August 10, 1998), \url{http://www.forbes.com/forbes/1998/0810/6203094a.html}.}

Такие описания в духе мифологии подчёркивают важный факт: 9~этаж Техносквера был для многих хакеров не только рабочим местом, но и родным домом.

Слово \enquote{дом} использовал сам Ричард Столлман, а мы прекрасно знаем, насколько точен и осторожен он в высказываниях. Пройдя через \enquote{холодную войну} с собственными родителями, Ричард до сих пор считает, что до Карриер-Хауса, его Гарвардского общежития, родного дома у него просто не было. По его словам, в гарвардские годы его мучил только один страх -- оказаться исключённым. Я выразил сомнение, что у такого блестящего студента, как Столлман, был риск вылететь. Но Ричард напомнил мне о своих характерных проблемах с дисциплиной.

\enquote{В Гарварде очень ценят дисциплину, и если ты пропускаешь пары, тебя быстро попросят на выход}, -- сказал он.

После окончания Гарварда Столлман лишился права на общежитие, а желания возвращаться к родителям в Нью-Йорк у него никогда и не было. Так что он пошёл по дорожке, проторённой Гринблаттом, Госпером, Сассменом и многими другими хакерами -- поступил в аспирантуру МТИ, снял рядом комнату в Кембридже, и большую часть времени стал проводить в Лаборатории ИИ. В своей речи 1986 года Ричард так описал этот период:

\begin{quote}
Наверное, я немного больше других имею оснований сказать, что жил в Лаборатории, потому что каждый год-два я по разным причинам лишался жилья, и в общем счёте я прожил в Лаборатории несколько месяцев. И мне там всегда было очень комфортно, особенно в жару летом, потому что внутри было прохладно. Но вообще это было в порядке вещей, что люди ночевали в Лаборатории, хотя бы из-за бешеного энтузиазма, что владел тогда всеми нами. Хакер порой просто не мог остановиться и работал за компьютером до полного истощения, после чего отползал на ближайшую мягкую горизонтальную поверхность. Словом, очень непринуждённая, домашняя атмосфера.\footnote{Stallman (1986).}
\end{quote}

Но эта домашняя атмосфера порой создавала проблемы. В том, что некоторые считали домом, другие видели притон электронного опиума. В книге \enquote{Сила компьютера и мотивация человека} научный сотрудник МТИ Джозеф Вейзенбаум в резком тоне раскритиковал \enquote{компьютерный взрыв} -- так он назвал заселение хакерами компьютерных центров вроде Лаборатории ИИ. \enquote{Их мятая одежда, немытые волосы и небритые лица говорят о том, что они полностью забросили себя в пользу компьютеров, и не хотят видеть, к чему это может их привести, -- писал Вейзенбаум, -- эти компьютерные бичи живут только ради компьютеров}. \footnote{Joseph Weizenbaum, \textit{Computer Power and Human Reason: From Judgment to Calculation} (W. H. Freeman, 1976): 116.}

Спустя почти четверть века Столлман всё ещё выходит из себя, когда слышит выражение Вейзенбаума: \enquote{компьютерные бичи}. \enquote{Он хочет, чтобы мы все были всего лишь профессионалами -- делали работу ради денег, в означенное время вставали и уходили, выкинув из головы всё, что с нею связано, -- говорит Столлман так яростно, будто Вейзенбаум рядом и может его услышать, -- но то, что он считает нормальным порядком вещей, я считаю удручающей трагедией}.

Впрочем, жизнь хакера тоже не лишена трагедии. Сам Ричард утверждает, что его превращение из хакера-по-выходным в хакера-24/7 -- результат целой череды болезненных эпизодов юношества, от которых получалось спастись лишь в эйфории хакерства. Первой такой болью стало окончание Гарварда, оно резко меняло привычный, спокойный уклад жизни. Столлман поступил в аспирантуру МТИ на отделение физики, чтобы пойти по стопам великих Ричарда Фейнмана, Вильяма Шокли и Мюррея Гел-Манна, и чтобы не пришлось ездить 2 лишних мили до Лаборатории ИИ и новенького PDP-10. \enquote{Почти всё своё внимание я по-прежнему уделял программированию, но думал, может, попутно смогу заниматься физикой}, -- рассказывает Столлман.

Изучая физику днём и хакерствуя ночью, Ричард старался достичь идеального баланса. Точкой опоры этих гиковских качелей были еженедельные встречи клуба народных танцев. Это была его единственная социальная связь с противоположным полом и вообще миром обычных людей. Однако ближе к концу первого курса в МТИ случилось несчастье -- Ричард повредил колено и не смог танцевать. Он думал, что это временно, и продолжал ходить в клуб, слушать музыку, болтать с друзьями. Но кончилось лето, колено всё ещё болело и нога плохо действовала. Тогда Столлман заподозрил неладное и забеспокоился. \enquote{Я понял, что лучше уже не станет, -- вспоминает он, -- и что я больше никогда не смогу танцевать. Меня это просто убило}.

Без общежития Гарварда и без танцев вселенная социальной жизни Столлмана тут же схлопнулась. Танцы -- единственное, что не только связывало его с людьми, но и давало реальную возможность встречаться с женщинами. Нет танцев -- нет свиданий, и это особенно сильно расстроило Ричарда.

\enquote{Большую часть времени я был совершенно подавлен, -- описывает Ричард этот период, -- я ничего не мог и не хотел, кроме хакерства. Полнейшее отчаяние}.

Он почти перестал пересекаться с миром, полностью уйдя в работу. К октябрю 1975 года он фактически забросил физику и учёбу в МТИ. Программирование из хобби превратилось в главное и единственное занятие всей жизни.

Сейчас Ричард говорит, что это было неизбежно. Рано или поздно зов сирен хакерства пересилил бы все остальные влечения. \enquote{В математике и физике у меня не получалось создать что-то своё, я даже не представлял себе, как это делается. Я только комбинировал уже созданное, и меня это не устраивало. В программировании же я сразу понял, как создавать новые вещи, и самое главное -- ты сразу видишь, что они работают, и что они полезны. Это приносит огромное удовольствие, и программировать хочется снова и снова}.

Столлман не первый, кто связывает хакерство с сильнейшим удовольствием. Многие хакеры Лаборатории ИИ тоже могут похвастаться заброшенной учёбой и недополученными степенями в математике или электротехнических областях -- только потому, что в чистом восторге программирования утонули все академические амбиции. Говорят, что Фома Аквинский своими фанатичными занятиями схоластикой доводил себя до видений и ощущения бога. Хакеры достигали схожих состояний на грани неземной эйфории после многочасовой концентрации на виртуальных процессах. Наверное, поэтому Столлман и большинство хакеров избегали наркотиков -- после часов двадцати хакерства они были всё равно что \enquote{под кайфом}.

Но ещё приятнее было чувствовать себя успешным в любимой области. Хакерство было естественным продолжением Столлмана. В детстве он привык учить уроки по ночам, так что ему нужно было совсем немного времени для сна. Долгие годы на положении изгоя приучили его работать в одиночку. А мощное математическое мышление позволяло ему заранее видеть трудности и изящно обходить их на полном ходу, тогда как многие другие хакеры начинали буксовать.

\enquote{У Ричарда феноменальный интеллект. Он очень ясно мыслит и конструирует стройные системы}, -- говорит Джеральд Сассмен, сотрудник Лаборатории и с 1985 года -- участник фонда свободного ПО. Оценив Столлмана по достоинству, он приглашал его работать в исследовательских ИИ-проектах в 1973 и 1975 году. Там требовались глубокие познания языка LISP, а также понимание, как вообще можно подойти к задаче. В результате проекта 1975 года появился ИИ, основанный на откате с учётом зависимостей -- проверке высказываний на противоречия и их решении в случае обнаружения таковых.

В свободное от официальной работы время Столлман занимался личными проектами. Одним из главных желаний Ричарда было усовершенствовать программную инфраструктуру Лаборатории, в частности -- доработать текстовый редактор TECO.

Эпопея Столлмана и TECO 70-х годов тесно связана с последующим основанием движения свободного софта. Это очень важный эпизод, так что стоит рассказать о нём подробнее. Прежде всего, нужно понять, что нынешняя лёгкость и простота написания текстов и программ на компьютерах была далеко не всегда. В 50-х, 60-х годах, когда компьютеры впервые появились в университетах, компьютерное программирование было очень абстрактным, долгим и утомительным занятием. Взаимодействие с компьютером проходило через огромные наборы перфокарт, на которых программист набирал программу, определённым образом пробивая отверстия на каждой перфокарте. Затем он передавал перфокарты системному администратору, тот вставлял их одну за одной в компьютер, ждал конца вычислений, получал перфокарты с набранным на них выводом программы и отдавал программисту. Этот процесс назывался \enquote{пакетной обработкой} и был очень медленным. К тому же, многие системные администраторы злоупотребляли своим положением, произвольно меняя очередь обработки или вовсе отказывая программистам. Именно из-за этого многие ранние хакеры возненавидели власть администраторов.

В 1962 году хакеры и научные сотрудники, привлечённые к проекту MAC в Лаборатории ИИ, взялись за эту проблему. Принцип разделения времени, который поначалу называли \enquote{похищением времени}, делал возможным выполнять на компьютере сразу несколько процессов. Для вывода результатов приспособили телетайпы, чтобы программист сразу мог читать нормальный текст, а не расшифровывать таблицы с пробитыми отверстиями. Программист сам набирал команды и читал построчный текстовый вывод.

На исходе десятилетия в интерфейсах произошёл качественный скачок -- в 1968 году Дуглас Энгельбарт, учёный Стэнфордского института, представил прототип графического интерфейса. Подключив к компьютеру телевизор и специальный манипулятор, названный \enquote{мышью}, Энгельбарт показал совершенно недосягаемый уровень интерактивности -- с компьютером можно было работать в режиме реального времени. Пользователь мог в любой момент добавить, изменить, удалить текст в любом месте экрана.

Это изобретение могло бы ещё два десятка лет пробиваться на рынок, но уже в 70-х годах телеэкраны вовсю начали вытеснять телетайпы, реализуя полноэкранное редактирование вместо построчного.

TECO (сокращение от слов \enquote{текстовый редактор и корректор}) стал одной из первых программ с поддержкой полноэкранного редактирования. Её сделали из старого построчного редактора, разработанного под телетайпы на PDP-6. \footnote{Согласно \textit{Jargon File}, TECO поначалу означал \enquote{Tape Editor and Corrector\enquote{, т.е. ленточный редактор, а не текстовый.}}}

Этот редактор был мощнее и удобнее прежних, но всё ещё недостаточно удобным и мощным. Например, чтобы создать документ и начать работать над ним, нужно было ввести целую серию команд и символ конца строки. TECO не умел реагировать на каждое нажатие клавиши, как делают сегодняшние текстовые редакторы. Опытные хакеры наловчились вводить множество команд одной строкой, но назвать это удобным язык не поворачивался. Столлман сравнивал это с \enquote{игрой в шахматы с завязанными глазами}. \footnote{Источник: Richard Stallman, \enquote{EMACS: The Extensible, Customizable, Display Editor,} AI Lab Memo (1979), \url{http://www.gnu.org/software/emacs/emacs-paper.html}}.

Для упрощения работы хакеры Лаборатории ИИ изменили программу так, чтобы она делила экран на 2 части, в одной отображался текст, в другой -- командная строка. Но даже с этой удобной функцией работа в TECO была далека от комфортной.

TECO был не единственной программой с поддержкой полноэкранного редактирования. Когда Столлман посетил лабораторию информатики Стэнфорда в 1976 году, он увидел там редактор, который назывался просто Е. В нём была функция обновления экрана при нажатии на определённую клавишу. В мире программирования редактор Е был одним из первых, что работал по принципу \enquote{что ты видишь, то и получаешь} (WYSIWYG). Этот принцип позволял работать с файлом напрямую, а не через посредника, которому нужно отдавать команды. \footnote{Richard Stallman, \enquote{Emacs the Full Screen Editor} (1987)}.

Впечатлённый этим изящным хаком, Столлман по возвращении в МТИ задумался о том, чтобы улучшить TECO в похожем ключе. В коде редактора он нашёл функцию Control-R, написанную Карлом Миккельсоном, которая вызывалась нажатием одноимённых клавиш. Эта функция переключала TECO в более интерактивный режим, но ограничивалась только 5 строками, и потому не давала заметной разницы с обычным режимом. Столлман отредактировал функцию так, чтобы можно было использовать весь экран, и расширил её одной небольшой, но очень мощной возможностью задавать произвольные команды TECO на произвольные комбинации клавиш -- то есть, добавил в редактор так называемые \enquote{макросы}. У опытных пользователей TECO уже скопились файлы с самыми актуальными и полезными командами, так что Ричард в своём хаке сделал возможным подключать и эти файлы в качестве макросов. В результате получился полноценный WYSIWYG-редактор, который можно было ещё и программировать. \enquote{Это был прорыв}, -- говорит Гай Стил, один из тогдашних хакеров Лаборатории. \footnote{См. предыдущее примечание.}

Столлман вспоминает, как внедрение макросов породило целый взрыв улучшений. \enquote{Каждый стремился автоматизировать свою работу наборами макросов. Ими постоянно обменивались и улучшали их, делая всё более мощными и универсальными. Наборы этих макросов мало-помалу становились самостоятельными системными программами}.\footnote{См. предыдущее примечание.}

Началась настоящая макросомания, даже сам редактор TECO стал восприниматься как придаток к макросам. \enquote{Мы уже считали его языком программирования, а не текстовым редактором}, -- рассказывает Столлман. Пользователи получали огромное удовольствие от написания макросов и их обсуждения. \footnote{См. предыдущее примечание.}

Спустя пару лет начали проявляться негативные последствия неконтролируемого \enquote{макросного взрыва} -- в частности, огромное количество несовместимостей. \enquote{Это было вавилонское столпотворение}, -- говорит Гай Стил. По его словам, эти последствия угрожали основе хакерской этики -- коллективной работе над программами, когда каждый может открыть и улучшить программу любого другого хакера. \enquote{Иногда лучший способ показать кому-то, как надо написать тот или иной код -- просто сесть и написать его самому}, -- объясняет Стил.

Возможность расширения функциональности через макросы стала мешать этому принципу. Стремясь облегчить свою работу, хакеры писали сложнейшие макросы для TECO, и чтобы начать работать за чужим терминалом, нужно было порой битый час сидеть вникать в то, что редактор, собственно, делает.

Это изрядно огорчало Ричарда, и он занялся решением проблемы. Он взял 4 разных комплекта макросов, проанализировал их и выстроил диаграмму популярности команд и их сочетаний. Также он начал наблюдать за работой других хакеров.

\enquote{Он смотрел на мой экран из-за плеча и расспрашивал о моих действиях}, -- вспоминает Стил.

Хотя наблюдения за чужой работой были в порядке вещей в Лаборатории, Стилу хорошо это запомнилось, потому что он был тихим замкнутым хакером и почти не общался со Столлманом. В итоге Ричард назвал работу Стила интересной и использовал её в своём решении.

\enquote{Я обычно говорю, что первые 0,001 процента решения той проблемы -- моя заслуга, а Столлман лишь довёл дело до конца}, -- смеётся Стил.

Столлман дал и название новому проекту: Emacs, сокращение от \enquote{editing macros} или \enquote{редактирование макросами}. Название отражало эволюционный скачок, который случился двумя годами ранее при изобретении макросов. Нашлось в названии место и чисто техническим соображениям удобства -- на компьютерах Лаборатории не было программ, название которых начиналось бы на \enquote{е}, поэтому достаточно было ввести одну эту букву, чтобы автодополнение вызвало Emacs. В очередной раз хакерская жажда эффективности оставила свой след.\footnote{См. предыдущее примечание.}

Конечно, не все и не сразу перешли на Emacs. Некоторые пользователи продолжали работать в TECO и расширять его функциональность макросами, но многие всё-таки выбрали Emacs. Он обеспечивал унифицированную платформу, от которой можно было отталкиваться. К тому же, его функции можно было расширять без необходимости переписывать старые функции, что резко сокращало количество проблем с совместимостью.

\enquote{С одной стороны, мы пытались создать единую систему команд, с другой -- не ограничивать её расширяемость, потому что программируемость была крайне важна для нас}, -- вспоминает Стил.

Но проблема несовместимости скоро снова дала о себе знать уже в других местах. Главной причиной её возвращения были тихие модификации кода отдельных экземпляров Emacs, о которых авторы не рассказывали остальным хакерам. В результате поведение этих экземпляров шло вразрез с поведением общей версии Emacs. Тогда Столлман решил внести в код специальную функцию, реализующую один из основных постулатов хакерской этики: пользователь получал право изменять код только в том случае, если он обязывался возвращать свои изменения в общую версию Emacs. Столлман назвал это \enquote{вступлением в коммуну Emacs}. Так же, как и TECO ранее, Emacs стал чем-то большим, чем просто компьютерной программой. Для Столлмана это был общественный договор. В документации 1981 года Ричард изложил его условия: \enquote{Emacs распространяется как общественный продукт, это значит, что все свои улучшения вы должны выслать мне для их внедрения и распространения в основной версии Emacs}.\footnote{Stallman (1979): \#SEC34.}

Оригинальный Emacs работал только на PDP-10, но скоро пользователи захотели работать в нём на других компьютерах. Снова произошёл взрывной рост, только теперь уже не макросов, а версий редактора, похожих на Emacs, с очень разным уровнем совместимости. Правила коммуны Emacs на них не распространялись, потому что те редакторы были написаны с нуля. Некоторые из них в названии иронично обыгрывали название оригинала: Sine (\enquote{Sine is not Emacs} или \enquote{Sine это не Emacs}), Eine (\enquote{Eine is not Emacs}), и даже Zwei (\enquote{Zwei was Eine initially} или \enquote{Zwei поначалу был Eine}). Оригинальный Emacs был полностью программируемым, но некоторые клоны предоставляли лишь ограниченный набор команд без возможности расширения, такие версии назывались \enquote{эрзац-Emacs}. Таким был Mince (\enquote{Mince is Not Complete Emacs} или \enquote{Mince это не совсем Emacs}).

В то время как Ричард разрабатывал Emacs в Лаборатории, хакерское сообщество потрясали тревожные известия. В 1979 году Брайан Рид встроил \enquote{тайм-бомбы} в Scribe, чтобы воспрепятствовать свободной раздаче редактора, и это ужаснуло Столлмана. \enquote{Он называл это самым фашистским поступком, который он только видел в своей жизни}, -- вспоминает сам Рид. Даже после того, как благодаря усилиям Рида появилась иерархия \textit{alt} в Usenet, поступок 1979 года продолжал омрачать его репутацию, по крайней мере, в глазах Столлмана. \enquote{Он говорил, что все программы должны быть бесплатными, и что взимание денег за ПО это преступление против человечества}.\footnote{В интервью 1996 года интернет-журналу \textit{MEME} Столлман назвал историю вокруг Scribe раздражающей, и даже не захотел называть Рида по имени. \enquote{Проблема в том, что никто не осудил и не наказал этого студента за то, что он сделал, -- сказал Столлман, -- в результате другие люди стали брать его поступок за пример}. \textit{MEME} 2.04, \url{http://memex.org/meme2-04.html}.}

Столлман был не в силах помешать действиям Рида, но он мог бороться с другими формами поведения, которые противоречили хакерской этике. Будучи главным сопровождающим разработки Emacs, Ричард начал вовсю пользоваться своим положением для продвижения своих идей. На последнем этапе долгого конфликта хакеров с управляющими Лаборатории, когда дело дошло до реального внедрения систем безопасности с паролями, Столлман забастовал, отказываясь раздать этим сотрудникам последнюю версию Emacs, пока они не откажутся от своих намерений. \footnote{Steven Levy, \textit{Hackers} (Penguin USA [paperback], 1984): 419.} Это был скорее жест, чем реальное ограничение, потому что этим сотрудникам ничего не мешало самим скопировать и установить последнюю версию Emacs. Но посыл Ричарда был очевиден всем.

\enquote{Многие на меня рассердились, говорили что я взял их в заложники, что я шантажирую их, и действительно так и было, -- рассказывает Столлман, -- я прибегнул к насилию над ними в ответ на их насилие надо мной}.\footnote{См. предыдущее примечание.}

Со временем Emacs стал платформой для продвижения хакерской этики. Уровень гибкости, заданный Столлманом, не только поощрял сотрудничество, но даже требовал его. Пользователи, которые отстранялись от сообщества Emacs, не получали важных изменений -- новых возможностей и исправлений ошибок. При этом история развития Emacs впечатляет. За 20 лет GNU Emacs научился быть электронной таблицей, базой данных, веб-браузером, личным психиатром, и даже простым текстовым редактором, и разработчики даже не думают удалять какие-то функции, заботливо перенося их в код новых версий. \enquote{Это наше видение идеального текстового редактора, -- говорит Столлман, -- его возможности восхищают и ужасают одновременно}.

Другие работники Лаборатории отзываются о редакторе куда милосерднее. Хэл Абельсон, аспирант МТИ, который работал с Сассменом в 70-х, и позже стал участником фонда свободного ПО, называет Emacs \enquote{совершенно гениальным творением}. Дав свободу программистам конструировать какие угодно функции, не нарушая работу системы, Столлман проложил путь к сложным программным проектам, над которыми работали огромные коллективы. \enquote{Структура редактора оказалась достаточно устойчивой, чтобы тысячи людей со всего мира развивали его согласно своим вкусам, -- сказал Абельсон, -- даже не знаю, бывало ли такое раньше.}\footnote{В этой главе я уделил больше внимания социальной значимости Emacs, нежели технической. Чтобы оценить уровень программной части, рекомендую прочитать заметку Столлмана 1979 года, особенно обратить внимание на раздел \enquote{Исследования в разработке программных инструментов} (\enquote{Research Through Development of Installed Tools} \#SEC27). Раздел не только понятен неспециалисту, но и показывает, как сильно связана политическая философия Столлмана с его философией программирования. Например:

\begin{quote}
Функциональность Emacs невозможно было создать заранее продуманным проектированием, потому что такой подход достигает целей, которые видны только поначалу, и устанавливает априорный ценз на присутствие каких-то функций в программе. Ни я, ни кто-либо другой не мог заранее представить себе все возможности Emacs, их можно только увидеть по мере их появления в реальности. Emacs это бесконечный процесс улучшений.
\end{quote}}

Гай Стил выражает похожее мнение. Он помнит Столлмана как \enquote{гениального программиста, который способен был без устали и ошибок генерировать огромное количество кода}. Хотя их характеры во многом не сходились, они успели поработать вместе, и Стил получил хорошее представление об интенсивности программирования Ричарда. Однажды Стил задумал создать функцию \enquote{приятной печати} для Emacs, чтобы при нажатии некоторой комбинации клавиш она преображала исходный код в компактную форму. Функция заинтересовала и Столлмана.

\enquote{Утром мы сели за терминал, я за клавиатуру, а Столлман -- рядом, чтобы говорить мне, что печатать}.

Около десяти часов подряд они занимались кодом, и за всё это время они не делали перерыва и не болтали о посторонних вещах. В итоге им удалось создать красивую функцию, которая занимала меньше 100 строк. \enquote{На клавишах были мои пальцы, но мне казалось, что мысли обеих наших голов перетекали на экран. Он говорил мне, что набирать, и я набирал}.

Выйдя из здания Техносквера на улицу, Стил удивился наступившей темноте. Он привык работать в марафонском стиле, а в этот раз всё прошло как-то слишком быстро, слишком интенсивно. Это было полное отключение от окружающих раздражителей и погружение в задачу. Стил признаётся, что такое мощное мышление Столлмана одновременно восхищает и пугает его. \enquote{Я впервые в жизни работал настолько интенсивно, почти яростно, и для меня это чересчур. Не хотелось бы снова через это пройти}.
